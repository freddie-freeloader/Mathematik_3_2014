
\documentclass[a4paper, 11pt, twosite, german, titlepage]{article}

% Packages 
\usepackage[german]{babel}
\usepackage[utf8]{inputenc}
\usepackage{textcomp}
\usepackage{lmodern}
\usepackage[T1]{fontenc}
\usepackage{amsmath}
\usepackage{amssymb}
\usepackage{amsthm}
\usepackage{mathtools}
\usepackage{wasysym} 
\usepackage{paralist}
\usepackage{varioref}
\usepackage{footnpag} % fussnoten seitenweise
\usepackage{fancyhdr}
\usepackage{newtxtext,newtxmath}
\usepackage[margin=3.5cm]{geometry}
\usepackage{polynom}
\usepackage[pdftex, pdftitle={Mathe III WiSe1415}, pdfauthor={getenv}]{hyperref}
\usepackage{pgf, tikz}


% % % % % % % % % % % % % %
\title{Mathematik III - Wintersemester 14/15}
%\author{names}
% % % % % % % % % % % % % % 

% % STYLE % %
\pagestyle{fancy}
\renewcommand{\sectionmark}[1]{%
\markboth {#1}{}}
\fancyhead[L]{Mathe III WiSe1415} 
\fancyhead[R]{\leftmark}	
% % % % % % % 
\renewcommand{\labelenumi}{\alph{enumi})}
% % % % % % %


% % % % % % % % % % % % % % %
% Selbstdefinierte Befehle  %
% bitte so eindeutig, wie   %
% moeglich halten!			%
% % % % % % % % % % % % % % %

% % Mengen % % % % % % % % % %
\newcommand{\N}{\mathbb{N}}
\newcommand{\Z}{\mathbb{Z}}
\newcommand{\Q}{\mathbb{Q}}
\newcommand{\R}{\mathbb{R}}
\newcommand{\C}{\mathbb{C}}

% % Kuerzel % % % % % % % % % 
\newcommand{\id}{\mathrm{ id }}
\newcommand{\Abb}{\mathrm{ Abb }}
\newcommand{\ggT}{\mathrm{ ggT }}
\newcommand{\kgV}{\mathrm{ kgV }}
\newcommand{\Eig}{\mathrm{ Eig }}
\newcommand{\rg}{\mathrm{ rg }}
\newcommand{\Grad}{\mathrm{ Grad }}
\renewcommand{\mod}{\;\mathrm{ mod }\;}
\newcommand{\gdw}{\Leftrightarrow}

% % Symbole % % % % % % % % %
\newcommand{\zerovec}{\mathcal{O}} %Nullvektor
\newcommand{\bigdot}{\,{\bullet}\,} %Verknuepfungsmal

% % "Konstrukte" % % % % % %
\newcommand{\aset}[1]{\left\{#1\right\}} % Menge
\newcommand{\abs}[1]{\left|#1\right|} % Betrag
\newcommand{\without}[1]{\backslash \{#1\}} 

% % Vektoren / Matrizen % % %
\newcommand{\aMatrix}[2]{\arraycolsep=1.8pt\def\arraystretch{0.8}\left(\begin{array}{#1}
#2 \end{array}\right)} % Matrix - erwartet als erstes Argument so viele c's, wie die Matrix Spalten hat, also etwa \aMatrix{cc}{1 & 2 \\ 3 & 4} fuer 2x2-Matrix
\newcommand{\sPer}[3]{\aMatrix{ccc}{1 & 2 & 3 \\ #1 & #2 & #3}}
% Definiere \vct ueber \aMatrix{c}{...}
\newcommand{\vct}[1]{\aMatrix{c}{#1}} % Vektor - benutze \\ fuer neue Spalte
% etwas andere Spaltenvektor Definition
\newcommand{\weirdvct}[2]{\aMatrix{c}{#1\\#2}} % Vektor - benutze \\ fuer neue Spalte
% % Textstuff % % % % % % % %
\newcommand{\textunderset}[2]{\begin{tabular}[t]{@{}l@{}}
#1 \\ #2 
\end{tabular}}
 % Vorsicht, zweites Element unten!
 \newcommand{\mathunderset}[2]{\begin{array}[t]{@{}l@{}}
 #1 \\ #2 
 \end{array}}
  % Vorsicht, zweites Element unten!
% % % % % % % % % % % % % % %


% !! muss direkt vor \begin{document} bleiben !!
\usepackage[pdftex]{hyperref}
\hypersetup{pdftitle={Mathematik 2 SS14}, pdfnewwindow, colorlinks, linkcolor=black}

\begin{document}

\maketitle
\tableofcontents
\newpage

% Unterdrueckt Einrueckungen nach Absatz - gehoert direkt vor den Inhalt
\setlength{\parindent}{0pt}
\setlength{\parskip}{1ex plus 0.5ex minus 0.2ex}



% % % % % % % % % % % % %
% 		INHALT			%
% Hier geht es los :))	%
% % % % % % % % % % % % %

 %1.1 - 1.5
\section[Algebraische Strukturen mit einer Verknüpfung]{Algebraische Strukturen mit einer Verknüpfung \\ HALBGRUPPEN, MONOIDE, GRUPPEN}

\subsection{Definition}

Sei $X \neq \varnothing$ eine Menge.

Eine \emph{Verknüpfung} oder (abstrakte) Multiplikation auf $X$ ist eine Abbildung
\[
{\begin{array}{lc@{}l}
	\bigdot:& X \times X 	&\rightarrow X \\
			& (a, b)		&\mapsto	 a \bigdot b
		\end{array}}
\]
$\underset{(ab)}{a\bigdot b}$ heißt \emph{Produkt} von $a$ und $b$, muss aber mit der üblichen Multiplikation von Zahlen nichts zu tun haben.

Beschreibung bei endlichen Mengen oft durch Multiplikationstafeln.

\subsection{Beispiel}

\begin{enumerate}
	
	\item
	$X = \aset{a, b}$ 
	\quad
	$\begin{array}{c|cc}
		\bigdot	& a	& b \\
		\hline 
		a		& b	& b \\
		b		& a	& a
	\end{array}$
	
	$(a \bigdot a) \bigdot a = b \bigdot a = a \\
	 a \bigdot (a \bigdot a) = a \bigdot b = b$
	\qquad  $\rightarrow$ nicht assoziativ
	
	
	\item
	$X = \Z^- \; (= \aset{0, -1, -2, \dots})$
	
	Die normale Multiplikation ist auf $\Z^-$ keine Verknüpfung!
	\\(zum Beispiel ist $(-2) \cdot (-3) = 6 \notin  \Z^-$)
	\\ Aber auf $X = \N, X = \Z$ oder $X = \aset{1}, X = \aset{0, 1}$
	
\end{enumerate}

\subsection{Definition}

Sei $H \neq \varnothing$ eine Menge mit Verknüpfung.

$(H, \bigdot)$ heißt \emph{Halbgruppe}, falls gilt:
\[ \tag{Assoziativgesetz (AG)}
\forall a, b, c \in H \::\: (a \bigdot b) \bigdot c = a \bigdot (b \bigdot c)
\]

\subsection{Bemerkung}

AG sagt aus: bei endlichen Produkten ist die Klammerung irrelevant, z.B.

$(a \cdot b) \cdot (c \cdot d) = ((a \cdot b) \cdot c) \cdot d = (a \cdot (b \cdot c)) \cdot d$ \:(usw.)

Deshalb werden Klammern meistens weggelassen.

Die Reihenfolge der Elemente ist i.A. relevant!

\subsection{Beispiel} \label{bspHalbgruppe}

\begin{enumerate}
	
	\item
	$(\N, \bigdot), (\Z, \bigdot), (\Q, \bigdot), (\R, \bigdot)$ \footnote{$\bigdot$ normale Multiplikation}
	sind Halbgruppen.
	
	Ebenso $(\N, +), (\Z, +), (\Q, +), (\R, +)$ \footnote{$+$ normale Addition} 
	
	
	\item
	$(\Q \without{0}, :)$ \footnote{$:$ normale Division} ist \emph{keine} Halbgruppe, denn z.B.
	$\begin{array}{ccc}
	(12 : 6) : 2	&=& 1 \\
	12 : (6 : 2)	&=& 4
	\end{array}$
	
	\item
	vgl. Vorlesung Theoretische Informatik
	
	$A \neq \varnothing$ endliche Menge (''Alphabet'')
	
	$A^+ = \cup_{n \in N} A^n$ = Menge aller endlichen Wörter über $A$
	\\(z.B. $A=\aset{a, b}$, dann ist z.B. $\underset{aab}{\underbrace{(a, a, b)}} \in A^3$)
	
	Verknüpfung: Konkatenation (Hintereinanderschreiben)
	\\z.B. $aab \bigdot abab = aababab$
	
	$A^* = A^+ \cup \aset{\lambda}$ 
	\quad$\lambda$ (oder $\epsilon$) ist das leere Wort 
	
	Es gilt: 
	$\lambda \cdot w = w \cdot \lambda = w \;\forall w \in A^*$
	
	$(A^+, \bigdot), (A^*, \bigdot)$ \emph{Worthalbgruppe} über $A$
	
	
	\item
	$M \neq \varnothing$ Menge, Abb($M,M$): Menge aller Abbildungen $M \to M$
	mit $\circ$ (Komposition) ist Halbgruppe.
	
	\item (WICHTIG)
	
	$n \in \N,\; \Z_n = \aset{0, 1, \dots, n-1}$
	
	Verknüpfung:
	$\begin{array}{c@{\;:\;}r@{\;:=\;}r}
		\oplus	& a \oplus b	& (a+b) \mod{n} \\
		\odot	& a \odot b		& (a\cdot b)\mod{n}
	\end{array}$
	
	$(\Z_n, \oplus), (\Z_n, \odot)$ sind Halbgruppen.
	
\end{enumerate}










 %1.6 - 1.16
 %1.6 - 1.16
 % % % % % % % % % % % % % % % % % % % % % % % % % % % % %
 
 % % % 1.6
 \subsection[Definition: kommutative Halbgruppe]{Definition}
 
 Eine Halbgruppe $(H, \bigdot)$ heißt \emph{kommutativ}, falls gilt:
 %
 \[ \tag{Kommutativgesetz, KG}
 \forall a, b \in H\::\: a \cdot b = b \cdot a \]
 
 % % % 1.7
 \subsection{Beispiel}
 
 Beispiele \ref{bspHalbgruppe} a), e) sind kommutative Halbgruppen.
 \\ (hallo $\neq$ ollah, ab $\neq$ ba, Worthalbgruppe nicht kommutativ) 
 
 % % % 1.8
 \subsection[Definition: Unterhalbgruppe]{Definition}
 
 Sei $(H, \bigdot)$ Halbgruppe, $\varnothing \neq U \subseteq H$
 
 $U$ heißt \emph{Unterhalbgruppe} von $H$, falls $u \cdot v \in U \; \forall u, v \in U$ gilt.
 
 $(U, \odot)$ ist dann selbst Halbgruppe.
 
 % % % 1.9 
 \subsection{Beispiel}
 
 $(\Z, +)$ Halbgruppe
 
 $G =$ Menge aller gerade ganzen Zahlen $\subseteq \Z$
 \\$(G, +)$ ist Unterhalbgruppe von $(\Z, +)$
 
 $U =$ Menge aller ungerade Zahlen $\subseteq \Z$
 \\$(U, +)$ ist \underline{keine} Unterhalbgruppe!
 
 % % % 1.10
 \subsection[Lemma: Eins eindeutig]{Lemma} \label{nullelemEindeutig}
 
 \emph{Eindeutigkeit des neutralen Elements:}
 
 Sei $(H, \bigdot)$ Halbgruppe, $e_1, e_2 \in H$ mit 
 $^{(*)} e_1 \cdot x = x \cdot e_1 = x$ und $^{(**)} e_2 \cdot x = x \cdot e_2 = x \; \forall x \in H$

 Dann ist $e_1 = e_2$
 
 \begin{proof}
 $e_1 \stackrel{(**)}{=} e_1 \cdot e_2 \stackrel{(*)}{=} e_2$
 \end{proof} 
 
 % % % 1.11
 \subsection[Definition: Monoid]{Definition}
 
 Eine Halbgruppe $(H, \bigdot)$ heißt \emph{Monoid}, falls $e \in H$ existiert mit $e\cdot x = x \cdot e = x \; \forall x \in H$
 
 $e$ heißt \emph{neutrales Element} / Einselement / Eins in $H$.
 \\Schreibweise: $(H, \bigdot, e)$
 
 Für \textunderset{additive}{multiplikative} Verknüpfung oft \textunderset{0}{1} für $e$ (Nullelement)
 
 Nach \ref{nullelemEindeutig} ist das neutrale Element eindeutig!
 
 % % %  1.12
 \subsection{Beispiele}
 
 \begin{enumerate}

	\item
	$(\N, \bigdot)$ Monoid mit $e=1$
	\\$(\N, +)$ kein Monoid
	\\$(\N_0, +)$ Monoid mit $e=0$
	\\$(\Z, +), (\Q, +), (\R, +)$ Monoide mit $e=0$
	\\$(\Z, \bigdot), (\N_0, \bigdot), (\Q, \bigdot), (\R, \bigdot)$ Monoide mit $e=1$
	
	\item
	$(\Abb(M, M), \circ)$ Monoid, $e = \id$
	
	\item
	$(\Z_n, \oplus)$ Monoid, $e=0$
	\\ $(\Z_n, \odot)$ Monoid, $e=1$
	
	\item
	$(A^*, \bigdot)$ Monoid, $e= \lambda$ (hallo$\lambda$ = $\lambda$hallo = hallo)

 \end{enumerate}
 
 % % % 1.13
 \subsection[Definition: Untermonoid]{Definition}
 
 Sei $(M, \bigdot, e)$ Monoid. Eine Teilmenge $\varnothing \neq U \subseteq M$ heißt \emph{Untermonoid} von $M$, falls $U$ mit $\bigdot$ selbst ein Monoid mit neutralem Element $e$ ist (also $e \in U$)

% % %  1.14
 \subsection[Lemma: Inverses eindeutig]{Lemma} \label{neutElemEindeutig}
 
  \emph{Eindeutigkeit des inversen Elements:}
  
 Sei $(H, \bigdot, e)$ Monoid und es gebe zu jedem Element $h \in H$ Elemente $x, y \in H$ mit ${h \cdot x \stackrel{(*)}{=} e \stackrel{(**)}{=} y \cdot h}$.
 
 Dann ist $x = y$
 
 \begin{proof}
 	$y = y \cdot e \stackrel{(*)}{=} y \cdot (h \cdot x) \stackrel{(\text{AG})}{=} (y \cdot h) \cdot x \stackrel{(**)}{=} e \cdot x = x$
 \end{proof}
 
 % % % 1.15
 \subsection[Definition: Gruppe, Inverse, Ordnung]{Definition}
 
 {\renewcommand{\labelenumi}{(\roman{enumi})}
 \begin{enumerate}
 
 	\item
 	$(H, \bigdot, e)$ Monoid, $h \in H$
 	
 	Falls ein $x \in H$ existiert mit $hx = xh = e$, so nennt man $h$ \emph{invertierbar} und $x$ das \emph{Inverse} zu $h$, bez. $h^{-1}$
 	(bei additiven Verknüpfungen oft auch $-h$)
 	
 	Nach \ref{neutElemEindeutig} ist $h^{-1}$ eindeutig bestimmt!
 	
 	Es gilt: $e$ ist immer invertierbar, $e^{-1} = e$
 	
 	\item
 	Ein Monoid $(G, \bigdot, e)$ heißt \emph{Gruppe}, falls \underline{jedes} Element in $G$ invertierbar ist.
 	
 	\item
 	Für eine endliche Gruppe $G$ heißt die Anzahl der Elemente in $G$ die \emph{Ordnung} von $G$, $\abs{G}$
 	
 \end{enumerate}
 }
 
 % % % 1.16
 \subsection{Bemerkung}
 
 $(H, \bigdot, e)$ Monoid.
 
 Sei $G$ die Menge aller invertierbaren Elemente von $H$, dann ist $(G, \bigdot, e)$ eine Gruppe.
 
 Es gilt: $e$ invertierbar $(e^{-1} = e)$
 
 und falls $g$ invertierbar, dann ist auch $g^{-1}$ invertierbar: $(g^{-1})^{-1} = g$
 
 falls $g, h$ invertierbar, dann auch $g \cdot h$: \quad
 $(g \cdot h)^{-1} = h^{-1} \cdot g^{-1} $
 
 
 
 
 
 
 

 %1.17 - 1.19
 %1.17 - 1.19
 % % % % % % % % % % % % % % % % % % % % % % % % % % % % %

% % % 1.17
\subsection{Beispiele}

\begin{enumerate}
	
	\item
	$(\N_0, +, 0)$ ist keine Gruppe aber $(\Z, +, 0), (\Q, +, 0), (\R, +, 0)$ sind Gruppen.
	
	\item
	$(\Z, \bigdot, 1)$ ist keine Gruppe.
	\\ Die Menge der invertierbaren Elemente ist $\aset{1, -1}$, diese bilden eine Gruppe.
	
	\item
	$(\Q, \bigdot, 1)$ ist keine Gruppe, aber $(\Q \without{0}, \bigdot, 1), (\R \without{0}, \bigdot, 1)$ sind Gruppen.
	
	\item
	$A^*$ ist keine Gruppe, nur $\lambda$ ist invertierbar.
	
\end{enumerate}

% % % 1.18
\subsection{Beispiele}
\label{gruppenbeispiele}
\begin{enumerate}
	
	\item
	$(\Z_n, \oplus, 0)$ ist Gruppe (was ist das Inverse zu $x \in \Z_n$? Siehe PÜ1, A9)
	
	\item
	Sei $n \geq 2$. $(Z_n, \odot, 1)$ ist Monoid aber keine Gruppe. 
	
	Wann ist ein Element aus $Z_n$ invertierbar bezüglich $\odot$?
	
	$\begin{array}{lcl}
	z \in \Z_n \text{ invertierbar} 
		&\gdw&
		\exists x \in \Z_n : z \odot x = 1 \\
		&\gdw&
		 \exists x \in \Z: (z \cdot x) \mod{n} = 1 \\
		&\gdw&
		\exists x, q \in \Z: z \cdot x = q \cdot n + 1 \\
		&\gdw&
		\exists x, q \in \Z: z \cdot x + (-q \cdot n) = 1 \\
		&\stackrel{\text{Mathe I}}{\gdw}&
		\ggT(z, n) = 1 
	\end{array}$
	
	also sind nur zu $n$ teilerfremde Elemente invertierbar!
	
	(vgl. $(Z_6, 0, 1)$: $0, 2, 3, 4$ nicht invertierbar, $1, 5$ invertierbar)
	
	\underline{Bezeichnung:}
	\[Z_n^* = \aset{z \in \Z_n \;|\; \ggT(z, n) = 1}\]
	%
	ist Gruppe bezüglich $\odot$ (vgl. Bemerkung \ref{invbarkeitMonoid}) mit Ordnung $\abs{Z_n^*}=\varphi(n)$ (''phi von $n$'', Eulersche $\varphi$-Funktion) = Anzahl aller $z \in \N$, die teilerfremd zu $n$ sind und $1 \leq z \leq n$.
	
	$\varphi(3) = 2, \varphi(4)=2, \varphi(7) = 6$

	Wie berechnet man das Inverse von $z \in \Z_n^*$?
	
	Mathe I, Erweiterter Euklidischer Algorithmus (WHK, S. 80/81) liefert zu $z$ und $n$ $(\ggT(z, n) = 1)$ Zahlen $s, t \in \Z$ mit 
	
	$\begin{array}{cl}
	& z \cdot s + n \cdot t = 1 \\
	\Rightarrow & (z \cdot s) \mod n = 1 \\
	\Rightarrow & (z^{-1}) = s \mod n
	\end{array}$
	
	Beispiel:
	
	$n=8$: $(\Z_8, \odot)$, $z=5$ ist invertierbar, $\ggT(8, 5) = 1$
	
	EEA: $5 \cdot (-3) + 8 \cdot 2 = 1 \Rightarrow z^{-1} = -3 \mod 8 \Rightarrow z^{-1} = 5$


	\item
	$\Abb(M, M)$: invertierbare Elemente sind genau die \emph{bijektiven} Abbildungen auf $M, \mathrm{Bij}(M)$ (Mathe I)
	
	Speziell: $M = \aset{1, 2, \dots, n}$, dann heißt $\mathrm{Bij}(M)$ die symmetrische Gruppe von Grad $n$, $S_n$
	
	$\abs{S_n} = n!$, Elemente heißen Permutationen.
	
	\underline{Bsp:} $n=2$
	\[S_2= \aset{
	\underbrace{\aMatrix{cc}{
	1 & 2 \\ 
	1 & 2}}_{\id},
	\aMatrix{cc}{
	1 & 2 \\
	2 & 1}}\]
	%
	$n=3$
	%
	\[S_3 = \aset{
	\underbrace{\sPer{1}{2}{3}}_{\id},
	\sPer{1}{3}{2},
	\sPer{3}{2}{1},
	\sPer{2}{1}{3},
	\sPer{2}{3}{1},
	\sPer{3}{1}{2}
	}\]
	
	$\pi = \sPer{1}{3}{2}, \varrho = \sPer{3}{1}{2} \in S_3$
	
	$\pi \circ \varrho = \sPer{2}{1}{3},\; \varrho \circ \pi = \sPer{3}{2}{1}$ (nicht kommutativ!)
	
	$\pi^{-1} = \sPer{1}{3}{2} = \pi,\; \varrho^{-1}= \sPer{2}{3}{1}$
\end{enumerate}

% % % 1.19
\subsection[Satz: Gleichungen lösen in Gruppen]{Satz (Gleichungen lösen in Gruppen)} \label{gLösenGruppen}

	Sei $G$ Gruppe, $a, b \in G$
	
	{\renewcommand{\labelenumi}{(\roman{enumi})}
	\begin{enumerate}
		\item
		Es gibt genau ein $x \in G$ mit $ax = b$ (nämlich $x = a^{-1}b$)
		\item
		Es gibt genau ein $y \in G$ mit $ya = b$ (nämlich $y=ba^{-1}$)
		\item
		Ist $\underset{ya=yb}{ax=bx}$ für ein $\underset{y \in G}{x \in G}$, dann gilt $a=b$ (Kürzungsregel)
	\end{enumerate}
	
	\begin{proof}
	\begin{enumerate}
		\item
		\begin{itemize}
			\item
			$x=a^{-1}$ ist Lösung (prüfe $ax=b$):
			
			$a \cdot \underbrace{a^{-1}b}_{x} \stackrel{\text{AG}}{=}(a \cdot a^{-1}) \cdot b = e \cdot b = b$
			
			\item Es gibt genau eine Lösung:
			
			Es gelte $ax=b$
			\\ $\Rightarrow x = ex = (a^{-1}a)x\stackrel{\text{AG}}{=}a^{-1}(ax) = a^{-1}b$
		\end{itemize}
		
		\item
		analog
		
		\item
		Multipliziere von \textunderset{rechts}{links} mit $\underset{y^{-1}}{x^{-1}}$
	\end{enumerate}
	\end{proof}
	
	
	}

 %1.20 - 1.24
 
 \subsection{Beispiel}
 $
 \aMatrix{ccc}{
 	1 & 2 & 3 \\
 	2 & 1 & 3
 	}\circ
 	x =
 \aMatrix{ccc}{
 	1 & 2 & 3 \\
 	3 & 1 & 2
 	}
 	$
 - Was ist $x$?
 
 $ a \cdot x = b \Leftrightarrow x=a^{-1} \cdot b$
 
 $
 \lambda =
  \aMatrix{ccc}{
  	1 & 2 & 3 \\
  	2 & 1 & 3
  }^{-1}\circ
 \aMatrix{ccc}{
 	1 & 2 & 3 \\
 	3 & 1 & 2
 } = 
  \aMatrix{ccc}{
  	1 & 2 & 3 \\
  	2 & 1 & 3
  }\circ
  \aMatrix{ccc}{
  	1 & 2 & 3 \\
  	3 & 1 & 2
  } =
  \aMatrix{ccc}{
  	1 & 2 & 3 \\
  	3 & 2 & 1
  }
  $

\subsection{Definition}

$(G,\cdot)$ Gruppe, $\varnothing \neq U \subseteq G$ Teilmenge.

$U$ heißt \emph{Untergruppe} von $G$ ($U\leqslant G$), falls $u$ bzgl. $\cdot$ selbst eine Gruppe ist.

Insbesondere gilt dann:
$\forall u,v \in U$ ist $u \cdot v \in  U$.\\
$e$ von $G$ ist auch neutrales Element von $u$.\\
Inversen in $U$ sind die gleichen wie in $G$.

Angenommen $e$ neutrales Element in $G$, aber $f$ neutrales Element in $U$, $f^{-1}$ Inverses von $f$ in $G$.\\
Dann ist $f^{-1} \cdot f= f \cdot f^{-1} = e$ und $f\cdot f = f$.

$\Rightarrow f = e \cdot f = (f^{-1} \cdot f) \cdot f = f^{-1} \cdot (f \cdot f) = f^{-1} \cdot f = e$

\subsection{Beispiele}
\begin{enumerate}
	\item
	$(\Z, +) \leqslant (\Q, +) \leqslant (\R, +)$
	
	\item
	$(\{-1,1\}, \cdot) \leqslant (\Q \backslash \{0\}, \cdot) \leqslant (\R \backslash \{0\}, \cdot)$
	
	\item
	$(e, \cdot)$ ist Untergruppe jeder beliebigen Gruppe mit Verkn"upfung $\cdot$ und neutralem Element $e$.
	
	\item
	$\pi =
	\aMatrix{ccc}{
		1 & 2 & 3 \\
		2 & 1 & 3
		}
	\in S_3
	$,
	$\pi = \pi^{-1}, \pi^{-1} \circ \pi = \text{id} =
	\aMatrix{ccc}{
		1 & 2 & 3 \\
		1 & 2 & 3
		}
	$
	
	$\Rightarrow (\pi, \text{id})\leqslant S_3$
	
	\subsection{Satz und Definition}
	$G$ Gruppe, $U \leqslant G$
	\begin{enumerate}
		\item
		Durch $x \sim y \Leftrightarrow x \cdot y^{-1} \in U$\\
		TODO "Das muss unter die obere Zeile: bei additiver Verkn"upfung: $x + (-y) \in U (x-y \in U)$\\
		wird auf $G$ eine "Aquivalenzrelation definiert
		
		\underline{Beweis}:
		
		$\sim$ ist reflexiv: $x \sim x$ gilt $\forall x \in G$, denn $x \cdot x^{-1}= e \in U \; \checkmark$
		
		$\sim$ ist symmetrisch: $x \sim y \Rightarrow y \sim x$\\
		 Sei $x \sim y$, also $x \cdot y^{-1} \in U$ (zzg.: $y \sim x$, also $y \cdot x^{-1} \in U$) dann ist $y \cdot x^{-1} = (x \cdot y^{-1})^{-1} \in U$, da auch $x \sim y \Leftrightarrow x \cdot y^{-1} \in U$.
		 
		$\sim$ ist transitiv: $x \sim y, y \sim z \Rightarrow x \sim z$\\
		Sei $x \sim y$, also $x \cdot y^{-1} \in U$ und $y \sim z$, also $y \cdot z^{-1} \in U$ (zzg.: $x \sim z$, d.h. $x\cdot z^{-1} \in U$)
		
		$x \cdot z^{-1} = (\underbrace{x \cdot y^{-1}}_{\in U}) \cdot (\underbrace{y \cdot z^{-1}}_{\in U}) \in U$, also $x \sim z$.
		
		\item
		F"ur $x \in G$ ist $Ux = \{u \cdot x | u \in U\}$ die "Aquivalenzklasse von $x$ bzgl. $\sim$ und heißt Rechtsnebenklasse von $U$ in $G$.
		
		Also (Eigenschaften von "Aquivalenzklassen siehe Mathe I):
		\begin{enumerate}
			\item
			$Ux = Uy \Leftrightarrow x \sim y$, also $x \cdot y^{-1} \in U$
			\item
			$x,y \in G$, dann ist entweder $Ux = Uy$ oder $Ux \cap Uy = \varnothing $
		\end{enumerate}
		
		Beweis:
		\begin{enumerate}
			\item
			Seit $x \sim y \Rightarrow y \sim x \Rightarrow y \cdot x^{-1} \in U \Rightarrow y=y(x^{-1} \cdot x) = \underbrace{(y \cdot x^{-1})}_{\in U}x \in Ux$
			\item
			Sei $y \in Ux$, dann zeige: $x \sim y$ \\
			$y \in Ux \Rightarrow y = u \cdot x$ f"ur ein $u \in U\\
			\Rightarrow x \cdot y^{-1} = x \cdot (ux)^{-1} = x \cdot x^{-1} \cdot u^{-1} = u^{-1} \in U$\\
			Es wurde gezeigt, dass $x \sim y$ gilt.
		\end{enumerate}
	\end{enumerate}
\end{enumerate}

\subsection{Beispiel}

$G = (\Z, +), 3 \Z = \{\dots, -3,0,3,6,\dots\}\\
U = (3 \Z, +) \leqslant G$ ("UA, Blatt 2)\\
Inverses zu y in $(\Z,+)$ ist $-y$.\\
$x \sim y \Leftrightarrow \underbrace{x \cdot y ^{-1} \in U}_{\text{bzw.:}\, x-y \in U}$ 

$ x=0 : U+0 = \{u+0 | u \in U\}= \{\dots, -3,0,3,6,\dots\}\\
   x=1 : U+1 = \{u+1 | u \in U\} = \{\dots\}$
 %1.25 - 1.31
 %1.25 - 1.31
 % % % % % % % % % % % % % % % % % % % % % % % % % % % % %

\subsection[Lemma: Mächtigkeit von Untergruppen]{Lemma}
\label{maechtigkeituntergruppen}
$G$ Gruppe, $U$ endliche Untergruppe von $G$, $x \in G$

Dann ist $\abs{U} = \abs{Ux}$

\subsubsection*{Beweis}

$\begin{array}[b]{rrcl}
\Abb \; \varphi: 	& U &\to& Ux \\
				& u &\mapsto& ux
\end{array}$
\\ist surjektiv
und injektiv 
(falls $u_1x=u_2x$, dann ist $u_1 = u_2$ (Satz \ref{gLösenGruppen} (iii), Kürzungsregel))

Also ist $\varphi$ bijektiv, also $U, Ux$ gleich mächtig.

\subsection[Theorem: Satz von Lagrange]{Theorem (Satz von Lagrange)} \label{lagrange}

$G$ endliche Gruppe, $U \leqslant G$

Dann gilt $\abs{U}$ ist Teiler von $\abs{G}$ und $q = \frac{\abs{G}}{\abs{U}}$ ist die Anzahl der Rechtsnebenklassen von $U$ in $G$

\subsubsection*{Beweis}

Seien $Ux_1, \dots, Ux_q$ die $q$ verschiedenen Rechtsnebenklassen von $U$ in $G$

\[\begin{split}
\text{Mathe I \& \ref{rechtsnebenklassen}}\Rightarrow G = \bigcup_{i=1}^{q}Ux_i \text{ (disjunkte Vereinigung der Äquivalenzklassen)}
\\\Rightarrow \abs{G} = \sum_{i=1}^{q}\underbrace{\abs{Ux_i}}_{\abs{U}} \stackrel{\ref{maechtigkeituntergruppen}}{=} q \cdot \abs{U} \end{split}\]


\subsection[Definition: Potenzen]{Definition}

$(G, \bigdot, e)$ Gruppe, $a \in G$

Definiere
$\begin{array}[t]{lcll}
a^0	&:=& e \\
a^1 &:=& a \\
a^m &:=& a^{m-1} \cdot a & \text{für } m \in \N \\
a^m &:=& (a^{m})^{-1} 	& \text{für } m \in \Z^-
\end{array}$

(Potenzen von $a$)

Bei additiver Schreibweise:
$\begin{array}[t]{lcl}
0 \cdot a &=& e \\
1 \cdot a &=& a \\
m \cdot a &=& 
\begin{cases}
(m-1) \cdot a + a & \text{für } m \in \N \\
(-m) \cdot (-a) 	& \text{für } m \in \Z^-
\end{cases}
\end{array}$

\subsection[Satz: Potenzgesetze]{Satz}
\label{potenzgesetze}
$G, a$ wie oben

{\renewcommand{\labelenumi}{(\roman{enumi})}
\begin{enumerate}
	\item
	$(a^{-1})^m = (a^m)^{-1} = a^{-m} \quad \forall m \in \Z$
	
	\item
	$a^m \cdot a^n = a^{m+n} \quad \forall m, n \in \Z$
	
	\item
	$(a^m)^n=a^{m \cdot n} \quad \forall m, n \in \Z$
\end{enumerate}

\subsubsection*{Beweis}
\begin{enumerate}
	\item
	$m \in \N: (a^{-1})^m \cdot a^m = \underbrace{a^{-1} \cdot a^{-1}\cdot \dots \cdot a^{-1}}_{m \text{ mal}} \cdot \underbrace{a \cdot \dots \cdot a \cdot a}_{m \text{ mal}} = e$
	
	$\Rightarrow (a^{-1})^m = (a^m)^{-1}$ (Inverses von $a^m$)
	
	nach Definition ist $a^{-m} = (a^{-1})^m$
	\\ $\Rightarrow $ (i) gilt $\forall m \in \N$
	
	$m = 0: \; e = e = e \checkmark$
	
	$m \in \Z^-$: dann ist $-m \in \N$
	\\ Wende den bewiesenen Teil an auf $a^{-1}$ statt $a$ und $-m$ statt $m$, Behauptung folgt.
	
	\item[(ii), (iii)]
	per Induktion und mit (i) \qed
\end{enumerate}}

\subsection[Satz und Definition: Ordnung, zyklische Gruppe]{Satz und Definition}

$G$ endliche Gruppe, $g \in G$

{\renewcommand{\labelenumi}{(\roman{enumi})}
\begin{enumerate}
	\item
	Es existiert eine kleinste natürliche Zahl $n$ mit $g^n = e$, diese heißt die \emph{Ordnung} $\mathrm{o}(g)$ von $G$ 
	
	\item
	Die Menge $\aset{g^0 = e, g^1 = g, g^2, \dots, g^{n-1}}$ ist eine Untergruppe von $G$, die von $g$ erzeugte zyklische Gruppe $<g>$
	
	Es gilt $\mathrm{o}(g) = \abs{<g>} = n \text{ teilt } \abs{G}$
	
	\item
	$g^{\abs{G}} = e$
	
	Bemerkung: Eine endliche Gruppe heißt \emph{zyklisch}, falls sie von einem Element erzeugt werden kann.
\end{enumerate}

\subsubsection*{Beweis}

\begin{enumerate}
	\item
	$G$ endlich $\Rightarrow \exists i, j \in \N, i > j$ mit $g^i=g^j$ \quad(Schubfachschluss \emph{-Editor})
	
	Dann ist $g^{i-j} \stackrel{\ref{potenzgesetze} ii)}{=} g^i \cdot g^{-j} \stackrel{\ref{potenzgesetze}}{=}\underbrace{g^i}_{=g^j} \cdot (g^j)^{-1} = e$
	
	\item
	Das Produkt zweier Elemente aus $<g>$ liegt wieder in $<g>$
	
	Neutrales Element ist $g^0 = e$
	
	Inverses Element zu $g^i$ ist $(g^i)^{-1} = g^{n-i}$
	
	$\Rightarrow <g> \leqslant G$
	
	\item
	Satz von Lagrange (\ref{lagrange}): $n = \mathrm{o}(g)= \abs{<g>} \;\Big|\; \abs{G}$
	
	Also ist $\abs{G} = n \cdot k$ für ein $k \in \N$
	
	$g^{\abs{G}} = g^{n \cdot k} = (g^{n})^k = e^k = e$ 
\end{enumerate}} \qed


\subsection{Beispiel}

$(\Z_3 \without{0}, \odot, 1)$

\begin{itemize}

	\item[$g=1$:]
	$<1> = \aset{g^0 = 1^0 = 1},\; \mathrm{o}(1) = 1$
	
	\item[$g=2$:]
	$<2> = \aset{g^0 = 1, g^1 = 2},\; \mathrm{o}(2)=2$

\end{itemize}

$(\Z_5 \without{0}, \odot, 1)$

\begin{itemize}
	\item[$g=2$:]
	$<2> = \aset{2^0 = 1, 2^1 = 2, 2^2 = 4, 2^3 = 3}, \; \mathrm{o}(2)=4$
\end{itemize}

\subsection{Korollar}

{\renewcommand{\labelenumi}{(\roman{enumi})}
\begin{enumerate}
	\item
	\underline{Satz von Euler}
	
	Sei $n\in \N, a \in \Z, \ggT(a, n) = 1$	
	\\Dann ist \[a^{\varphi(n)} \equiv 1 (\mod n)\]
	
	\item
	\underline{Kleiner Satz von Fermat}
	
	Ist $p$ eine Primzahl, $a \in \Z, p \nmid a$, dann gilt
	\[a^{p-1} \equiv 1 (\mod p)\]
	
\end{enumerate}}










 %1.32 - 2.4
 
 \subsection{Beweis}
 \begin{enumerate}
 \item
 Wir k"onnen annehmen, dass $1 \leq a < n $ (denn $a^{\varphi(n)} \mod{n}= (a \mod{n})^{\varphi(n)})$\\
 wegen ggT$(a,n)=1$ ist $a \in \Z^*_n$, das ist eine endliche Gruppe.
 
 $\stackrel{\ref{zyklischeGruppe}(iii)}{\Rightarrow} a^{|\Z^*_n|}=1(=e) \hspace{1.5cm} a \odot a \odot \dots$\\
 $\Rightarrow a^{\varphi(n)} \equiv 1 (\mod{n}) \hspace{0.7cm} a \cdot a \cdot \dots $ 
 \item
 Folgt aus (i) $(n=p,\; \varphi (p) = -1)$

\end{enumerate}
\section{Algebraische Strukturen mit 2 Verkn"upfungen: Ringe und K"orper}

\subsection[Definition: Ring]{Definition} \label{ring}
Sei $R \neq \varnothing$ eine Menge mit zwei Verkn"upfungen $+$ und $\bigdot$.
{\renewcommand{\labelenumi}{(\roman{enumi})}
\begin{enumerate}
	\item
	Wir nennen $(R, +, \cdot)$ einen \emph{Ring}, falls gilt:
	%TODO 1), 2) etc
	{\renewcommand{\labelenumi}{\arabic{enumi})}\begin{enumerate}
		\item
		$(R,+)$ ist eine abelsche Gruppe (Eselsbr"ucke: KAIN)\\
		Das neutrale Element bezeichnen wir hier mit $0$, das zu $a \in \R$ Inverse mit $-a$ (schreibe auch $a-b$ f"ur $a+(-b)$.
		\item
		$(R,\cdot)$ ist eine Halbgruppe.
		\item
		Es gelten die Distributivgesetze:\\
		$a\cdot (b+c) = (a \cdot b) + (a \cdot c) = ab + ac\\
		(a+b) \cdot c - (a \cdot c) + (b \cdot c) = ac = bc$ \qquad $\forall a, b, c \in R$
	\end{enumerate}}
	\item
	Ein Ring $(R,+, \cdot)$ heißt \emph{kommutativ} falls $\cdot$ ebenfalls kommutativ ist, also falls ${\forall a,b \in \R: a \cdot b = b \cdot a}$
	\item
	Ein Ring $(R,+, \cdot)$ heißt \emph{Ring mit Eins}, falls $(R, \cdot)$ ein Monoid ist mit neutralen Element $1\neq 0$ \;($\forall a \in R: a \cdot 1 = 1 \cdot a = a$).
	\item
	Ist $(R, +, \cdot)$ Ring mit Eins, dann heißen die bez"uglich $\cdot$ invertierbaren Elemente \emph{Einheiten}. Das zu $a$ bez"ugliche $\cdot$
	invertierbare Element bezeichnen wir mit $a^{-1}$.\\ $R^* :=$ Menge der Einheiten in $R$.
\end{enumerate}}
\subsection{Beispiel}
\begin{enumerate}
	\item
	($\Z, +, \cdot$) ist kommutativer Ring mit Eins (1)\\
	$\Z^* = \{1,-1\}$,
	$(\Q, +, \cdot), (\R, +, \cdot)$ ebenso\\
	$\Q^*=\Q \backslash \{0\}, \R^* = \R \backslash \{0\}$.
	\item
	$(2\Z, +, \cdot)$ ist ein kommutativer Ring ohne Eins
	\item
	trivialer Ring $(\{0\},+, \cdot)$ ohne Eins
	\item
	$n \in \N, n \geq 2,  (\Z_n, \oplus, \odot)$ kommutativer Ring mit Eins
	\item
	$(\R^n, \underbrace{+\; ,\; \cdot}_{\text{Komponentenweise}})$; allgemein: $R_1, \dots , R_n$ Ringe, dann $R_1, \times \cdots \times R_n$ Ring.
	\item
	$M_n (\R)$ - Menge aller $n \times n$-Matrizen  "uber $\R$, mit Matrixaddition und -multiplikation ist Ring mit Eins (=$E_n$), nicht kommutativ f"ur $ n \geq 2$.
\end{enumerate}
\subsection[Satz: Rechnen mit Ringen]{Satz (Rechnen mit Ringen)}
Sei $(R, +, \cdot)$ ein Ring, $a,b,c \in R$. Dann gilt:
{\renewcommand{\labelenumi}{(\roman{enumi})}\begin{enumerate}
	\item
	$a \cdot 0 = 0 \cdot a = 0$
	\item 
	$(-a)\cdot b = a \cdot (-b) = -(a \cdot b)$
	\item
	$(-a) \cdot (-b) = a \cdot b$
\end{enumerate}

\subsubsection*{Beweis}

\begin{enumerate}
	\item
	$a \cdot 0 = a \cdot (0+0) \underset{\ref{ring}(3)}{=}a \cdot 0 + a \cdot 0$\\
	addiere $-(a \cdot 0)$ (Inverses von $a \cdot 0$) auf beiden Seiten,  erhalte $0=a \cdot 0$\\
	Analog $0 \cdot a = 0$
	\item
	$(-a)\cdot b + a \cdot b \underset{\ref{ring}(3)}{=} (-a+a) \cdot b = 0 \cdot b \overset{(i)}{=}0$\\
	also ist $(-a \cdot b)$ Inverses zu $a \cdot b$, also $=-(a \cdot b)$.\\
	Analog $a \cdot (-b) = -(a \cdot b)$
	\item
	$(-a) \cdot (-b) \underset{(ii)}{=} -(a \cdot (-b)) \underset{(ii)}{=}-(-(a \cdot b)) = a \cdot b   $
\end{enumerate} } \qed
 %2.5 - 2.11
 %2.5 - 2.11
 % % % % % % % % % % % % % % % % % % % % % % % % % % % % %



% % %  2.5
\subsection[Definition: Körper]{Definition}

Ein kommutativer Ring $(K, +, \cdot)$ heißt \emph{Körper}, wenn jedes Element $0 \neq x \in K$ eine Einheit ist, also wenn
%
\[K^* = K \without{0}\]

% % % 2.6
\subsection{Beispiele}

\begin{enumerate}
	\item
	$(\Q, +, \cdot), (\R, +, \cdot)$ sind Körper. $(\Z, +, \cdot)$ ist kein Körper.
	
	\item
	vgl. Beispiel \ref{gruppenbeispiele} b)
	\[\Z_n^* = \aset{z \in \Z_n \;|\; \ggT(z, n) = 1}\]
	ist Gruppe bezüglich $\odot$
	
	$\Rightarrow (\Z_n, \oplus, \odot)$ ist genau dann ein Körper, wenn $n$ eine Primzahl ist.
\end{enumerate}

% % % 2.7
\subsection[Satz: Rechnen im Körper, Nullteilerfreiheit]{Satz (Rechnen im Körper, Nullteilerfreiheit)}

Sei $(K, +, \cdot)$ ein Körper, $a, b \in K$

Dann gilt 
\[a \cdot b= 0 \gdw a = 0 \text{ oder } b = 0 \]
%
Gegenbeispiel: $(\Z_6, \oplus, \odot)$ ist kein Körper. 
Hier gilt $2 \odot 3 = 0$, aber weder $2=0$, noch $3=0$

\subsubsection*{Beweis}
\begin{itemize}

	\item[''$\Leftarrow$'':]
	klar: $0 \cdot b = 0$ oder $a \cdot 0 = 0$ \;\;(Satz \ref*{rechnenmitringen} (i), Rechenregeln für Ringe)
	
	\item[''$\Rightarrow$'':]
	Sei $a \cdot b = 0$.
	Angenommen $a \neq 0$ (d.h. $a$ hat Inverses)
	
	Dann ist 
	$\begin{array}[t]{l@{}c@{}l}
	b	&=& 1 \cdot b = (a^{-1} \cdot a) \cdot b \\
		&=& a^{-1} \cdot (a \cdot b) \\
		&=& a^{-1} \cdot 0 \\
		&\stackrel{\ref{rechnenmitringen}(i)}{=}& 0
	\end{array}$
	
\end{itemize} \qed

% % %  2.8
\subsection[Definition: Homomorphismus, Isomorphismus]{Definition}

Seien $(R, +, \cdot)$ und $(\tilde{R}, \boxplus, \boxdot)$ Ringe.

{\renewcommand{\labelenumi}{(\roman{enumi})}
\begin{enumerate}
	\item
	$\varphi: R \to \tilde{R}$ heißt (Ring-)\emph{Homomorphismus}, falls gilt:
	\[\varphi(\underbrace{x+y}_{\in R}) = \underbrace{\varphi(x)}_{\in \tilde{R}} \boxplus \underbrace{\varphi(y)}_{\in \tilde{R}}
	\text{\;\; und \;\;}
	\varphi(x \cdot y) = \varphi(x) \boxdot \varphi(y) \quad \forall x, y \in R\]
	
\end{enumerate}}

% % % 2.9
\subsection{Beispiel}

$\begin{array}[b]{r@{\;}c@{\;}l}
	\varphi(\Z, +, \cdot) &\to&  (\Z_n, \oplus, \odot) \\
	x	&\mapsto& x \mod n 
\end{array}$
ist Ringhomomorphismus (kein Isomorphismus), da $\varphi$ nicht injektiv ist, z.B. $n=5: \varphi(1)= \varphi(6) = \varphi(11) \dots$

% % % 2.10
\subsection[Satz: Chinesischer Restsatz]{Satz (Chinesischer Restsatz)} \label{chin.restsatz}

Seien $m_1, \dots, m_n \in \N$ paarweise teilerfremd,
$M := m_1 \cdot \ldots \cdot m_n, \;\; a_1, \dots, a_n \in \Z$

Dann existiert ein $x$, $0 \leq x < M$ mit 

$\begin{array}{lcll}
x &\equiv& a_1	& (\mod m_1) \\
x &\equiv& a_2	& (\mod m_2) \\
\dots \\
x &\equiv& a_n	& (\mod m_n)
\end{array}$

\subsubsection*{Beweis}
Für jedes $i \in \aset{1, \dots, n}$ sind die Zahlen $m_i$ und $M_i := \frac{M}{m_i}$ teilerfremd.

$\Rightarrow$ EEA liefert $s_i$ und $t_i \in \Z$ mit $t_i \cdot m_i + s_i \cdot M_i = 1$

Setze $e_i := s_i \cdot M_i$, dann gilt:
\[\begin{array}{ll}
	e_i \equiv 1	& (\mod m_i) \\
	e_i \equiv 0 	& (\mod m_j) \;\; (j \neq i)
\end{array}\]
%
Die Zahl $x := \sum_{i=1}^{n}a_ie_i (\mod M)$ ist dann die Lösung der simultanen Kongruenz. \qed

% % % 2.11
\subsection{Beispiel}

\begin{enumerate}

	\item
	Finde $0 \leq x < 60$ mit 
	$x \equiv \begin{cases}
	2 & (\mod 3) \\
	3 & (\mod 4) \\
	2 & (\mod 5)
	\end{cases}$
	
	$M = 3 \cdot 4 \cdot 5 = 60$
	
	\[\begin{array}{lr@{\quad\Rightarrow\;}l}
	M_1 = \frac{60}{3} = 20 & 7 \cdot 3 + (-1) \cdot 20 = 1	& e_1 = -20 \\
	M_2 = \frac{60}{4} = 15 & 4 \cdot 4 + (-1) \cdot 15 = 1	& e_2 = -15 \\
	M_3 = \frac{60}{5} = 12	& 5 \cdot 5 + (-2) \cdot 12 = 1	& e_3 = -24
	\end{array}\]
	
	$x = (2 \cdot (-20) + 3 \cdot (-15) + 2 \cdot (-24)) \mod 60 = 47$
	
	\item
	Was ist $2^{1000} \mod \underbrace{1155}_{3 \cdot 5 \cdot 7 \cdot 11}$
	
	\begin{enumerate}
		\item
		Berechne $2^{1000} \mod 3, 5, 7, 11$
		
		$\begin{array}{lcrcl}
		2^{1000} \mod 3 &=& (-1)^{1000} \mod 3 &=& 1 \\
		2^{1000} \mod 5 &=& 4^{500} \mod 5 = (-1)^{500} \mod 5 &=& 1 \\
		2^{1000} \mod 7 &=& 2^{3 \cdot 333+1} \mod 7 = (8^{333} \cdot 2) \mod 7 = (1 \cdot 2) \mod 7 &=& 2 \\
		2^{1000} \mod 11 &=& 2^{5 \cdot 200} \mod 11 = 32^{200} \mod 11 = (-1)^{200} \mod 11 &=& 1
		\end{array}$
		
		\item
		Suche $0 \leq x < 1155$ mit
		$x \equiv \begin{cases}
			1 & (\mod 3) \\
			1 & (\mod 5) \\
			2 & (\mod 7) \\
			1 & (\mod 11)
			\end{cases}$ 
			
		Der chinesische Restsatz liefert $x = 331$
	\end{enumerate}
	
\end{enumerate}

























 %2.12 - 2.19
 %2.12 - 2.19
 % % % % % % % % % % % % % % % % % % % % % % % % % % % % %

\subsection{Bemerkung} \label{chinsatzeindeutig}
Man kann auch zeigen, dass die L"osung $x$ aus Satz \ref{chin.restsatz} eindeutig ist:

Durch 
$\begin{array}[t]{lr@{\;}c@{\;}l}
\psi:&	\Z_M &\to&  Z_{m_1} \times \Z_{m_2} \times \dots \times \Z_{m_n} \\
&	x	&\mapsto& (x \mod{m_1}, \dots, x \mod{m_n})
\end{array}$

wird ein Ringisomophismus definiert:

$\psi$ ist surjektiv (zu jedem n-Tupel aus $\Z_{m_1}\times \dots \times \Z_{m_n}$ gibt es eine L"osung $x$, siehe Restsatz) und es gilt:

$\underbrace{|\Z_M|}_{M}=\underbrace{|\Z_{m_1}\times \dots \times \Z_{m_n}|}_{m_1\cdot \dots \cdot m_n = M}$

also ist $\psi$ bijektiv, also auch injektiv, also ist L"osung $x$ eindeutig.

\subsection[Korollar: Phi-Funktion berechnen]{Korollar}
$M=m_1\cdot \dots \cdot m_n$, $m_i$ paarweise teilerfremd.\\
Dann ist $\varphi(M)=\varphi(m_1)\cdot \dots \varphi(m_n),$ insbesondere:

$n=p^{a_1}_1 \cdot \dots \cdot p^{a_k}_k$ ($p_i$ Primzahlen, $a_1>0, p_i \neq p_j$ f"ur $i\neq j$)

\subsubsection*{Beweis}

Nach \ref{chinsatzeindeutig} ist $\Z_M \cong \Z_{m_1} \times \dots \times \Z_{m_n}$ mittels $\psi$\\
$\Rightarrow x$ Einheit $\Leftrightarrow \psi(x) = ( x\mod{m_1}, \dots, x \mod{m_n})$ Einheit
\\ $\Leftrightarrow x \mod{m_i}$ Einheit $\forall i = 1 \dots n$\\
$\Rightarrow \varphi(M) = \varphi(m_1) \cdot \dots \cdot \varphi(m_n)$\\
$\varphi(p^a)\underbrace{=}_{\text{"Uberlegen}}p^a - p^{a-1} = p^{a-1}(p-1)$

\subsection[Definition: Polynom]{Definition}
Sei $K$ K"orper mit Nullelement $0$ und Einselement 1:
{\renewcommand{\labelenumi}{(\roman{enumi})}\begin{enumerate}
	\item
	Ein \emph{Polynom "uber K} ist Ausdruck $f=a_0x^0+a_1x^1+\dots + a_nx^n$, $n\in \N_0, a_i\in K$.\\
	$a_i$ heißen \emph{Koeffizienten} des Polynoms.
	\begin{enumerate}
		\item
		Ist $a_i=0$, so kann man $0 \cdot x^i$ bei der Beschreibung weglassen.
		\item
		Statt $a_0x^0$ schreibt auch $a_0$
		\item
		Sind alle $a_i=0$, so schreibt man $f=0$, das Nullpolynom.
		\item
		Ist $a_i =1$, so schreibt man $x^i$ statt $1 \cdot x^i$
		\item
		Die Reihenfolge der $a_ix^i$ kann ver"andert werden, ohne dass das Polynom sich ver"andert ($x^4+2x^3 + 3= 2x^3 +  3 + x^4$)
	\end{enumerate} 
	\item
	Zwei Polynome $f$ und $g$ sind \emph{gleich}, wenn ($f=0$ und $g=0$) oder (${f=a_0 + a_1x^1+ \dots + a_nx^n},\\ {g=b_0+b_1x^1+ \dots + b_mx^m}, {a_n \neq  0, b_m \neq 0}$ und $n=m$, ${a_i = b_i}$ f"ur $i=0, \dots , n$) gilt.
	\item
	Die	Menge aller Polynome "uber $K$ bezeichnet man als $K[x]$
\end{enumerate} }

\subsection{Beispiel}
\begin{enumerate}
	\item
	$\underbrace{f}_{f(x)} = 3x^2 + \frac{1}{2}x -1 \in \Q [x] \land f \in \R [x]$
	\item
	$g= x^3 + x^2 +1 \in \Z_2[x] $ 
\end{enumerate}
Wir wollen in $K[x]$ wie in einem Ring rechnen k"onnen. Wir brauchen dazu $+$ und $\cdot$ f"ur Polynome.

\subsection[Satz und Definition: Polynomring]{Satz und Definition}
$K$ K"orper, dann wird $K[x]$ zu einem kommutativen Ring mit Eins durch folgende Verkn"upfungen:\\
$f= \underbrace{\sum_{i=0}^{n} a_ix^i}_{\text{z.B. }x+2}, \;\;\;g=\underbrace{\sum_{j=0}^{m} b_j x^j}_{x^3 + 2x + 1}$,

dann \[f + g = \underbrace{\sum_{i=0}^{\max(m,n)}(a_i+b_i)x^i}_{x^3 + 3x +3}\]

\[f \cdot g = {\sum_{i=0}^{n+m} c_ix^i}\] 
\[\text{mit } c_i= a_0b_i+a_1b_{i-1}+ \dots + a_ib_0 = \sum_{j=0}^{i} a_jb_{i-j}  \tag{Faltungsprodukt}\]
(setze $a_i$ mit $i>n$ bzw. $b_j$ mit $j>m$ gleich 0)
\begin{itemize}
	\item
	Einselement: $f=1 \;(a_0=1, a_j = 0 \;\;$f"ur $ j\geq 1)$
	\item
	Nullelement: $f= 0$
\end{itemize}
$K[x]$ heißt der \emph{Polynomring} in einer Variablen "uber $K$.\\
Beweis: Ringeigenschaften nachrechnen.

\subsection{Bemerkung}
Die +-Zeichen in der Beschreibung der Polynome entsprechen der Ring-Addition der \emph{Monome} $a_0, ax, a_2x^2, \dots, a_nx^n$

\subsection{Beispiel}

\begin{enumerate}
	\item 
	in $\Q[x], \R[x]$ Addition, Multiplikation klar
	
	\item
	in $\Z_3[x]$:
	$f = 2x^3 + 2x + 1$,
	$g = 2x^3 + x$
	
	$\begin{array}{lcl}
	f + g &=& x^3 + 1 \\
	f \cdot g &=& (2x^3 + 2x + 1)(2x^3 + x) \\
	&=& x^6 + 2x^4 + x^4 + 2x^2 + 2x^3 + x \\
	&=& x^6 + 2x^3 + 2x^2 + x
	\end{array}$
	
	\item
	in $\Z_2[x]$:
	$f = x^2 + 1$,
	$g = x + 1$
	
	$\begin{array}{lcl}
	f+g &=& x^2 + x \\
	f+f &=& 0 \\
	g \cdot g &=& x^2 + 1
	\end{array}$
	
	
\end{enumerate}


\subsection[Definition: Grad eines Polynoms]{Definition}

Sei $0 \neq f \in K[x]$

$f=a_0 + a_1x + \dots + a_nx^n$ mit $a_n \neq 0$

Dann heißt $n$ der \emph{Grad} von $f$ Grad($f$)

Grad($0$) $:= - \infty$ \\
Grad($f$) $=0$ \quad \emph{für konstante Polynome $\neq 0$}

















 %2.20 - 2.29
% % % 2.20
\subsection{Satz}
\label{sub:satz}

$K$ Körper, $f,g \in K[x]$

Dann ist $\Grad(f\cdot g) = \Grad(f) + \Grad(g)$

(Konvention: $-\infty + (-\infty) = -\infty + n = -\infty$)

\subsubsection*{Beweis}

	
	Stimmt für $f = 0$ oder $g = 0$
	
	\begin{align*}
		f &= a_0 + a_1x^1 + \ldots +  a_nx^n \quad &\text{mit} \quad  a_n \neq 0\\
		g &= b_0 + b_1x^1 + \ldots +  b_mx^m \quad &\text{mit} \quad  b_m \neq 0\\
		f \cdot g &= (\ldots) \cdot (\ldots) = \ldots + \underbrace{(a_nb_n)}_ {\mathclap{\substack{\neq 0, \\ \text{(siehe Satz 2.7
						Nullteilerfreiheit in Körpern)}}}}\cdot x^{n+m}
	\end{align*}
	
	
	Höhere Potenzen mit Koeffizienten $\neq 0$ gibt es nicht
	
	$\Rightarrow \Grad(f \cdot g) = n + m$

% % % 2.21
\subsection{Korollar}

$K$ Körper, dann
$K[x]^* = \{f \in K[x]\; |\; \Grad(f) = 0\}$,

d.h. nur die konstanten Polynome $\neq 0$ sind in $K[x]$ bez"uglich $\cdot$ invertierbar.

\[\underbrace{f}_{\mathclap{\Grad \;n}} \cdot \underbrace{f^{-1}}_{\text{müsste Grad $-n$ haben}} = \underbrace{1}_{\mathclap{\Grad \;0}} \leftarrow \text{geht nicht}\]

% % % 2.22
\subsection{Definition}


Sei $b \in K$

 $\varphi_b: K[x] \rightarrow K$, 
$f := \sum\limits_{i = 0}^n a_ix^i \mapsto f(b) := \sum\limits_{i = 0}^n a_ib^i$

ist ein surjektiver Ringhomomorphismus, der sogenannte \emph{Auswertungshomomorphismus} an der Stelle $b$.

(setze $b$ für $x$ ein)


% % % 2.23
\subsection{Definition}

$K$ Körper, $f, g \in K[x]$

$f$ \underline{teilt} $g$, $f|g$, falls ein $q \in K[x]$ existiert mit $g = q \cdot f$

(Nach \ref{sub:satz} ist dann $\Grad(f) \leq \Grad(g)$, falls $g \neq 0$)

% % % 2.24
\subsection{Definition (Division mit Rest)}

$K$ Körper, $0 \neq f \in K[x],\; g \in K[x]$

Dann existieren eindeutig bestimmte Polynome $q, r \in K[x]$ mit
$g = q \cdot f + r \; \text{und} \;\\ \Grad(r) < \Grad(f)$.

Bezeichnung:
\begin{align*}
	 r &=: g \mod f\\
	 q &=: g\; \text{div}\;  f
\end{align*}

\subsubsection*{Beweis}
Vgl. Mathe I für $\mathbb{Z}$, siehe z.B. WHK Satz 4.69

% % % 2.25
\subsection{Beispiel}
\label{sub:2.25}

\begin{enumerate}[a)]
	\item
	\begin{align*}
	g &= x^4+2x^3-x+2 \quad \in \mathbb{Q}[x]\\
	f &= 3x^2-1 \quad \in \mathbb{Q}[x]
	\end{align*}
	
	Rechne:
		
	\polylongdiv[style=C, div=:]{x^4+2x^3+0x^2-x+2}{3x^2-1}
	\item
	\[g = x^4+x^2+1 \quad \quad f=x^2+x \quad \in \mathbb{Z}_2[x]\]
	Rechne:
	\[(x^4+x^2+1) : x^2+x= \underbrace{x^2+x}_{q}\]
	
	
\end{enumerate}


% % % 2.26
\subsection{Korollar}

$K$ Körper, $a \in K$

$f \in K[x]$ ist genau dann durch $(x - a)$ teilbar, wenn $f(a) = 0$ ist (d.h. $a$ ist \underline{Nullstelle} von $f$).

\subsubsection*{Beweis}

"$\Rightarrow$" $\quad$ sei $f$ durch $(x - a)$ teilbar, d.h.
\[f = q \cdot (x - a) \Rightarrow f(a) = q(a) \cdot (\underbrace{a-a}_{0}) = 0 \quad q \in K\]

"$\Leftarrow$" $\quad$ Division mit Rest: $f = q(x - a) + r$, wobei
$\Grad(r) < \underbrace{\Grad(x-a)}_{1}$

$\Rightarrow r$ ist konstantes Polynom ($\Grad \; 0$) oder Nullpolynom $(\Grad \;(-\infty)$) also $r \in K$

\[0 = f(a) = q(a) \cdot 0 + r \Rightarrow r = 0 \quad \]
\qed

% % % 2.27
\subsection{Definition}

$K$ Körper

\begin{enumerate}[(i)]
	\item
	Ein Polynom dessen höchster von $0$ verschiedener Koeffizient gleich $1$ ist, heißt normiert.
	
	\item
	$g, h \in K[x]$, nicht beide $0$ 
	
	$f \in K[x]$ heißt \emph{größter gemeinsamer Teiler} von $g$ und $h$ ($f=\ggT(g,h)$), falls
	$f$ normiertes Polynom von maximalem Grad ist, das $g$ und $h$ teilt.
	
	\item
	$g, h \in K[x] \;\backslash\; \{0\}$ beide nicht $0$
	
	$f \in K[x]$ heißt \emph{kleinstes gemeinsames Vielfaches} von $g$ und $h$\\ ($f = kgV(g,h)$), falls
	$f$ normiertes Polynom von kleinstem Grad ist, das von $g$ und $h$ geteilt wird.
\end{enumerate}


% % % 2.28
\subsection{Bemerkung}

\begin{enumerate}[a)]
	\item
	$f = \sum\limits_{i = 0}^n a_ix^i, a_n \neq 0$, dann ist $a_n^{-1}f = x^n + \ldots\;$ normiertes Polynom.
	
	(z.B.: $f = 3x^2+x+7 \in \mathbb{R}[x]$)
	
	dann $\frac{1}{3}f = x^2+\frac{x}{3} + \frac{7}{3}$ normiert.
	
	In $\mathbb{Z}_{11}[x]: \underbrace{4}_{\mathclap{\text{Inverses von 3, denn $3\cdot 4 = 12 \equiv 1 \pmod{11}$}}}f = x^2+4x_6$ normiert.
	
	\item
	$\kgV(g,h)$ existiert und ist eindeutig:
	
	\begin{align*}
	\text{sei}\; f_1 &= \kgV(g,h), \; f_2 = \kgV(g,h)\\
	&\Rightarrow g,h | f_1, \quad g,h | f_2\\
	&\Rightarrow g,h | (f_1-f_2)
	\end{align*}
	
	\item
	$\ggT(g,h)$ existiert. Beweis Eindeutigkeit wie in $\mathbb{Z}$ (Mathe I), folgt aus.
\end{enumerate}


% % % 2.29
\subsection{Satz (von Bezout)}
\label{sub:satz_von_bezout_}

$K$ Körper, $g, h \in K[x]$, nicht beide $0$.

Dann existieren $s, t \in K[x]$, sodass
\[f = s \cdot g + t\cdot h\]

ein $\ggT$ von $g$ und $h$ ist.

(Beweis: EEA in $K[x]$, später)

 %2.30 - 2.35
 %2.30 - 2.25
 % % % % % % % % % % % % % % % % % % % % % % % % % % % % %

% % % 2.30
\subsection{Satz} Euklidischer Algorithmus in $K[x] \rightarrow$ siehe ,,Blatt''

% % %  2.31
\subsection{Satz} EEA in $K[x] \rightarrow$ siehe ,,Blatt''

% % % 2.32
\subsection{Beispiel}
$g=x^4+x^3 + 2x^2+1,  h= x^3+2x^2+2 \in \Z_3 [x]$\\
\dots TBD \dots

% % % 2.33
\subsection{Definition}
$k$ K"orper. Ein Polynom $p\in K[x]$, Grad$(p)\geq 1$ (d.h. $p\neq 0$, $p$ nicht konst., also keine Einheit) heißt \emph{irreduzibel}, falls golt: \\
Ist $p=f \cdot g$ ($f,g\in K[x]$), so ist Grad($f)= 0$ oder Grad($g)=0$ (d.h. $f$ oder $g$ ist konst. Polynom).

Bemerkung: $p= a \cdot a^{-1} \cdot p$ f"ur $a \in K \backslash \{0\}$ geht immer.

% % % 2.34
\subsection{Beispiel}
\begin{enumerate}
	\item
	$ax+b$ ($a \neq 0$) ist irreduzibel in $K[x]$ f"ur jeden K"orper $K$
	\item
	$x^2-2 \in \Q[x]$ ist irreduzibel:\\
	angenommen nicht, dann $(x^2-2) = (ax+b)(cx+d)$ mit $a,b,c \in \Q \land a,c \neq 0$\\
	$(ax+b)$ hat Nullstelle $-\frac{b}{a}$, also m"usste auch $(x^2-2)$ Nullstelle $\underbrace{-\frac{b}{a}}_{\in \Q}$ haben.
	Nullstellen von $(x^2-2)$ sind aber nur $\sqrt{2}$ und $-\sqrt{2}$, beide nicht in $\Q$ !
	\item
	$x^2-2 \in \R[x]$ ist nicht irreduzibel.\\
	$x^2-2 = \underbrace{(x+ \sqrt{2})}_{\in \R[x]} \cdot \underbrace{(x-\sqrt{2})}_{\in \R[x]}$
	\item
	$x^2+1 \in \R[x]$ ist irreduzibel
	\item
	$x^2+1 \in \Z_5[x]$ ist nicht irreduzibel:\\
	$(x^2 +1) = (x+2) \cdot (x+3) = (x^2 + 3x +2x +1) = (x^2+1)$\\
	$2 \Rightarrow (x^2+1)$ ist teilbar durch $(x-2) \hat{=} (x+3)$
\end{enumerate}

% % % 2.35
\subsection{Abschlussbemerkung}
\begin{enumerate}
	\item
	Irreduzibel Polynome in $K[x]$ entsprechen den Primzahlen in $\Z$. Man kann zeigen:
	$f = \sum_{i=0}^{n} a_i x^i \in K[x]$, $a_n \neq 0, n \geq 1$.\\
	Dann existieren eindeutig bestimmte irreduzibel Polynome $p_1, \dots, p_e$ und nat"urlichen Zahlen $m_1, \dots , m_e \in \N$ mit $f= a_n \cdot p_1^{m_1} \cdot \dots \cdot p_e^{m_e}$
	\item
	geg: Primzahl $p$, dann gibt es K"orper mit $p$ Elementen:\\
	$(\Z_p, \oplus, \odot)$
	
	Man kann zeigen: zu jeder Primzahlpotenz $p^a$ gibt es K"orper mit $p^a$ Elementen, diesen konstruiert man "uber irreduzible Polynome in $\Z_p[x]$. 
\end{enumerate}
 %3.1  - 3.5
% % % 3
\section{Der K"orper der $\C$ der Komplexen Zahlen} 

% % % 3.1
\subsection{Definition}
Eine komplexe Zahl $\Z$ ist von der Form $z=x+i\cdot y$ mit $x, y \in \R$ und einer ,,Zahl'' $i$ mit $i^2=-1$ (,,imagin"are Einheit''). $x$ heißt Realteil von $z$, $x=$Re $z$\\
$y$ heißt Imagin"arteil, $y=$In $z$.

Die Menge aller komplexen Zahlen bezeichnen wie mit $\C$ und definieren auf $\C$ Addtition und Multiplikatio wie folgt:

F"ur $z=x+iy$ und $w=a+ib$ ist \\$z+w:= (x+a) + i (y+b)$, \\$z-w:= (x-a) + i (y-b)$ und \\$z\cdot w:= (xa-yb) + i(xb+ya).$

Erl"auterung zur Multiplikation: $((x+iy)(a+ib) = xa + xib + iya + i^2yb = (xa-yb) + i(xb+ya)$.

Mit diesen Verkn"upfungen ist $\C$ ein K"orper:
\begin{enumerate}
	\item
	AG, kG, DG: nachrechnen
	\item
	$0=0+i\cdot 0$
	\item
	additiv Inverses: $-z = -x-iy$
	\item
	$1=1+i\cdot 0$
	\item
	multiplikativ Inverses: $z^{-1}= \frac{1}{z}= \frac{1}{x+iy}=\frac{1}{x+iy}\cdot\frac{x-iy}{x-iy}= \frac{x-iy}{x^2+y^2}=\underbrace{\frac{x}{x^2+y^2}}_{\in \R}+ i \cdot \underbrace{\frac{-y}{x^2+y^2}}_{\in \R}$
\end{enumerate}
	Man nennt f"ur $z=x+iy$ die Zahl $ \overline{z} = x-iy$ die zu $z$ \emph{konjugiert komplexe Zahl} und $|z| := \sqrt{x^2+y^2}$ den \emph{Betrag} von $z$.

% % % 3.2
\subsection{Beispiel}
\begin{enumerate}
	\item
	$z= 2+3i$ mit Re$(z)=2$ und Im$(z)=3$.\\
	$\overline{z} = 2-3i, |z|=\sqrt{2^2+3^2}= \sqrt{13}$\\
	$z \cdot \overline{z}=(2+3i)\cdot(2-3i)\\
	=4-6i+6i-9i^2
	=4+9=13$
	\item
	$w= 1+i = 1+1 \cdot i:\;\;$ Re$(w)=1$, Im$(w)=1$, $\overline{w}= 1-i,\; |w|=\sqrt{1^2+1^2}=\sqrt{2}$
	\item
	Selbst nachrechnen: $u=7=7+0 \cdot i,\; v= 5i=0+5i$
	\item
	$u+w+z = 7+ (1+i) + (2+3i) = 10 + 4i$\\
	$u \cdot w = 7 \cdot (1+i) = 7+7i$\\ 
	$ \frac{w}{z}= \frac{1+i}{2+3i}= \frac{(1+i)\cdot(2-3i)}{4+9} = \frac{2-3i+2i=3i^2}{13}= \frac{5-i}{13}=\frac{5}{13}-\frac{1}{13}i$
\end{enumerate}

% % % 3.3
\subsection{Bemerkung: komplexe Zahlenebene}
%TODO Tikz-Action
Man kann $\C$ veranschaulichen in der ,,Gaußschen Zahlenebene'':\\
Betrachte $z=x+iy$ als Punkt $(x|y)$ in $\R^2$:
%TODO Some more tikzzzz 

% % % 3.4
\subsection{Satz (Eigenschaften)}


\begin{enumerate}
	\item %a
	$\left.
	\begin{array}{cl}
	\overline{w+z} &= \overline{w} + \overline{z}\\
	\overline{w\cdot z} &= \overline{w} \cdot \overline{z}\\
	\overline{\frac{w}{z}} &= \frac{\overline{w}}{\overline{z}} \quad (z \neq 0)\\
	\overline{\overline{z}} &= z
	\end{array}
	\right\rbrace \mathbb{C} \rightarrow \mathbb{C}, z \mapsto \overline{z}\; \text{ist Körperisomorphismus}$
	
	\item %b
	$Re(z) = \frac{z+\overline{z}}{2}, Im(z)= \frac{z - \overline{z}}{2i}$
	\item %c
	$|z| \geq 0,\; |z| = 0$ nur für $z = 0$
	\item
	$|z| = |\overline{z}| = \sqrt{z \cdot \overline{z}}$
	\item
	$|w \cdot z| = |w| \cdot |z|$
	\item
	$|w + z| \leq |w|+|z|$ (Dreiecksungleichung)
	
	$|w + z| \geq \bigg | |w|-|z| \bigg |$
\end{enumerate}

\subsubsection*{Beweis}

z.B.: d) sei $z = x + iy \quad x,y \in \mathbb{R}$

$\Rightarrow \overline{z} = x - iy, \quad |z| = \sqrt{x^2+y^2}$

$|\overline{z}|=\dots$

% % % 3.5
\subsection{Bemerkung}
\begin{enumerate}
	\item
	In $\C$ existiert $\sqrt{-1}: \pm i$, d.h. $x^2+1=0$ ist l"osbar in $\C$, das Polynom $x^2+1$ ist nicht irreduzibel in $\C[x]$: $ x^2+1 = (x+i)(x-i)$
	\item
	Mann kann jede quadratische Gleichung $ax^2+bx+c$ ($a,b,c\in \R$) in $\C$ l"osen:\\
	$x_{1|2} = \frac{-b \pm \sqrt{b^2-4ac}}{2a}$\\
	Jedes $b^2-4ac < 0$ ist, schreibe:\\
	$\frac{-b \pm \sqrt{4ac-b^2}\cdot i}{2a}$
	\item
	Es gilt sogar: Fundamentalsatz der Algebra:\\
	Jedes Polynom $f\in \C[x]$ vom Grad $n\geq 1$ hat genau $n$ Nullstellen in $\C$. 
\end{enumerate}
 %3.6 - 3.11
 % 3.6 - 3.11
 % % % % % % % % % % % % % % % % % % % % % % % % % % % % %

% % % 3.6
\subsection{Polarkoordinaten}

Eine andere M"oglichkeit, komplexe Zahlen zu beschreiben:

Angabe von Winkel ($\varphi$) und Abstand $r$ zum Nullpunkt.\\
Zu jedem $z \in \C$ gibt es ein eindeutig bestimmtes $r \leq 0$ und ein $\varphi \in \R$ mit \\$z=r(\cos\varphi+i \cdot \sin \varphi)$ (Polarkoordinatendarstellung von $z$) und zwar ist $r=|z|=\sqrt{x^2 + y^2}$ f"ur $z = x + iy, \;\frac{x}{r}= \cos\varphi, \; \frac{y}{r}=\sin \varphi$:
\begin{align*}
	z &= x +iy\\
	&= r \cdot \cos\varphi + i \cdot r \cdot \sin\varphi\\
	&= r \cdot (\cos\varphi + i \cdot\sin\varphi)
\end{align*}

Aus den Additionstheoremen f"ur $\sin, \, \cos$ folgt (P"U6):

\begin{align*}
z_1 \cdot z_2 &= |z_1| \cdot |z_2|\cdot (\cos(\varphi_1 + \varphi_2)+i \cdot \sin(\varphi_1 + \varphi_2))\\
z^2 &= |z|^2 \cdot (\cos(2\varphi)+ i \cdot \sin(2\varphi))\\
\pm \sqrt{z} &= \sqrt{|z|} \cdot (\cos (\frac{\varphi}{2}) + i \cdot \cos (\frac{\varphi}{2}))
\end{align*}

% % % 3.7
\subsection{Beispiel}
\begin{enumerate}
	\item
	$z_1=1,\; r_1=1, \; \varphi_1= 0 \Rightarrow z_1 = 1 \cdot (\cos 0+ i \cdot \sin 0 )$
	\item
	$z_2=i, \; r_2=1, \; \varphi_2=\frac{\pi}{2}\Rightarrow z_2 = 1 \cdot (\cos \frac{\pi}{2}+ i \cdot \sin \frac{\pi}{2} )$
	\item
	$z_3=1+i, \; r_2=\sqrt{2}, \; \varphi_2=\frac{\pi}{4}\Rightarrow z_3 = \sqrt{2} \cdot (\cos \frac{\pi}{4}+ i \cdot \sin \frac{\pi}{4} )$
\end{enumerate}

% % % 3.8
\subsection{Definition/Schreibweise}

$e^{i\varphi}:= \cos \varphi + i \cdot \sin \varphi$

$z= \underbrace{r}_{\mathclap{\text{Betrag}}} \cdot e^{i\varphi}$

% % % 3.9
\subsection{Bemerkung}

Statt Defintion 3.8:

Man kann auch die Definition von Folgen, Konvergenz, Grenzwert von $\R$ auf $\C$ "ubertragen, alles aus Mathe II (Analysis!), u.a. auch Potenzreihen, insbesondere die Exponentialfunktion definieren.

F"ur alle $z \in \C$ konvergiert $\sum_{k=0}^{\infty}\frac{z^k}{k!} := \exp(z), \; e^z $

Mit den Methoden aus Mathe II - ,,2. Teil'' kann man dann zeigen, dass \[e ^it = \cos t + i \cdot \sin t \; \forall t \in \R \; \text{(Eulersche Formel)}\]
\[z_1\cdot z_2 = (r_1\cdot r_2)\cdot (\cos(\varphi_1 + \varphi_2)+i \cdot \sin (\varphi_1 + \varphi_2)) = \underline{(r_1\cdot r_2) \cdot e^{i (\varphi_1 + \varphi_2)} }\]

% % % 3.10
\subsection{Beispiele}

\begin{enumerate}
	\item
	$1 \cdot e^{i\cdot 0 } = 1$
	\item
	$e^{i\pi} = -1 $ (und: $e^{i\pi}+1 = 0 $ \smiley)
	\item
	$2 \cdot e^{2\pi} = 2$ 
	\item
	$\ldots$ %TODO Koennte man hier noch verfolgestaendigen
	
\end{enumerate}

% % % 3.11
\subsection{Bemerkung}

$\C$ hat alle algrebraischen und analystischen Eigenschaften wie $\R$ (oder besser), außer:

Es gibt auf $\C$ keine vollst"andige Ordnung $\leq$, die mit + und $\cdot$ vertr"aglich ist, d.h. f"ur die gelten w"urde:
\begin{align*}
a \leq b, c \leq d&\Rightarrow a+c \leq b+d\\
a \leq b, r \geq 0&\Rightarrow ra \leq rb
\end{align*}
 % 4.1 - 5.8
%section 4
\section{Wiederholung und Erweiterung der linearen Algebra aus Mathe II}
%4.1
\subsection{Beispiel}

\begin{enumerate}
	\item
	$ K = \Z , V_1= \Z_2^2=\aset{
	\vct{x_1 \\ x_2} : x_1, x_2 \in \Z_2}$

	$V_1$ hat 4 Elemente: $
	\vct{0 \\0},\vct{0 \\ 1},\vct{1 \\ 0},\vct{1 \\ 1}$
	
	$\zerovec= \vct{0 \\ 0}, \vct{0 \\ 1} + \vct{0 \\ 1}= \vct{0 \\ 0},$ d.h. $-\vct{0 \\ 1} = \vct{0 \\ 1}, \vct{0 \\ 1} + \vct{1 \\ 1}=\vct{0 \\ 1}$
	 
	$\forall v \in V: 0 \cdot v = \zerovec = \vct{0 \\ 0}$ und $1\cdot v = v$
	\item
	$K = \Z_5, V_2=\Z_5^3=\aset{\vct{x_1 \\ x_2 \\ x_3}}$
	
	$v=\vct{0 \\ 1\\ 2 }, w = \vct{3 \\ 2 \\ 4} \in \Z_5^3$
	
	$-v= \vct{0 \\ 4 \\ 3}, -w = \vct{2 \\ 3 \\ 1}, v + w  \vct{3 \\ 2 \\ 1} $
	
	$1 \cdot w = w, 2 \cdot w = \vct{1 \\ 4 \\ 3}, 3 \cdot w = \cdots$
	
	$|V| = 5 \cdot 5 \cdot 5 = 125$
	\item 
	$U = \left\lbrace 
	\vct{x_1 \\ x_2} \in V_1: x_1 \oplus x_2 = 0 \right\rbrace$ ist UR von $V_1$
	\begin{itemize}
		\item
		$U = \left\lbrace 
		\vct{0 \\ 0}, \vct{1 \\ 1}  \right\rbrace \neq \emptyset$
		\item
		Sei $u= \vct{u_1 \\ u_2} \in U $, d.h. $u_1 \oplus u_2 = 0$ 
	\end{itemize}
	 $\Rightarrow$ f"ur $\lambda \cdot u= \vct{\lambda u_1 \\ \lambda u_2}$ gilt $\lambda u_1 \oplus \lambda u_2 = \lambda \cdot \underbrace{(u_1 \oplus u_2)}_{0}=0$ 
	 \item
	 $\Z_3^3:$
	 
	 $\vct{0 \\ 0 \\ 0}$ l.a.; $\vct{0 \\ 1 \\2}$ l.u.; $\vct{0 \\ 1 \\ 2}, \vct{0 \\ 2 \\ 1}$ sind l.a.
	 \item
	 Kanonische Basis von $V_2$ (Bsp. b)):
	 
	 $B_1 = \underbrace{\aset{e_1 = \vct{1 \\ 0 \\ 0}, e_2 = \vct{0 \\ 1 \\ 0}, e_3= \vct{0 \\ 0 \\ 1}}}_{\text{geordnete Basis}},\; \dim V_2 = 3$
	 
	 z.B.: $\vct{2 \\ 3 \\ 1}= \alpha \cdot e_1 + \beta \cdot e_2 + \gamma \cdot e_3$ mit $\alpha = 2, \;\beta = 3, \;\gamma =1 $ und $\alpha, \; \beta, \; \gamma $ sind die kartesischen Koordinaten.
	 
	 Eine andere (geordnete) Basis, z.B.:
	 
	 $B_2 = \aset {\vct{2 \\ 0 \\ 0}, \vct{0 \\ 1 \\ 1}, \vct{1 \\2 \\ 3}}$
	 
	 Zeige Vektoren sind linear unabh"angig:
	 
	 $\alpha \cdot \vct{2 \\ 0 \\ 0} + \beta \cdot \vct{0 \\  1\\ 1} + \gamma \cdot \vct{1 \\ 2 \\ 3} = \zerovec$
	 
	 $\Rightarrow  \cdots  \Rightarrow \cdots \Rightarrow \alpha = \beta = \gamma = 0$
	 
	 Koordinaten von $\vct{2 \\ 3 \\ 1}$ in $B_2$?
	 
	 Stelle LGS auf und l"ose es \dots
	 \end{enumerate}
	 % 4.2
	 \subsection{Definition}
	 
	 $A \in M_{n,n}(K)$ heißt \emph{invertierbar}, falls $\exists  A^{-1} \in M_{n,n}(K)$ mit $A^{-1} \cdot A = A \cdot A^{-1} = E_n$
	 
% % % % % % % % % %
% %  2014-11-25 % %
% % % % % % % % % %
%TODO Create another file for this code
%5
\section{Lineare Abbildungen}
%5.1
\subsection{Definition}

Seien $V,W$ $K$-Vektorr"aume.
\begin{enumerate}
	\item
	$\varphi: V \rightarrow$ heißt \emph{lineare} Abbildung ($VR$-Homomorphismus), falls:
	\begin{itemize}
		\item
		$\forall v_1, v_2 \in V: \varphi(v_1+v_2) = \varphi(v_1) + \varphi(v_2)$ (Additivit"at)
		\item
		$\forall v \in V, \forall \lambda \in K: \varphi(\lambda \cdot v) = \lambda \cdot \varphi(v)$ (Homogenit"at)
	\end{itemize}
	\item
	Ist die lineare Abbildung $\varphi: V \rightarrow W$ bijektiv, so heißt $\varphi$ \emph{Isomorphismus}, $V$ und $W$ heißen dann \emph{isomorph}, $V \cong W$.
\end{enumerate}
%5.2
\subsection{Bemerkung}
$\varphi: V \rightarrow W$ ist eine lineare Abbildung:
\begin{enumerate}
	\item
	$\varphi( \zerovec) = \zerovec$
	\item
	$\varphi \left(\sum_{i=1}^{n} \lambda_i v_i \right) = \sum_{i=1}^{n} \lambda_i \varphi(v_i)$ 
\end{enumerate}
%5.3
\subsection{Beispiel}
\begin{enumerate}
	\item
	Nullabbildung:\\
	$\varphi: V \rightarrow W, v \mapsto \zerovec$
	\item
	$\varphi : V \rightarrow V, v \mapsto \lambda v$ f"ur jedes festes $\lambda \in K$ ist lineare Abbildung $(\lambda = 1: \varphi = \id_V )$
	\item
	$\varphi : \R^3 \rightarrow \R^3, \vct{x_1\\ x_2\\ x_3}\mapsto \vct{x_1 \\ x_2 \\ z_3}$ ist eine lineare Abbildung (Spiegelung an $x_1, x_2$-Ebene) 
	\item
	$\varphi : \R^ 2 \rightarrow R^2, \left(	\vct{x_1 \\ x_2} \mapsto \vct{(x_1)^2 \\ x_2 }\right)$ ist nicht linear
	
	$v= \vct{x_1 \\ x_2}, \lambda =3:$
	
	 $\varphi(3v)= \varphi\left(\vct{3 \\ 6}\right)=  \vct{9 \\ 6} \neq \vct{3 \\ 9} = 3 \cdot \vct{1\\2}= 3 \cdot \varphi \left(\vct{1\\2} \right) = 3 \cdot \varphi(v)$
	 \\

\end{enumerate}

% % % 5.4
\subsection{Satz}

$A \in M_{m,n}(K)$

Dann ist $\varphi: K^n \rightarrow K^m$, $x \mapsto Ax$
%TODO: Zeichnung einfügen

eine lineare Abbildung \bigskip

% % %
\subsubsection*{Beweis}

folgt aus Rechenregeln für Matrizen:

\begin{align*}
\varphi(x+y) = A(x+y) &= Ax + Ay\\
&= \varphi(x) + \varphi(y)\\
\\
\varphi(\lambda \cdot x) = A(\lambda x) &= \lambda A x\\
&= \lambda \varphi(x)
\end{align*}
\qed

Alle bisherigen Beispiele waren von dieser Form!

5.3
\begin{enumerate}
\item $A = 0 = Nullmatrix$

\item $A =
\begin{pmatrix}
\lambda & \cdots & 0\\
\vdots & \ddots & \vdots\\
0 & \cdots & \lambda
\end{pmatrix} =
\lambda \cdot E_n
$
\medskip

\item $A =
\begin{pmatrix}
1 & \cdots & 0\\
\vdots & \ddots & \vdots\\
0 & \cdots & 1
\end{pmatrix}
$ 
\end{enumerate}

Es gilt ($\rightarrow$ später):

\underline{alle} lineare Abbildungen $K^n \rightarrow K^m$ sind von der Form in $5.4$
%TODO

% % % 5.5
\subsection{Satz}

$\varphi: V \rightarrow W$ lineare Abbildung

\begin{enumerate}[(i)]
	\item
	$U \subseteq V$ UR von $V$
	
	$\Rightarrow \varphi(U) \subseteq W$ UR von $W$ und $\varphi(V)$ (Bild von $V$) ist UR von $W$
	
	\item
	falls $\dim(U) {\text{ endlich}}: \dim(\varphi(U)) \leq \dim(U)$
\end{enumerate}

\subsubsection*{Beweis}

\begin{enumerate}[(i)]
	\item
	$U \subseteq V$ Unterraum, d.h.\ für $u,v \in U$ ist $\lambda u + \mu v \in U$
	
	$\varphi(U) = \{\varphi(u) \big | u \in U\}$ ist auch UR:
	
	für $\varphi(u), \varphi(v) \underset{\text{lin. Abb.}}{=} \varphi(\lambda u + \mu v) \in \varphi(U)$
	
	außerdem ist $\varphi(U) \neq \emptyset$, da $\varphi(\zerovec) = \zerovec$
	
	\item
	$v_1,\dots,v_k$ Basis von $U$
	
	$\Rightarrow \varphi(u_1),\cdots,\varphi(u_k)$ ist Erzeugendensystem von $\varphi(U)$
	
	$\Rightarrow$ enthält Basis (Mathe II)
	
	$\Rightarrow$ Behauptung
	\qed 
\end{enumerate}

% % % 5.6
\subsection{Definition}

$\varphi: V \rightarrow W$ lineare Abbildung, $V$ endlich dimensional

Dann heißt die $\dim(\varphi(V))$ der \underline{Rang von $\varphi$}, $\rg(\varphi)$.

% % % 5.7
\subsection{Definition/Satz}
	\label{kern}

$\varphi: V \rightarrow W$ lineare Abbildung

\begin{enumerate}[(i)]
	\item
	$\ker(\varphi) := \{v \in V\; \big |\; \varphi(v) = \zerovec\}$
	
	(alle Vektoren die von $\varphi$ auf $\zerovec$ abgebildet werden)
	
	heißt der \underline{Kern von $\varphi$} und ist ein UR von $V$.
	

	\item
	$\varphi:$ injektiv $\Leftrightarrow \ker(\varphi) = \{\zerovec\}$
\end{enumerate}

\subsubsection*{Beweis}

\begin{enumerate}[(i)]
	\item
	$\ker(\varphi)$ ist UR:
	
	\begin{itemize}
		\item
		$\ker(\varphi) \neq \emptyset$, da $\varphi(\zerovec) = \zerovec$
		
		\item
		seien $u,v \in \ker(\varphi)$, d.h. $\varphi(u) = \zerovec, \varphi(v) = \zerovec$, seien $\lambda, \mu \in K$
		
		$\Rightarrow \lambda u + \mu v \in \ker(\varphi)$, dann:
		
		$\varphi(\lambda u + \mu v) \underset{\text{lin. Abb.}}{=} \lambda \cdot \underbrace{\varphi(u)}_{\zerovec} +
		\mu \cdot \underbrace{\varphi(v)}_{\zerovec} = \zerovec$
	\end{itemize}
	
	\item
	"$\Rightarrow$"\\
	$\varphi(\zerovec) = \zerovec$, wegen Injektivität kann kein weiteres Element auf $\zerovec$ abgebildet werden.
	
	"$\Leftarrow$"\\
	Angenommen es gibt $v_1,v_2 \in V$ mit $\varphi(v_1) = \varphi(v_2)$, dann ist $\zerovec = \varphi(v_1) - \varphi(v_2) \\= \varphi(v_1 - v_2)$ (lineare Abbildung!)
	
	$\Rightarrow v_1 - v_2 = \zerovec$ (nur $\zerovec$ wird auf $\zerovec$ abgebildet)
	
	$\Rightarrow v_1 = v_2$
	
	$\Rightarrow \varphi$ injektiv
\end{enumerate}
\qed

% % % 5.8
\subsection{Beispiel}

$\varphi: \mathbb{R}^3 \rightarrow \mathbb{R}^3, \;
\begin{pmatrix}
x_1\\x_2\\x_3
\end{pmatrix} \rightarrow
\begin{pmatrix}
x_1\\2x_1\\x_1 + x_2 + 2x_3
\end{pmatrix}
\;$
 ist lineare Abbildung\bigskip

$
A =
\begin{pmatrix}
1 & 0 & 0 \\
2 & 0 & 0 \\
1 & 1 & 2 \\
\end{pmatrix}
$\bigskip

$U = \big <e_2,e_3\big >, \quad \dim(U) = 2$\bigskip

$\varphi(U),\; dim(\varphi(U)),\; ker(\varphi)$?\bigskip

$\varphi(U) = \big <\varphi(e_2), \varphi(e_3)\big > = \bigg <
\color{red}
\begin{pmatrix}
0\\0\\1
\end{pmatrix}\color{black},\color{blue}
\begin{pmatrix}
0\\0\\2
\end{pmatrix}\color{black}\bigg > = \bigg <
\begin{pmatrix}
0\\0\\1
\end{pmatrix}\bigg > = x_3$-Achse\bigskip

$\varphi(e_2) = \varphi(
\begin{pmatrix}
0\\1\\0
\end{pmatrix}) = \color{red}
\begin{pmatrix}
0\\0\\1
\end{pmatrix} \color{black} \quad
\varphi(e_3) = \varphi(
\begin{pmatrix}
0\\0\\1
\end{pmatrix}) = \color{blue}
\begin{pmatrix}
0\\0\\2
\end{pmatrix}\color{black}$\bigskip

$\dim(\varphi(U)) = 1$

% 5.8 - 5.13
% % % 5.9
\subsection{Satz}

$V,W$ $K$-VR, $\dim(V)=n$

$\{v_1, \dots, v_n\}$ Basis von $V$ \\
$w_1,\dots, w_n $ Vektoren aus $W$ (nicht notwendig verschieden)

Dann $\exists!$ lineare Abbildung



$\varphi: V \rightarrow W \;$mit$\; \varphi(v_i) = w_i \; (i=1,...,n)$

und zwar: 

$\left.
\begin{array}{l}
\varphi: V \rightarrow W\\
v=\sum_{i=1}^{n} \lambda_i v_i \mapsto \sum_{i=i}^{n} \lambda_i w_i
\end{array}
\right\rbrace$\huge*\normalsize\\

D.h.: wenn man weiß, wie die Basisvektoren abgebildet werden, dann kennt man die lineare Abbildung vollständig.

% % % 
\subsubsection*{Beweis}
Für $\varphi$ aus * gilt:
\begin{itemize}
	\item $\varphi$ ist linear
	\item $\varphi(v_i) = w_i$\\
	$\varphi(v_1)=\varphi(1\cdot v_1 + 0 \cdot v_2 + ... + 0 \cdot v_n) = 1 \cdot w_1 + 0 \cdot w_2 + ... + 0 \cdot w_n = 1 \cdot w_1 = w_1$ usw.
	\item $\varphi$ ist eindeutig.
	
	Angenommen $\exists \;\psi: V \rightarrow W$ lin. Abb. mit $\psi(v_i) = w_i \; \forall i=1...n$
	
	Dann ist $\psi(\sum_{i=1}^{n} \lambda_i v_i) = \sum_{i=1}^{n} \lambda_i(\psi(v_i)) = \sum_{i=1}^{n} \lambda_i w_i = \varphi(\sum_{i=1}^{n} \lambda_i v_i) $
	\qed
\end{itemize}

% % % 5.10
\subsection{Beispiel}
\label{beispiel:drehung}

$V = \R^2, \varphi$ Drehung um Winkel $\alpha\; (0 \leq \alpha < 2 \pi)$ um Nullpunkt gegen den Uhrzeigersinn.

$\varphi$ ist lin. Abb.:\\
$\varphi(\alpha_1 + \alpha_2) = \varphi(\alpha_1) + \varphi(\alpha_2)$\\
$\varphi(\lambda \alpha) = \lambda \varphi(\alpha)$\\

$\varphi: e_1 = \vct{1\\0} \mapsto \vct{\cos \alpha\\\sin \alpha}$

$e_2 = \vct{0\\1} \mapsto \vct{-\sin \alpha\\\cos \alpha}$

allg. Vektor $x = \vct{x_1\\x_2} = x_1 \cdot \vct{1\\0} + x_2 \cdot \vct{0\\1}$
\begin{align*}
\varphi: x \mapsto& \;x_1 \cdot \varphi(e_1) + x_2 \cdot \varphi(e_2)\\
=&\; x_1 \cdot \vct{\cos \alpha \\ \sin \alpha} + x_2 \cdot 
\vct{-\sin \alpha \\ \cos \alpha}\\
=& \vct{x_1 \cdot \cos \alpha - x_2 \cdot \sin \alpha \\
	x_1 \cdot \sin \alpha + x_2 \cdot \cos \alpha}\\
=&\; A \cdot x
\end{align*}
 mit $A = \begin{bmatrix}
\cos \alpha & \sin \alpha\\
-\sin \alpha & \cos \alpha
\end{bmatrix}$

% % % 5.11
\subsection{Satz (Dimensionsformel)}
\label{dimensionsformel}

$V$ endl. dim. $K$-VR, $\varphi: V \rightarrow W$ lin. Abb.

Dann gilt: \\$\dim(V) = \dim(\ker(\varphi)) +  \underbrace{\rg(\varphi)}_{\mathclap{\dim(\varphi(V))}}$

% % %
\subsubsection*{Beweis}
Sei $u_1,...,u_k$ Basis von ker($\varphi$)

Ergänze zu Basis $u_1,...,u_n$ von V (Mathe 2, Basisergänzungssatz)

Setze $U:= \left< u_{k+1},..., u_n \right>$

Dann ist $\ker(\varphi) \cap U = \{ \zerovec \}$, \\d.h. kein Element außer $\zerovec$ liegt in U, \\also hat die Abb. $\varphi|_U$ den 

$\ker(\varphi|_U) = \{ \zerovec \}$, 

ist damit nach Satz 5.7 (ii) injektiv. \\
Deshalb ist $\dim(U) = \dim(\varphi(U))$. 

Außerdem ist $\varphi(U) = \varphi(V)$\\
$\Rightarrow \dim(V) = \dim(\ker(\varphi)) + \underbrace{\dim(U)}_{\mathclap{\dim(\varphi(U)) = \dim(\varphi(V)) = \rg(\varphi)}}$

\qed
% % % 5.12
\subsection{Korollar}
$V,\; W$ endlich. dim. $K$-VR mit $\dim V = \dim W, \; \\\varphi: V \rightarrow W$ lin. Abb. 

Dann sind folgende Aussagen äquivalent:
\begin{enumerate}[(i)]
	\item $\varphi$ ist surjektiv
	\item $\varphi$ ist injektiv
	\item $\varphi$ ist bijektiv
\end{enumerate}

% % % 
\subsubsection*{Beweis}
$\dim V = \dim W = n$\\

Nach \ref{dimensionsformel} gilt:\\
$n = \dim(\ker(\varphi)) + \rg(\varphi)$

Also: $\underbrace{\rg(\varphi)=n}_{\mathclap{\varphi \text{ surjektiv}}} \Leftrightarrow \underbrace{\dim(\ker(\varphi))}_{\mathclap{\varphi \text{ injektiv (Satz~\ref{kern}})}} = 0\\
\Rightarrow$ Beh.
\qed

% % % 5.13
\subsection{Zusammenhang lin. Abb. und hom. LGS, Matrizen, Rang}
\begin{itemize}
	\item homogenes LGS: $A \in M_{m,n}(K)$ gesucht: \\
	Menge aller $x \in K^n$ mit $Ax = \zerovec$
	\item lin. Abb. dazu:\\
	 $\varphi: K^n \rightarrow K^m, x \mapsto Ax$\\
	Dann ist der Lösungsraum des homogenen LGS = $\ker(\varphi)$
\end{itemize}

Dimensionsformel:

$\underbrace{\dim(\ker(\varphi))}_{\mathclap{\dim(\text{Lösungsraum LGS})}} = \underbrace{\dim(K^n)}_{n} -
\underbrace{rg(\varphi)}_{\dim(\varphi(K^n))}$\bigskip
\begin{align*}
\varphi(K^n) &= \big < \varphi(e_1),\cdots,\varphi(e_n)\big>_K\\
&= \big < Ae_1,\cdots,Ae_n \big >
\end{align*}

($Ae_i$ ist gerade die i-te Spalte $S_i$ von A)

$\binom{1\quad2\quad3}{4\quad5\quad6} \cdot
\vct{1 \\ 0\\ 0} = \vct{1\\4} $\

Also:\\
 $\rg(\varphi) = \dim\big(\big < s_1,\cdots,s_n \big >_K\big) =$ Maximale Anzahl linear unabhängiger Spalten von A.\bigskip

Also: $\dim($Lösungsraum) $= n - $ Spaltenrang von A

Mathe II: $\dim($Lösungsraum) $= n - $ Zeilenrang von A

$\Rightarrow$ für beliebige $A \in M_{m,n}(K)$ gilt:

\begin{align*}
\text{Zeilenrang von } A &= \text{Spaltenrang von } A\\
&= \text{Zeilenrang von } A\\\\
\Rightarrow\text{f"ur beliebigen } A \in M_{m,n}(K) \text{ gilt:}\\
&= \text{Rang von A}\\
&= \rg(\varphi) \text{ mit } \varphi \text{ wie oben}
\end{align*}


% 6.1 - 6.15
 %6.1 - 6.15
 % % % % % % % % % % % % % % % % % % % % % % % % % % % % %
 
 
 
% % %  6
\section{Matrizen und lineare Abbildungen}

% % %  6.1
\subsection{Definition}

Seien $V,W$ endlich dimensionale VR mit geordneter Basis

\[\mathcal{B} = (v_1,\cdots,v_n) \text{ von } V\] und
\[\mathcal{C} = (w_1,\cdots,w_n) \text{ von } W\] Sei
\[\varphi:V\rightarrow W \text{ lineare Abbildung}\]
\\
Stelle die Bilder $\underbrace{\varphi(v_1)}_{\in W}, \ldots, \underbrace{\varphi(v_n)}_{\in W}$ bzgl. Basis $C$ dar:

\begin{align*}
\varphi(v_1) &= a_{11} \cdot w_1 + \cdots + a_{11}w_m\\
\vdots\\
\varphi(v_n) &= a_{1n} \cdot w_1 + \cdots + a_{n1}w_m\\
\end{align*}

Dann heißt die $m\times n$ Matrix
\[\left. A_\varphi^{\mathcal{B},\mathcal{C}} :=
\begin{pmatrix}
a_{11} \cdots a_{1n}\\
\vdots \hspace{1cm} \vdots\\
a_{m1}\cdots a_{m,n}
\end{pmatrix}\right. \;\text{(Spalte $i$ enthält Koordinaten von $\varphi(v_i)$ bzgl. $\mathcal{C}$)}\]

die \emph{Darstellungsmatrix} von $\varphi$ bzgl. der Basen $\mathcal{B}$ und $\mathcal{C}$ \\(Schreibweise f"ur den Fall $\mathcal{B} =
\mathcal{C}$, dann auch  $A_\varphi^\mathcal{B}$)

Bemerkung: $\varphi$ durch $A_\varphi^{\mathcal{B},\mathcal{C}}$ eindeutig bestimmt, vgl. \ref{kern}

% % %  6.2
\subsection{Beispiel}

\begin{enumerate}[a)]
	\item
	\[V=W=\mathbb{R}^2, \quad \mathcal{B}=\mathcal{\color{red}C\color{black}}=(e_1,e_2) =
	\bigg ( \color{red}\weirdvct{1}{0}\color{black},\color{red}\weirdvct{0}{1}\color{black}\bigg )\]
	\[\varphi: V \rightarrow V, \quad v \mapsto 2v\]
	
	\[A_\varphi^{\mathcal{B},\mathcal{C}} = A_\varphi^{\mathcal{B}} = \text{?}\]
	\begin{align*}
	\begin{rcases}
	\varphi\left(\weirdvct{1}{0}\right) &= \weirdvct{2}{0} = \underline{2} \color{red}\weirdvct{1}{0}\color{black} +
	\underline{0} \color{red}\weirdvct{0}{1}\color{black}\\
	\varphi\left(\weirdvct{0}{1}\right) &= \weirdvct{0}{2} = \underline{0} \color{red}\weirdvct{1}{0}\color{black} +
	\underline{2} \color{red}\weirdvct{0}{1}\color{black}\\
	\end{rcases}
	A_\varphi^\mathcal{B} = \weirdvct{2\quad0}{0\quad2}
	\end{align*}
	
	andere Basis $\mathcal{D} = \left( \weirdvct{1}{2},\weirdvct{0}{2}\right) \quad
	A_\varphi^{\mathcal{B},\color{green}\mathcal{D}\color{black}}$
	
	\begin{align*}
	\begin{rcases}
	\varphi\left(\weirdvct{1}{0}\right) &= \weirdvct{2}{0} = \underline{2} \color{green}\weirdvct{1}{2}\color{black} -
	\underline{2} \color{green}\weirdvct{0}{2}\color{black}\\
	\varphi\left(\weirdvct{0}{1}\right) &= \weirdvct{0}{2} = \underline{0} \color{green}\weirdvct{1}{2}\color{black} +
	\underline{1} \color{green}\weirdvct{0}{2}\color{black}\\
	\end{rcases}
	A_\varphi^{\mathcal{B},\color{green}\mathcal{D}\color{black}} = \weirdvct{2\quad0}{-2\quad1}
	\end{align*}
	
	\item
	$V = W$ mit $\dim V = n,\;\; \mathcal{B}$ bel. Basis,$\;\; \varphi = id_V,$ dann ist: \[A_\varphi^\mathcal{B} = E_n\]
	
	\item
	$V = W = \mathbb{R}^2,\; \mathcal{B} = \mathcal{C}= (e_1,e_2)$
		
	$\varphi$ Drehung um Nullpunkt um $\alpha$ gegen Uhrzeigersinn
	
	\[\Rightarrow A_\varphi^\mathcal{B} = \weirdvct{\cos \alpha \quad -\sin \alpha}{\sin \alpha\quad \cos \alpha}\]
	Vgl. Beispiel \ref{beispiel:drehung}
	
	\item
	$V=W=\mathbb{R}^2,\; \mathcal{B} = (e_1,e_2)$
	
	$\varphi:$ Spiegelung an der $\underbrace{\big<e_1\big>}_{\mathclap{x_1-\text{Achse}}}$, d.h. $\varphi:\weirdvct{x_1}{x_2}\mapsto \weirdvct{x_1}{-x_2}\quad
	A_\varphi^\mathcal{B} = \weirdvct{1\quad0}{0\quad-1}$
	
	andere Basis $\mathcal{B}' = \left( \weirdvct{1}{1},\weirdvct{1}{-1} \right)$
	
	$A_\varphi^{\mathcal{B},\mathcal{B}'} =$ ?
	
	\begin{align*}
	\varphi\left(\weirdvct{1}{0}\right) &= \weirdvct{1}{0} = a_{11}\weirdvct{1}{1} + a_{21}\weirdvct{1}{-1}\\
	\varphi\left(\weirdvct{0}{1}\right) &= \weirdvct{0}{-1} = a_{12}\weirdvct{1}{1} + a_{22}\weirdvct{1}{-1}\\
	\end{align*}
	
	$\Rightarrow$ LGS, ausrechnen, erhalte:
	\[A_\varphi^{\mathcal{B},\mathcal{B}'} = \weirdvct{\frac{1}{2}\quad \frac{1}{2}}{-\frac{1}{2}\quad \frac{1}{2}}\]
	
	\item
	andersherum:
	
	$V=W=\mathbb{R}, \quad \mathcal{B}=(e_1,e_2)$
	
	$A_\varphi^\mathcal{B} = \weirdvct{1\quad2}{3\quad4}$
	
	Was ist $\varphi\left(\weirdvct{7}{-5}\right)$
	
	$\varphi\left(\weirdvct{1}{0}\right) = \weirdvct{1}{3}$
	
	$\varphi\left(\weirdvct{0}{1}\right) = \weirdvct{2}{4} = \varphi\left(7 \weirdvct{1}{0} - 5 \weirdvct{0}{1}\right) = 7 \varphi\left(\weirdvct{1}{0}\right) -
	5 \varphi\left(\weirdvct{0}{1}\right) = 7 \weirdvct{1}{5} + (-5) \weirdvct{2}{4} = \weirdvct{-3}{1}$\\
	
	\underline{Gegeben:}
	
	Koordinaten eines Punktes bzgl. Basis $\mathcal{B}$ (z.B. Roboterkoordinaten), Abbildung $\varphi$\\
	
	\underline{Gegeben:}
	
	Koordinaten dieses Punktes bzgl. Basis $\mathcal{C}$ (Weltkoordinatensystem) $\rightarrow$ später
	
	Koordinaten des mit $\varphi$ abgebildeten Punktes bzgl. $\mathcal{C} \rightarrow$ jetzt
	
\end{enumerate}



% % %  6.3
\subsection{Satz}

$V,W,\mathcal{B},\mathcal{C},\varphi$ wie in 6.1\bigskip

Sei $v \in V,K_B(v)$ sei Koordinatenvektor von $v$ bzgl. $\mathcal{B}$ (enthält Koordinaten von $v$ bzgl. $\mathcal{B}$)

Dann lässt sich der Koordinatenvektor von $\varphi(b)$ bzgl. $\mathcal{C}$ berechnen als
\[K_\mathcal{C}(\varphi(V)) = A_\varphi^{\mathcal{B},\mathcal{C}} \cdot K_\mathcal{B}(v)\]

\emph{Beweis}: nacher

% % % 6.4
\subsection{Beispiel}

$\dim(V) = 3 \quad \mathcal{B} = (v_1,v_2,v_3) \quad \varphi: V \rightarrow W$\\
$\dim(W) = 2 \quad \mathcal{C} = (w_1,w_2)$

mit
\[A_\varphi^{\mathcal{B},\mathcal{C}} \weirdvct{1\quad1\quad-2}{2\quad0\quad3}\]

$v = \underline{5}\cdot v_1 \underline{-2} \cdot v_2 +\underline{4} \cdot v_3$, d.h. Koordinaten von $v$ bzgl. $\mathcal{B}$ sind $5,-2,4$

\[K_\mathcal{B} =
\aMatrix{c}
{5\\-2\\4}
\]

Was sind Koordinaten von $\varphi(v)$ in Basis $\mathcal{C}$?

\[
K_\mathcal{C} = \aMatrix{ccc}{1 & 1 & -2 \\ 2 & 0 & 3}\cdot
\aMatrix{c}{5\\-2\\4}= \weirdvct{-5}{22}
\]

d.h. $\varphi(v) = -5 \cdot w_1 + 22 \cdot w_2$\bigskip

\subsubsection*{Beweis}

\begin{align*}
A_\varphi^{\mathcal{B},\mathcal{C}} =
\aMatrix{c}{a_{11}\cdots a_{1n}\\\vdots\\a_{m1}\cdots a_{mn}},
\quad v = \sum\limits_{i=0}^n \lambda_i v_i, \quad K_\mathcal{B}(v) =
\aMatrix{c}{\lambda_1\\\vdots\\\lambda_n}
\end{align*}
\begin{align*}
A_\varphi^{\mathcal{B},\mathcal{C}} \cdot K_\mathcal{B}(v) =
\aMatrix{c}{\sum\limits_{i=1}^n a_{1i}\lambda_i\\\vdots\\\sum\limits_{i=1}^n a_{mi}\lambda_i}
\end{align*}
\begin{align*}
\varphi(v) &= \varphi\left(\sum\cdots\right)\\
&= \sum\limits_{i=1}^n \lambda_i \underbrace{\varphi(v_i)}_{\mathclap{\sum\limits_{k=1}^m a_{ki}w_k}}\\
&= \sum\limits_{k=1}^m \underbrace{\left(\sum\limits_{i=1}^n \lambda_i a_{ki}\right)}_{\mathclap{\text{Koordinaten von }
		\varphi(v) \text{ bzgl. } \mathcal{C}}} \cdot w_k\\
\end{align*}
\begin{align*}
K_\varphi(\varphi(v)) =
\aMatrix{c}{\sum\limits_{i=1}^n \lambda_i a_{1i}\\\vdots\\\sum\limits_{i=1}^n \lambda_i a_{mi}}
\end{align*}
\qed

% % % 6.5
\subsection{Bemerkung / Korollar zu 6.3}
Der Koordinatenvektor kann als Bild der ''Koordinatenabbildung''\\
$K_B: V\rightarrow K^n$\\
$v=\sum_{i=0}^n \lambda v_i \mapsto \begin{pmatrix}\lambda_1 \\ \vdots \\ \lambda_n\end{pmatrix}$\\
aufgefasst werden, dann erhalte folgende Übersicht\\
(dim V=n, Basis B) $\begin{array}{lcr}
V& \stackrel{\varphi}{\rightarrow} & W\\
\downarrow K_B & & \downarrow K_C\\
K^n & \stackrel{\rightarrow}{\text{Multiplikation mit } A_{\varphi}^{B,C}} & K_m (*)
\end{array}$ (dim W=m, Basis C)\\
(*): $\underbrace{K_{C}\varphi(v))}_{(*)}=A_{\varphi}^{B,C} K_B(v)$
Damit folgt:\\
jede lin. Abb $K^n\rightarrow K^m$ (K Körper) ist von der Form $\varphi(x)=Ax$ für ein $A \in M_{m,n}(K)$

\textbf{Beweis:}
benutze kanonische Basis von $K^n$ bzw. $K^m$. Dann stimmen Elemente von $K^n$ bzw. $K^m$ mit ihren Koodrdinatenvektoren bzgl. Basis überein, Beh. folgt aus 6.3

% % % 6.6
\subsection{Satz (Eigenschaften der Darstellungsmatrix)}
U, V, W VR mit Basen B, C, D\\
$\varphi_1,\varphi_2,\varphi: U\rightarrow V, \Psi: V\rightarrow W$
\begin{enumerate}
	\item
	$A_{\varphi_1+\varphi_2}^{B,C} = A_{\varphi_1}^{B,C} + A_{\varphi_2}^{B,C}$
	
	\item
	$A_{\lambda \varphi}^{B,C} = \lambda*A_{\varphi}^{B,C} \ (\lambda\in K)$
	
	\item
	$A_{\Psi_0\varphi}^{B,D} = A_{\Psi}^{C,D}*A_{\varphi}^{B,C}$
\end{enumerate}
(D.h.: Der Hintereinanderausführung von lin. Abb. entspricht das Matrixprodukt der Darstellungsmatrizen)

\textbf{Beweis:}
Übungsaufgabe\\
\hspace*{13cm}$\square$


% % % 6.7
\textbf{Folgerung:}
\subsection{Satz:}
V ein K-VR, dim(V)=n, Basis B\\
$\varphi: V\rightarrow V$ lin. Abb. mit $A_{\varphi}^B$\\
Dann gilt:\\
$\varphi$ invertierbar (bij.) $\Rightarrow A_{\varphi}^B$ invertierbar und $A_{\varphi^{-1}}^B$ ist dann $=(A_{\varphi}^B)^{-1}$

\textbf{Beweis:}
\begin{enumerate}
	\item[''$\Rightarrow$'']
	Sei $\varphi$ invertierbar, d.h. $\exists \varphi^{-1}$\\
	Dann ist $A_{\varphi}^B*A_{\varphi^{-1}}^B \underbrace{=}_{(6.6)} A_{\varphi\circ\varphi^{-1}}^B = A_{id}^B = E_n$\\
	analog $A_{\varphi^{-1}}^B=A_{\varphi}^B$
	
	\item[''$\Leftarrow$'']
	Sei $A_{\varphi}^B$ invertierbar, d.h. $\exists Y$ mit $A_{\varphi}^B*Y=Y*A_{\varphi}^B = E_n$\\
	Dann ist Y Abbildungsmatrix für eine eindeutig bestimmte lineare Abbildung $\Psi: V\rightarrow V,  \ Y=A_{\Psi}^B$\\
	$\stackrel{(6.6)}{\Rightarrow} A_{\varphi\circ\Psi}^B = A_{\varphi}^B*A_{\Psi}^B = E_n$\\
	d.h. $\varphi\circ\Psi=\Psi\circ\varphi=idv \Rightarrow \varphi$ ist bij. (invertierbar.)\\
	\hspace*{13cm}$\square$
	
	\textbf{Fragen:} wann ist eine Matrix (lineare Abbildung) invertierbar?\\
	Wie berechnet man inverse?
\end{enumerate}

% % % 6.8
\subsection{Satz:}
$A\in M_{n,n}(K)$\\
Dann gilt: A ist invertierbar $\Leftrightarrow \underbrace{\mathrm{rg}(A)=n}_{\text{d.h. alle Zeilen/Spalten sind l.u.}}$

\textbf{Beweis:}
Betrachte $\varphi: K^n\rightarrow K^n$ mit $\varphi(x)=Ax$\\
Dann ist $A = A_{\varphi}^B$ (B Basis von $K^n$)\\
A invertierbar $\stackrel{(6.7)}{\Leftrightarrow} \varphi$ invertierbar (bij.)\\
A invertierbar $\stackrel{(5.12)}{\Leftrightarrow} \varphi$ surjektiv\\
A invertierbar $\Leftrightarrow \mathrm{rg}(\varphi)=n$\\
A invertierbar $\stackrel{(5.13)}{\Leftrightarrow} \mathrm{rg}(A)=n$\\
\hspace*{13cm}$\square$

% % % 6.9
\subsection{Berechnung von Inversen}
$\rightarrow$ Blatt (Gauß) + Bsp.

\textbf{Gesehen:} Darstellungsmatrix hängt von der Wahl der Basen ab. Wie ändert sie sich, wenn man Basen ändert? Dieser Vorgang wird als Basistransformation bezeichnet.

% % % 6.10
\textbf{Dazu:}
\subsection{Definition/Satz:}
Sei V ein VR, $B=(v_1,\dots,v_n)$ und $B'=(v_1',\dots,v_n')$ Basis von V\\
Schreiben $v_i'$ als LK der Basisvektoren von B$(i=1\dots n)$, also\\
$v_1'=s_{11}v_1+\dots+s_{n1}v_n$\\
$v_n'=s_{1n}v_1+\dots+s_{nn}v_n$\\
Dann heißt $S_{B,B'}=\begin{pmatrix}s_{11} & s_{12} & \dots & s_{1n} \\ \vdots & \vdots & \dots & \vdots \\ s_{n1} & s_{n2} & \dots & s_{nn}\end{pmatrix}$ Basiswechselmatrix\\
Ihre Spalten sind die Koordinatenvektoren der Basisvektoren von $B'$ bzgl. B

Analog:\\
Stelle $v_k$ als LK der Basisvektoren von $B'$ das ($v_k=\sum_{l=1}^n t_{lk}v_l')$\\
erhalte so $S_{B',B} (=(t_{lk})_{l,k=1\dots n})$\\
Es gilt $(S_{B,B'})^{-1}=(S_{B',B})$\\
(nachrechnen: $S_{B,B'}*S_{B',B}=E_n$)

% % % 6.11
\subsection{Satz: Koordinaten umrechnen}
V mit $B,B'$ wie in 6.10, $v\in V$

Dann ist $K_{B'}*(v)=S_{B',B}*K_B(v)$

\textbf{Beweis:} $v=\sum_{k=1}^n \lambda_k*\underbrace{v_k}_{\sum_{l=1}^n t_{lk}v_l'}$, also $K_B(v)=\begin{pmatrix}\lambda_1 \\ \vdots \\ \lambda_n\end{pmatrix}$\\
also $V=\sum_{l=1}^n\underbrace{\left(\sum_{k=1}^n \lambda_k t_{lk}\right)}_{\text{neue Koordinaten (bzgl. B')}}*\underbrace{v_l'}_{\in B'}$

% % % 6.12
\subsection{Beispiel}
$V=\mathbb{R}^2, \ B=(e_1,e_2), \ B'=\left(v_1=\begin{pmatrix}1 \\ 1\end{pmatrix}, v_2=\begin{pmatrix}1 \\ -2\end{pmatrix}\right)$\\
$S_{B,B'}=\begin{pmatrix}1 & 1 \\ 1 & -2 \end{pmatrix}, \ \ v_1=1*e_1+1*e_2, \ v_2=1*e_1-2*e_2$\\
$S_{B',B}=(S_{B,B'})^{-1}=(\dots$ Gauß $\dots)=\begin{pmatrix}\frac{2}{3} & \frac{1}{3} \\ \frac{1}{3} & -\frac{1}{3}\end{pmatrix}$

$i=\begin{pmatrix}3 \\ 6\end{pmatrix} \ K_B(u)=\begin{pmatrix}3 \\ 6\end{pmatrix} \ (u=3*e_1+6*3_2)$\\
Koordinaten von u in Basis B'?\\
$K_{B'}(u)=S_{B',B}*K_B(u)=\begin{pmatrix}\frac{2}{3} & \frac{1}{3} \\ \frac{1}{3} & -\frac{1}{3}\end{pmatrix}*\begin{pmatrix}3 \\ 6\end{pmatrix}=\frac{4}{-1}$

(also ist $u=4*v_1-1*v_2$)

Mit der Basiswechselmatrix kann man auch Darstellungsmatrizen umrechnen:

% % % 6.13
\subsection{Satz: Darstellungsmatrizen umrechnen}
$\varphi: V\rightarrow W$ lin. Abb.\\
$B,B'$ Basen von V, $C,C'$ Basen von W

$\Rightarrow A_{\varphi}^{B',C'}=S_{C',C}A_{\varphi}^{B,C}S_{B,B'}$

\textbf{Beweis:} Sei $v\in V$\\
nach 6.3:\\
$A_{\varphi}^{B',C'}*K_{B'}(v)=K_{C'}(\varphi(v))$\\
Koordinatenwechsel nach C (6.11): $= S_{C',C}*K_{\varphi}(v))$\\
6.3: $=S_{C',C}*A_{\varphi}^{B,C}*K_B(v)$\\
Koordinatenwechsel nach B' (6.11): $S_{C',C}*A_{\varphi}^{B,C}*s_{B,B'}*K_{B'}(v)$\\
\hspace*{13cm}$\square$

% % % 6.14
\subsection{Korollar}
$\varphi: V\rightarrow V$ lin. Abb.\\
$B,B'$ Basen von V. $S:=S_{B,B'}$\\
$\Rightarrow A_{\varphi}^{B'} = S^{-1}A_{\varphi}^B S$

% % % 6.15
\subsection{Beispiel}
$V, B, B'$ wie in 6.12\\
$\varphi:$ Spiegelung an der $x_1$-Achse\\
$\Rightarrow A_{\varphi}^B = \begin{pmatrix}1 & 0 \\ 0 & -1\end{pmatrix}$\\
$A_{\varphi}^{B'} \stackrel{=}{6.14} \begin{pmatrix}\frac{2}{3} & \frac{1}{3} \\ \frac{1}{3} & -\frac{1}{3}\end{pmatrix}\begin{pmatrix}1 & 0 \\ 0 & -1\end{pmatrix}\begin{pmatrix}1 & 1\\ 1 & -2\end{pmatrix} = \begin{pmatrix}\frac{1}{3} & \frac{4}{3} \\ \frac{2}{3} & -\frac{1}{3}\end{pmatrix}$ %-mar
% 7.1 - 8.2
 %7.1 - 8.2
 % % % % % % % % % % % % % % % % % % % % % % % % % % % % %
 
% % %
% 7 %
% % %
\section{Determinanten}

% % % 7.1
\subsection{Definition}
$A\in M_n(K) \ \ \ i,j\in\lbrace 1,\dots,n\rbrace$\\
$A_{ij}\in M_{n-1}(K)$ sei die Matrix, die man aus A durch Streichen der i-ten Zeile und der j-ten Spalte erhält.

z.B. $A=\begin{pmatrix}1 & 2 & 3 \\ 4 & 5 & 6 \\  7 & 8 & 9\end{pmatrix} \ \ A_{11}=\begin{pmatrix}5 & 6 \\ 8 & 9\end{pmatrix} \ \ A_{32}=\begin{pmatrix}1 & 3 \\ 4 & 6\end{pmatrix}$

% % % 7.2
\subsection{Definition: Determinante, rekursive Def.}
$A\in M_n(K)$
\begin{itemize}
	\item[$n=1$] $A=(a)$, dann $\mathrm{det}(A):=a$
	\item[$n>1$] \begin{align*}
	{det}(A):&=\sum_{j=1}^n (-1)^{1+j}a_{1j}\cdot \mathrm{det}(A_{1j})\\
	&=a_{11}\cdot \mathrm{det}(A_{11}-a_{12}\cdot \mathrm{det}(A_{12})\\
	& \ +a_{13}\cdot \mathrm{det}(A_{13}-a_{14}\cdot \mathrm{det}(A_{14})\\
	& \ +\dots -\dots\\
	& \ \dots +/- a_{1n}\cdot \mathrm{det}(A_{1n})
	\end{align*}
\end{itemize}
$\mathrm{det}(A)$ heißt \textbf{Determinante} von A\\
(Formel heißt auch ''Entwicklung nach der 1. Zeile'' $\rightarrow$ später)

% % % 7.3
\subsection{Beispiel}
\begin{enumerate}
	\item
	$\mathrm{det}\begin{pmatrix}a_{11} & a_{12} \\ a_{21} & a_{22}\end{pmatrix}=a_{11}\cdot a_{22}-a_{12}\cdot a_{21}$
	
	\item
	$\mathrm{det}\begin{pmatrix}a_{11} & a_{12} & a_{13} \\ a_{21} & a_{22} & a_{23} \\ a_{31} & a_{32} & a_{33}\end{pmatrix} = a_{11}\cdot (a_{22}\cdot a_{33}-a_{23}\cdot a_{32} -a_{12}\cdot (a_{21}\cdot a_{33}-a_{23}\cdot a_{31}) + a_{13}\cdot (a_{21}\cdot a_{32}-a_{22}\cdot a_{31}) = \dots$\\
	Regel von Sarrus:\\
	$\begin{pmatrix}a_{11} & a_{12} & a_{13} & a_{11} & a_{12}\\ a_{21} & a_{22} & a_{23} & a_{21} & a_{22} \\ a_{31} & a_{32} & a_{33} & a_{31} & a_{32}\end{pmatrix}$ Diagonalen von links oben nach rechts unten addieren, Diagonalen von rechts oben nach links unten subtrahieren
	
	\item
	für $n\times n$-Matrix erhalte $n!$ Summanden (nicht schön! $n=5: 120$, $n=10: 3.6$Mio)
	
	\item
	Ist A eine obere oder untere Dreiecksmatrix ist, so lässt sich $\mathrm{det}(A)$ einfach berechnen:\\
	$A=\begin{pmatrix}a_{11} \\ a_{21} & a_{22} \\ \dots & \dots & \dots \\ a_{n1} & \dots & \dots & \dots & a_{nn}\end{pmatrix}, \ \mathrm{det}(A)=a_{11}\cdot a_{22}\cdot \dots\cdot a_{nn}$
	
	klar für $n=1$: $A=(a)$\\
	$n>1$: $\mathrm{det}(A)= a_{11}\cdot \mathrm{det}(B)-\underbrace{a_{12}\mathrm{det}( )}_{0} +\underbrace{\dots}_{0}$, B: A ohne erste Spalte und erste Zeile\\
	Beweis durch Induktion
	
	\item
	damit klar: $\mathrm{det}(E_n)=1$
\end{enumerate}

% % % 7.4
\subsection{Entwicklungssatz von Laplace}
$A\in M_n(K)$
 \begin{enumerate}
 	\item
 	Entwicklung nach der i-ten Zeile:\\
 	für $i\in \lbrace 1,\dots,n\rbrace $ gilt:
 	%
 	\[\mathrm{det}(A)=\sum_{j=1}^n (-1)^{i+j} a_{ij}\mathrm{det}(A_{ij})\]
 	%
 	\item
 	Entwicklung nach der j-ten Spalte:
 	
 	für $i \in \aset{1, \dots, n}$ gilt:
 	%
 	\[\det(A) = \sum_{i=1}^{n}(-1)^{i+j}a_{ij}\det(A_{ij})\]
 	%
 	$(-1)^{i+j} \rightsquigarrow \aMatrix{cccccc}{
 	+ & - & + & - & + & \dots \\
 	- & + & - & + & \dots \\
 	+ & - & + & \dots \\
 	\dots}$
 \end{enumerate}
 
 % % % 7.5
 \subsection{Beispiel}
 
 
 $A = \aMatrix{ccc}{
 2 	& 1	& 1 \\
 -1	& 0 & 3 \\
 2	& 0 & 4}  \in M_3(\R)$


Mit Definition 7.2 %TODO Label
(Entwicklung nach der 1. Zeile):

$\det(A) = 2 \cdot \det \aMatrix{cc}{0 & 3 \\ 0 & 4} 
	- 1 \cdot \det \aMatrix{cc}{-1 & 3 \\ 2 & 4}
	+ 1 \cdot \det \aMatrix{cc}{-1 & 0 \\ 2 & 0}
  = 2 \cdot 0 - 1 \cdot (-10) + 1 \cdot 0 
  = 10$ 
	
oder: Entwicklung nach der 3. Zeile:

$\det(A) = 2 \cdot \aMatrix{cc}{1 & 1 \\ 0 & 3} 
	- 0 \cdot \det (\dots) 
	+ 4 \cdot \det \aMatrix{cc}{2 & 1 \\ -1 & 0}
  = 2 \cdot 3 - 0 + 4 \cdot 1 
  = 10$ 
  
oder (besser): Entwicklung nach der 2. Spalte:

$\det(A) = -1 \cdot \det \aMatrix{cc}{-1 & 3 \\ 2 & 4}
	+ 0 \cdot \det (\dots)
	- 0 \cdot \det (\dots)
  = -1 \cdot (-10)
  = 10$
  
Also: Es ist geschickt, nach einer Spalte oder Zeile zu entwickeln, in der viele Nullen stehen.

Falls es wenig Nullen gibt: Zuerst Gauß anwenden (Auchtung: det. ändert sich dabei eventuell!)

% % % 7.6
\subsection{Bemerkung}

Aus 7.4 %TODO Label
folgt $\det(A)=\det(A^T)$


% % % 7.7
\subsection{Satz (Eigenschaften der Determinanten)}

$A, B \in M_n(K), s_1, \dots, s_n$ Spalten von $A$,
$s_i' \in K^n, \lambda \in K$

\begin{enumerate}
	\item[(D1)]
	$\det(s_1, \dots, \underbrace{s_i+s_i'}_{i}, \dots, s_n)
		= \det(s_1, \dots, s_i, \dots, s_n) 
			+ \det(s_1, \dots, s_i', \dots, s_n)$
			
	\item[(D2)]
	Beim Vertauschen zweier Spalten von $A$ ändert sich das Vorzeichen von $\det(A)$
	
	\item[(D3)]
	$\det(s_1, \dots, \underbrace{\lambda \cdot s_i}_{i}, \dots, s_n) = \lambda \cdot \det(s_1, \dots, s_i, \dots, s_n)$
	
	(Beweis D1-D3 folgt aus 7.2 \& 7.4) %TODO Label
	
	\item[(D4)]
	$\det(\lambda \cdot A) = \det(\lambda s_1, \dots, \lambda s_n) \stackrel{(D3)}{=} \lambda^n \cdot \det(A)$
	
	\item[(D5)]
	Ist eine Spalte von $A$ gleich $\zerovec$, so ist $\det(A) = 0$
	
	\item[(D6)]
	Besitzt $A$ zwei identische Spalten, so ist $\det(A) = 0$
	\\ (Vertausche identische Spalten, erhalte Matrix $A' = A$.
	Nach (D2): $\det(A)=-\det(A')=-\det(A)$.
	Dies ist nur möglich, falls $\det(A)=0$ (oder in Körper mit $1+1=0$. Hier anders beweisen!))
	
	\item[(D7)]
	$\det(s_1, \dots, \underbrace{s_i + \lambda s_j}_i, \dots, s_n) = \det(A) \quad (i \neq j)$
	\\	mit D1, D3, D6
	
	\item[(D8)]
	$\det(A \cdot B) = \det(A)\cdot\det(B)$
	
	
\end{enumerate} 

Analog mit Zeilen statt Spalten!

Vorsicht: Im Allgemeinen ist $\det(A+B) \neq \det(A)+ \det(B)$


% % % 7.8
\subsection{Bemerkung / Beispiel}

Also: Erzeuge mit Gaußelimination viele Nulleinträge (! D2, D3: det ändert sich)
 
D7: det bleibt, entwickele nach guter Zeile / Spalte, oder bringe Matrix auf obere/untere $\triangle$-Form

z.B. 
\\$\det\aMatrix{ccc}{
0 & 1 & 2 \\
-2& 0 & 3 \\
2 & -2& 3}
	\stackrel{z_1\leftrightarrow z_2}{=}
	- \det\aMatrix{ccc}{
-2 & 0 & 3 \\
0  & 1 & 2 \\
0  & -2& 3}
	\stackrel{III=2\cdot II+III}{=}
	- \det \aMatrix{ccc}{
-2 & 0 & 3 \\
0  & 1 & 2 \\
0  & 0 & 7}
	= 
	- (-2) \cdot 1 \cdot 7 = 14$
 
 
% % % 7.9
\subsection{Satz (Charakterisietung invertierbarer Matrizen über det)}

$A \in M_n(K)$ ist invertierbar $\gdw \det(A) \neq 0$

In diesem Fall gilt:
\[\det(A^{-1}) = (\det(A))^{-1} \]

\subsubsection*{Beweis} 
\begin{itemize}
	\item[''$\Rightarrow$'':]
	
	Sei $A$ invertierbar, d.h. $\exists A^{-1}$ mit $A\cdot A^{-1} = A^{-1}\cdot A = E_n$
	
	$\Rightarrow \underbrace{\det(A\cdot A^{-1})}_{(D8): \det(A) \cdot \det(A^{-1})} = \det(E_n) = 1$
	
	 $\Rightarrow \det(A) \neq 0$ und $\det(A^{-1}) = (\det(A))^{-1}$
	 
	 \item[''$\Leftarrow$'':]
	 
	 Sei $A$ nicht invertierbar.
	 
	 $\Rightarrow \rg(A) < n
	 \\ \stackrel{6.8}{\Rightarrow}$ Spalten von $A$ sind l.a. %TODO Label
	 \\ d.h. $\exists i$ mit $s_i = \sum_{k=1\\k\neq i}^{n}\lambda_k s_k$ ($s_i$ als LK der anderen Spalten)
	 
	 $\det(A) \stackrel{(D7)}{=} \det(s_1, \dots, s_i- \sum\lambda_ks_k, \dots, s_n)
	 \\ = \det(s_1, \dots, \zerovec, \dots, s_n) 
	 \stackrel{(D5)}{=} 0$
	
	
\end{itemize} 

% % % 7.10
\subsection{Bemerkung}

Berechnung von $A^{-1}$ mittels 6.9 (Gauß mit $(A\;|\;E_n))$ oder auch mittels det, vgl. Übungsblatt 10, A1

$A = \aMatrix{cc}{a & b \\ c & d} \in M_2(\R) \Rightarrow A^{-1} = \frac{1}{\det A} \cdot \aMatrix{cc}{d & - b \\ - c & a}$



\section{Eigenwerte und Eigenvektoren}

% % % 8.1
\subsection[Definition (Eigenwert)]{Definition}

Sei $A \in M_n(K)$.

Ein Skalar $\lambda \in K$ heißt \emph{Eigenwert} von $A$, wenn es einen Vektor $\zerovec \neq x \in K^n$ gibt (,,nichttrivial'', d.h. $\neq 0$) mit

\[A\cdot x = \lambda x\]

Jedes solche $x$ heißt dann ein zu $\lambda$ geh"origer \emph{Eigenvektor} von $A$ und $\Eig(\lambda)=\Eig_A(\lambda)=\{x \in K^n | Ax=\lambda\}$ (alle zu $\lambda$ geh"or. EV \& der Nullvektor $\zerovec$) der $\lambda$ zugeordnete $\emph{Eigenraum}$.

% % % 8.2
\subsection{Satz}

$\lambda \in K$ ist Eigenwert von $A \in A \in M_n(K) \Leftrightarrow \det(A-\lambda \cdot E_n) = \zerovec)$,

und die zu $\lambda$ geh"origen Eigenvektoren sind genau die nichttrivialen L"osungen des LGS.

$[A - \lambda \cdot E_n] x = \zerovec$, also $\Eig_A(\lambda_ = \ker(A- \lambda \cdot E_n)$.

\subsubsection*{Beweis}

$Ax = \lambda x \Leftrightarrow Ax = \lambda \cdot E_n \cdot x \Leftrightarrow (A - \lambda E_n) x = \zerovec$

Also: $\lambda$ Eigenwert von $A \Leftrightarrow (A- \lambda \cdot E_n)x = \zerovec$ hat noch andere L"osungen als 

$x = \zerovec$\\
$\underbrace{\Leftrightarrow}_{\text{Mathe III}} \rg(A-\lambda \cdot E_n) < n$\\
$\underbrace{\Leftrightarrow}_{6.8}(A-\lambda \cdot E_n) \; \text{nicht invertierbar}$\\
$\underbrace{\Leftrightarrow}_{7.9}\det(A-\lambda \cdot E_n) = 0$

$x$ Eigenvektor $\Leftrightarrow x \neq \zerovec$ und $(A- \lambda \cdot E_n) x = \zerovec$


 %-mar
% 8.3 - 8.9
 %8.3 - 8.9
 % % % % % % % % % % % % % % % % % % % % % % % % % % % % %
 
% % % 8.3
\subsection{Definition}

F"ur $A \in M_n(K)$ heißt $p_a(\lambda):=det(A-\lambda \cdot E_n)$ das \emph{charakteristische Polynom} von $A$.

% % % 8.4
\subsection{Beispiel}
$A=\begin{pmatrix}1 & 1 \\ -2 & 4\end{pmatrix} \in M_2(\mathbb{R}$\\
Eigenwerte, Eigenvektoren, Eig(A), $p_A(\lambda)$?

$A-\lambda\cdot E_2 = \begin{pmatrix}1 & 1 \\ -2 & 4\end{pmatrix}-\lambda\cdot \begin{pmatrix}1 & 0 \\ 0 & 1\end{pmatrix} = \begin{pmatrix}1-\lambda & 1\\ -2 & 4-\lambda\end{pmatrix}$
\begin{itemize}
	\item
	$p_A(\lambda)=\mathrm{det}\begin{pmatrix}1-\lambda & 1 \\ -2 & 4-\lambda\end{pmatrix} = (1-\lambda)(4-\lambda)-(1\cdot (-2)) = \lambda^2-5\lambda+6 = (\lambda-2)(\lambda-3)$
	
	\item
	Eigenwerte von A:\\
	$\lambda\in W$ von A $\stackrel{8.2}{\Leftrightarrow} p_A(\lambda)=0 \Leftrightarrow \lambda=2$ oder $\lambda=3$\\
	$\Rightarrow \lambda_1=2, \lambda_2=3$ Eigenwerte von A
	
	\item
	Eigenvektoren von A:\\
	x ist EV von A zum EW $\lambda_1 \Leftrightarrow x\neq \mathcal{O}$ und $(A-\lambda_1E_2)x=\mathcal{O}$
	
	also $\begin{pmatrix}1-2 & 1 \\ -2 & 4-2\end{pmatrix}x=\begin{pmatrix}0 \\ 0\end{pmatrix} \Leftrightarrow \begin{pmatrix}-1 & 1 \\ -2 & 2\end{pmatrix}\begin{pmatrix}x_1 \\ x_2\end{pmatrix}=\begin{pmatrix}0 \\ 0\end{pmatrix}$\\
	$\begin{pmatrix}-1 & 1 & | & 0 \\ -2 & 2 & | & 0\end{pmatrix}\Leftrightarrow\begin{pmatrix}-1 & 1 & | & 0 \\ 0 & 0 & | & 0\end{pmatrix}$\\
	$-x1+x2=0$ ($x_2$ ist freie Variable)\\
	$\Leftrightarrow x_1=x_2$\\
	Lösung $\left\lbrace \begin{pmatrix}x_1 \\ x_2\end{pmatrix}\in \mathbb{R}^2 | x_1=x_2\right\rbrace$ alternativ $=<\begin{pmatrix}1 \\ 1\end{pmatrix}>_{\mathbb{R}}$\\
	(oder wähle z.B. $x_2=1 \Rightarrow x_1=1$, also ist $\begin{pmatrix}1 \\ 1\end{pmatrix}$ Lösung, restliche Lösungen sind $<\begin{pmatrix}1 \\ 1\end{pmatrix}>_{\mathbb{R}}$)\\
	$\mathrm{Eig}_A(\lambda_1)=\mathrm{Ker}\begin{pmatrix}-1 & 1 \\ -2 & 2\end{pmatrix}=<\begin{pmatrix}1 \\ 1\end{pmatrix}>$\\
	x ist EV von A zum EW $\lambda_2 \Leftrightarrow x_\neq 0$ und $\begin{pmatrix}-2 & 1 \\ -2 & 1\end{pmatrix}x=\begin{pmatrix}0 \\ 0\end{pmatrix}$\\
	$\mathrm{Eig}_A(\lambda_2)=\mathrm{Ker}\begin{pmatrix}-2 & 1 \\ -2 & 1\end{pmatrix}=<\begin{pmatrix}1 \\ 2\end{pmatrix}>$\\
	zu Lösung von homogenen LGS: siehe Blatt im Moodle
\end{itemize}

% % % 8.5
\subsection{Anwendungen}
\begin{enumerate}
	\item
	Matrixpotenzen\\
	Berechne $A^{2015}=\underbrace{A\cdot A\cdot \ldots\cdot A}_{\text{2015 mal}}$ für $A=\begin{pmatrix}1 & 1 \\ -2 & 4\end{pmatrix}$ aus Bsp. 8.4\\
	Definiere $S:=\begin{pmatrix}1 & 1 \\ 1 & 2\end{pmatrix}, \ \begin{pmatrix}1 \\ 1\end{pmatrix}$ EV zu $\lambda_1, \ \ \begin{pmatrix}1 \\ 2\end{pmatrix}$ EV zu $\lambda_2$\\
	$S^{-1}\stackrel{7.10}{=}\frac{1}{\mathrm{det}S}\cdot \begin{pmatrix}2 & -1 \\ -1 & 1\end{pmatrix}=\begin{pmatrix}2 & -1 \\ -1 & 1\end{pmatrix}$\\
	dann ist $A=S\cdot \underbrace{\begin{pmatrix}2 & 0 \\ 0 & 3\end{pmatrix}}_{D}\cdot S^{-1}$, D=Diagonalmatrix (stimmt, nachrechnen!)\\
	$\Rightarrow A^{2015}=(SDS^{-1})^{2015} = (SD\underbrace{S^{-1})\cdot (S}_{E_2}DS^{-1})\cdot (SDS^{-1})\cdot \dots\cdot (SDS^{-1})$\\
	$=S\cdot D^{2015}\cdot S^{-1}$\\
	$=S\cdot \begin{pmatrix}2^{2015} & 0 \\ 0 & 3^{2015}\end{pmatrix}S^{-1}$
	
	Mit lin. Abb./Darstellungsmatr. ausgedrückt:\\
	$\varphi: \mathbb{R}^2\rightarrow\mathbb{R}^2$ mit $A=A_{\varphi}^{\mathcal{B}}=\begin{pmatrix}1 & 1 \\ -2 & 4\end{pmatrix}, \ \mathcal{B}$ kanon. Basis\\
	Bezügl. Basis $\mathcal{B}'=\left(\begin{pmatrix}1 \\ 1\end{pmatrix},\begin{pmatrix}1 \\ 2\end{pmatrix}\right)$ hat Darstellungsmatrix Diagonalgestalt $A_{\varphi}^{\mathcal{B}'}=\begin{pmatrix}2 & 0 \\ 0 & 3\end{pmatrix}$
	
	Bem.: nicht jede Darstellungsmatrix lässt sich auf Diagonalgestalt bringen, z.B. $A=\begin{pmatrix}0 & -1 \\ 1 & 0\end{pmatrix}\in M_2(\mathbb{R})$, Drehung um $90^{\circ}$\\
	$\mathrm{det}(A-\lambda E_2)=\mathrm{det}\begin{pmatrix}-\lambda & -1 \\ 1 & -\lambda\end{pmatrix}=\lambda^2+1$, keine nullstellen in $\mathbb{R}$, also keine reelen Eigenwerte!
	
	\item
	\begin{itemize}
		\item Physik: Schwingungen, Eigenfrequenz, Tacoma Narrows Bridge
		\item Googles PageRank-Algorithmus
		\item Eigenfaces / Zähne ...\\
		$\vdots$
	\end{itemize}
\end{enumerate}

% % % 8.6
\subsection{Bemerkung}
Für $A\in M_n(K)$ ist $p_A(\lambda)=\mathrm{det}(A-\lambda E_n)=\mathrm{det}\begin{pmatrix}a_{11}-\lambda & a_{12} & \dots & a_{1m} \\ a_{21} & a_{22}-\lambda & \dots & \dots \\ \dots & \dots & \dots & \dots \\ a_{n1} & \dots & \dots & a_{nm}-\lambda\end{pmatrix}$\\
ein Polynom vom Grad n (folgt aus Def. der Det.)\\
Nullstellen von $p_A(\lambda)$ sind $\in W$ von A\\
$\Rightarrow K=\mathbb{R}: \le n$ Eigenwerte\\
$K\in\mathbb{C}: $ genau n Eigenwerte (nicht notwendig verschieden)

% % % 8.7
\subsection{Definition: diagonalisierbar}
\begin{enumerate}
	\item
	Eine Matrix $A\in M_n(K)$ heißt \textbf{diagonalisierbar}, wenn eine invertierbare Matrix $S\in M_n(K)$ existiert, so dass $A=SDS^{-1}$ gilt, wobei $D=\begin{pmatrix}\lambda_1 & \dots & 0 \\ \dots & \dots & \dots  \\ 0 & \dots & \lambda_n\end{pmatrix}$ Diagonalmatrix ist. (die $\lambda_i$ sind dann gerade die Eigenwerte von A, siehe 8.8)\\
	(Bem.: Dann gilt auch $D=S^{-1}AS$)
	
	\item
	für lin. Abb:\\
	Eine lin. Abb. $\varphi: V\rightarrow V$ heißt \textbf{diagonalisierbar}, falls $V$ eine Basis $\mathcal{B}$ aus Eigenvektoren (zur zugehörigen Darstellungsmatrix) besitzt, d.h. $A_\varphi^\mathcal{B}$ ist Diagonalmatrix.
\end{enumerate}
Ist jede Matrix diagonalisierbar? 

% % % 8.8
\subsection{Satz: Spektralsatz}
\begin{enumerate}
	\item
	$A\in M_n(K)$ ist diagonalisierbar $\Leftrightarrow$ Es gibt $n$ l.u. Eigenvektoren von A
	
	\item
	Besitzt A n verschiedene Eigenwerte, so ist A diagonalisierbar.
\end{enumerate}

\textbf{Beweis:}
\begin{enumerate}
	\item
	A diagonalisierbar, d,h, $\exists S$ invertierbar mit\\
	$S^{-1}AS=\begin{pmatrix}\lambda_1 & \dots & 0 \\ \dots & \dots & \dots  \\ 0 & \dots & \lambda_n\end{pmatrix} \Leftrightarrow  AS=S\cdot \begin{pmatrix}\lambda_1 & \dots & 0 \\ \dots & \dots & \dots  \\ 0 & \dots & \lambda_n\end{pmatrix}$
	
	Sei $S=(s_1,\dots,s_n)$ (s=Spalten) Für die i-te Spalte $s_i$ von S gilt dann $A_{si}=\lambda_i\cdot s_i \ \ (i=1,\dots,n)$\\
	Also ist $s_i$ Eigenvektor zum EW $\lambda_i$ von A\\
	S ist invertierbar $\Leftrightarrow$ Spalten $s_1,\dots,s_n$ l.u. (Satz 6.8)
	
	\item
	zeige per Induktion, dass die zugehörigen Eigenvektoren linear unabhängig sind, Behauptung folgt dann aus (i)\\
	\hspace*{13cm}$\square$
\end{enumerate}

% % % 8.9
\subsection{Bemerkung zu 8.8 (ii)}
Es gib auch diagonalisierbare Matrizen, die nicht n verschiedene Eigenwerte haben!\\
z.B. $E_n$ ist bereits in Diagonalform\\
$E_n=\begin{pmatrix}1 & \dots & 0\\ 0 & \ldots & 0\\ 0 & \dots & 1\end{pmatrix}=\underbrace{\begin{pmatrix}1 & \dots & 0\\ 0 & \ddots & 0\\ 0 & \dots & 1\end{pmatrix}}_{S}\underbrace{\begin{pmatrix}1 & \dots & 0\\ 0 & \ddots & 0\\ 0 & \dots & 1\end{pmatrix}}_{D}\underbrace{\begin{pmatrix}1 & \dots & 0\\ 0 & \ddots & 0\\ 0 & \dots & 1\end{pmatrix}}_{S^{-1}}$\\
aber alle n Ew sind 1 (mit lin. Abb. ausgedrückt: $\mathrm{id}_v$ ist diagonalisierbar, $A_{\mathrm{id}_v}^B$ hat n gleiche EW) %-mar
% 9.1 - 9.6
 % 9.1 - 9.6
 % % % % % % % % % % % % % % % % % % % % % % % % % % % % %
 

% % %
% 9 %
% % %

% % % 9.1
\section{Norm- und Skalarprodukt}
In diesem Kapitel betrachten wir nur $\mathbb{R}$-VR

% % % 9.2
\subsection{Definition: Norm}
Für $v=\begin{pmatrix}v_1 \\ \vdots \\ v_n\end{pmatrix}\in\mathbb{R}^n$ heißt $||v|| := (\sum_{i=1}^n v_i^2)^{\frac{1}{2}}$ die \textbf{Norm} oder \textbf{Länge}

% % % 9.3
\subsection{Eigenschaften}
\begin{enumerate}
	\item
	$||v|| \ge 0 \ \ \forall v\in\mathbb{R}^n$\\
	$||v|| = 0 \Leftrightarrow v=\mathcal{O}$
	
	\item
	$||\lambda v||=|\lambda|\cdot ||v|| \ \forall \lambda \in \mathbb{R}, \forall v\in\mathbb{R}^n$
	
	\item
	$||v+w|| \le ||v||+||w|| \ \forall v,w \in \mathbb{R}^n$
\end{enumerate}

% % % 9.4
\subsection{Definition: Skalarprodukt}
Sind $v,w \in \mathbb{R}^3$ Vektoren, die einen Winkel $\alpha$ einschließen, so heißt \[(v|w) := ||v||\cdot ||w||\cdot \cos \alpha\] das \textbf{Skalarprodukt} von v mit w.\\
anschaulich: $(v|w)=$ Flächeninhalt des von v und w erzeugten Projektionsrechtecks.

% % % 9.5
\subsection{Eigenschaften des Skalarprodukts}
seien $u,v,w \in \mathbb{R}^3, \lambda\in \mathbb{R}$
\begin{enumerate}
	\item $(v|w)\in R$ (d.h. ist Skalar, daher der Name)
	\item $(v|w)=(w|v)$ (denn: $(v|w)=||v||\cdot ||w||\cdot \cos \alpha = ||w||\cdot ||v||\cdot \cos\alpha = (w|v)$)
	\item $(\lambda\cdot v|w)=(v|\lambda\cdot w) = \lambda\cdot (v|w)$\\
	(denn $\lambda=0 \surd$\\
	$\lambda >0: \ (\lambda v|w)=||\lambda\cdot v||\cdot ||w||\cdot \cos \alpha=\lambda\cdot ||v||\cdot ||w||\cos\alpha = \lambda(v|w)$\\
	$\lambda <0:$ Winkel zw. $\lambda v$ und $w$ ist $\pi -\alpha \ \Rightarrow (\lambda v|w)=||\lambda\cdot v||\cdot ||w||\cdot \cos(\pi-\alpha) = -\lambda\cdot ||v||\cdot ||w||\cdot (-\cos \alpha) = \lambda\cdot (v|w)$)
	\item $(u+v|w)=(u|w)+(v|w)$ (z.B. grafisch klarmachen)\\
	wegen (ii) gilt (iii)\&(iv) auch im 2. Argument
	
	\item $(v|v)=||v||^2$ (denn: $\alpha =0: ||v||\cdot ||v||\cdot 1$)
\end{enumerate}
zur Berechnung:\\
$e_1,e_2,e_3$ kanon. Basisvektoren in $\mathbb{R}^3$\\
$(e_i|e_i)=1, \ (e_i|e_j)=1 \forall i\neq j$ (denn $\alpha=\frac{\pi}{2}$, Vektoren stehen senkrecht zueinander)\\
$v=\begin{pmatrix}v_1 \\ v_2 \\ v_3\end{pmatrix}, \ w=\begin{pmatrix}w_1 \\ w_2 \\ w_3\end{pmatrix}\in \mathbb{R}^3$\\
$\Rightarrow (v|w)=(v_1e_1+v_2e_2+v_3\cdot e_3 | w_1e_1+w_2e_2+w_3e_3) \stackrel{(ii),(iii)}{=} v_1w_1(e_1|e_1)+v_1w_2(e_1|e_2)+v_1w_3(e_1|e_3)+\dots = v_1w_1+v_2w_2+v_3w_3$\\
allgemein:

% % % 9.5
\subsection{Definition: Standardskalarprodukt, euklidischer Vektorraum, euklidische Norm \& Abstand}
\begin{enumerate}
	\item
	für $v=\begin{pmatrix}v_1 \\ \vdots \\ v_n\end{pmatrix}, \ w=\begin{pmatrix}w_1 \\ \vdots \\ w_n\end{pmatrix}\in \mathbb{R}^3$\\
	heißt \[(v|w):=\sum_{j=1}^n v_jw_j = v^Tw\] das \textbf{Standardskalarprodukt von v mit w}
	
	\item
	für beliebigen $\mathbb{R}-VR \ V$:\\
	Eine Abb $(\cdot | \cdot ): V\times V\rightarrow\mathbb{R} \ \ \ (v,w)\rightarrow (v,w)$ heißt \textbf{Skalarprodukt} auf V, falls $(\cdot | \cdot )$ Eigenschaften aus 9.4 erfüllt.\\
	V heißt dann \textbf{euklidischer Vektorraum}.
	
	\item
	für $v,w\in V$, V eukl. VR, so heißt \[||v|| := +\sqrt{(v|v)}\] die \textbf{(euklidische) Norm} von v, \[d(v,w)=||v-w||\] der \textbf{(euklid) Abstand} von v und w
\end{enumerate}

% % % 9.6
\subsection{Beispiel}
\begin{enumerate}
	\item
	$v=\begin{pmatrix}-1 \\ 2 \\ 1\end{pmatrix}, w=\begin{pmatrix}2 \\ 2 \\ 4\end{pmatrix}\in\mathbb{R}^3$\\
	$(v|w)=v^Tw=-1\cdot 2+2\cdot 2+1\cdot 4=6$\\
	$||v||=+\sqrt{(v|v)}=\sqrt{(-1)^2+2^2+1^2}=+\sqrt{6}$\\
	$d(v,w)=||v-w||=||\begin{pmatrix}-1-2 \\ 2-2 \\ 1-4\end{pmatrix}||=\sqrt{(-3)^2+0^2+(-3)^2}=\sqrt{18}$\\
	Winkel zwischen v und w:\\
	$(v|w)=||v||\cdot ||w||\cdot \cos \alpha \Leftrightarrow \cos \frac{(v|w)}{||v||\cdot ||w||}=\frac{6}{\sqrt{6}\sqrt{24}}=\frac{1}{2}$\\
	$\Rightarrow \alpha=\frac{\pi}{3}$
\end{enumerate}

% 10.1 - 10.7
 % 10.1 - 10.7
 % % % % % % % % % % % % % % % % % % % % % % % % % % % % %
 
% % % %
% 10  %
% % % %


\section{Orthogonalsysteme}

% % % % % % % % %
% % % 10.1    % %
% % % % % % % % %
\subsection{Definition: orthogonal, Orthogonalsystem, Orthonormalsystem, Orthonormalbasis}
V euklid. VR
\begin{enumerate}
\item
$v,w \in V$ heißen \textbf{orthogonal} (senkrecht), $v \perp w$, falls $(v|w)=0$ gilt.\\
(d.h. $v=\mathcal{O}$ oder $w=\mathcal{O}$ oder winkel zw. v und w ist $\frac{\pi}{2}$) ($\mathcal{O}$ ist $\perp$ zu allen Vektoren)

\item
$M\subseteq V$ heißt \textbf{Orthogonalsystem} (OGS), falls $(v|w)=0 \ \forall v,w \in M, \ v\neq w$, gilt.\\
(gilt zusätzlich $||v||=1 \ \forall v\in M$, so heißt M \textbf{Orthonormalsystem} (ONS))

\item
Ist V endlich dimensional, so heißt M \textbf{Orthogonalbasis} (ONB von V, falls M ONS und Basis von V ist.
\end{enumerate}

% % % % % % % % %
% % % 10.2    % %
% % % % % % % % %
\subsection{Bemerkung}
Jedes ONS ist l.u.:\\
$v_1,\dots,v_k$ ONS, $\lambda_1v_1+\dots+\lambda_kv_k=\mathcal{O}$
dann ist\\
$0 = (v_1|\underbrace{\lambda_1v_2+\dots+\lambda_kv_k}_0) $\\
$= \lambda_1\underbrace{(v_1|v_1)}_0 + \lambda_2\underbrace{(v_1|v_2)}_0+\dots+\lambda_k(v_1|v_k)$\\
$ = \lambda_1$\\
$\Rightarrow \lambda_1=0$\\
analog für $\lambda_2,\dots,\lambda_k$, alle =0\\
$\Rightarrow v_1,\dots,v_k$ l.u.

\subsection*{}
Man kann zu jedem Unterraum eines euklidischschen VR eine ONB berechnen.\\
Geg.: $v_1\dots v_k\in V$\\
Ges.: $w_1,\dots, w_k\in V$ (ONS) mit $<v_1,\dots,v_k>=<w_1,\dots,w_k>$\\
\textbf{Idee:} starte mit 1. Vektor, $w_1=v_1$\\
Baue $w_2$ aus $w_1$ und $v_2$:\\
$w_2=v_2+\lambda w_1$ mit $\lambda$ so, dass $w_2 \perp w_1$.\\
$w_1,w_2$ bilden dann OGS, $\frac{w_1}{||w_1||}$, $\frac{w_2}{||w_2||}$ bilden dann ONS.\\
$w_1 \perp w_2 \Leftrightarrow (w_1|v_2+\lambda w_1)=0$\\
$\Leftrightarrow (w_1|v_2)+\lambda\underbrace{(w_1|w_1)}_{||w_1||^2}=0$\\
$\Leftrightarrow \lambda=\frac{-(w_1|v_2)}{||w_1||^2}$


% % % % % % % % %
% % % 10.3    % %
% % % % % % % % %
\subsection{Satz: Orthogonalisierungsverfahren von Gram-Schmidt}
geg.: $v_1,\dots,v_k\in V$\\
def.: $w_1,\dots,w_k\in V$ wie folgt:\\
$w_1=v_1$\\
$w_{r+1}:=v_{r+1}+\sum_{i=1}^r \lambda_i^{(r+1)}w_i$\\
mit $\lambda_i^{(r+1)} := \frac{-(w_i|v_{r+1}}{||w_i||^2}$ falls $w_i\neq 0$\\
und $y_1,y_2\in V$ als $y_r:=\frac{w_r}{||w_r||}$ (falls $w_r\neq \mathcal{O}$)

\textbf{Dann gilt:}\begin{enumerate}
\item
Bricht die Iteration nach k Schritten nicht ab (d.h. $w_i\neq 0$ für $i=1\dots k$), so bilden $w_1,\dots,w_k$ ein OGS und $y_1,\dots,y_k$ ein ONS mit $<v_1,\dots,v_k>=<w_1,\dots,w_k>=<y_1,\dots,y_k>$

\item
Bricht die Iteration nach r Schritten ab (d.h. $w_r=\mathcal{O}$), so gilt: $v_1,\dots,v_{r-1}$ sind l.u., $v_1,\dots,v_r$ l.a.
\end{enumerate}

% % % % % % % % %
% % % 10.4    % %
% % % % % % % % %
\subsection{Beispiel}
$v_1=\begin{pmatrix}1 \\ 1 \\ 0\end{pmatrix}, v_2=\begin{pmatrix}1 \\ 3 \\ 2\end{pmatrix} \in \mathbb{R}^3$

gesucht:
\begin{enumerate}
\item
ONB für die Ebene $<v_1,v_2>$
\item
Vektor $v_3$, der diese ONB zu einer ONB von $\mathbb{R}^3$ ergänzt.
\end{enumerate}

Lösung:
\begin{enumerate}
\item
$w_1=v_1=\begin{pmatrix}1 \\ 1 \\ 0\end{pmatrix}$\\
$r=1, w_{r+1}=w_2$\\
$w_2=v_2+\sum_{i=1}^1 \lambda_i^{(2)} = v_2+\lambda_1^{(2)}w_1$\\
mit $\lambda_1^{(2)} = \frac{-(w_1|v_2)}{||w_1||^2}$\\
$(w_1|v_2)=1\cdot 1+1\cdot 3+0\cdot 2=4$\\
$||w_1||^2=1^2+1^2=2$\\
$\Rightarrow w_2=\begin{pmatrix}1 \\ 3 \\ 2\end{pmatrix}-\frac{4}{2}\cdot \begin{pmatrix}1 \\ 1 \\ 0\end{pmatrix}=\begin{pmatrix}-1 \\ 1 \\ 2\end{pmatrix}$\\
$\Rightarrow$ OGB $\left\lbrace \begin{pmatrix}1 \\ 1 \\ 0\end{pmatrix},\begin{pmatrix}-1 \\ 1 \\ 2\end{pmatrix}\right\rbrace$\\
$\Rightarrow$ ONB $\left\lbrace \frac{1}{\sqrt{2}}\begin{pmatrix}1 \\ 1 \\ 0\end{pmatrix}, \frac{1}{\sqrt{6}}\begin{pmatrix}-1 \\ 1 \\ 2\end{pmatrix}\right\rbrace$

\item
Vektor, der $\lbrace w_1,w_2\rbrace$ zu Basis ergänzt, ist z.B.\\
$v_3=\begin{pmatrix}1 \\ 0 \\ 0\end{pmatrix}$\\
(denn z.B. so zeigen: $\mathrm{det}\begin{pmatrix}1 & -1 & 1 \\ 1 & 1 & 0 \\ 0 & 2 & 0\end{pmatrix}=1\cdot \mathrm{det}\begin{pmatrix}1 & 1 \\ 0 & 2\end{pmatrix}=1\cdot 2\neq 0 \Rightarrow w_1,w_2,w_3$ l.u.)

$w_2=v_3-\frac{(w_1|v_3)}{||w_1||^2}\cdot w_1-\frac{(w_2|v_3)}{||w_2||^2}\cdot w_2 = \begin{pmatrix}1 \\ 0 \\ 0\end{pmatrix}-\frac{1}{2}\cdot \begin{pmatrix}1 \\ 1 \\ 0\end{pmatrix}+\frac{1}{6}\cdot \begin{pmatrix}-1 \\ 1 \\ 2\end{pmatrix}=\begin{pmatrix}\frac{1}{3} \\ -\frac{1}{3} \\ \frac{1}{3}\end{pmatrix}$,\\
$||w_3||=\sqrt{\frac{1}{3}}$\\
$y_3=\sqrt{3}\begin{pmatrix}\frac{1}{3} \\ -\frac{1}{3} \\ \frac{1}{3}\end{pmatrix}$
\end{enumerate}

% % % % % % % % %
% % % 10.5    % %
% % % % % % % % %
\subsection{Definition: orthogonale Matrix}
Eine Matrix $A\in M_n(\mathbb{R})$ heißt orthogonal, falls ihre Spektralvektoren eine ONB des $\mathbb{R}^n$ bilden.

% % % % % % % % %
% % % 10.6    % %
% % % % % % % % %
\subsection{Beispiel}
$\mathbb{R}^2:$\\
$A=\begin{pmatrix}\cos \alpha & -\sin \alpha\\ \underbrace{\sin \alpha}_{s_1} & \underbrace{\cos \alpha}_{s_2}\end{pmatrix} \ \ (\alpha\in\mathbb{R})$ ist orthogonal\\
$(s_1|s_2)=\cos \alpha\cdot (-\sin \alpha) + \sin \alpha\cdot \cos\alpha = 0$\\
$||s_1|| = ||s_2||\sqrt{\cos^2 \alpha + \sin^2 \alpha}=1$

% ***DATE: 2015-01-20***

% % % % % % % % %
% % % 10.7    % %
% % % % % % % % %
\subsection{Satz: Eigenschaften von orthogonalen Matrizen}
Sei $A\in M_n(\mathbb{R})$ orthogonal
\begin{enumerate}
\item $A^TA=E_n$
\item A invertierbar, $A^{-1}=A^T$
\item $||Av||=||v||$ (zugehörige Abb. ist ''Längentreu``)
\item $|\mathrm{det}A|=1$
\end{enumerate}
\textbf{Beweis:}
\begin{enumerate}
\item $s_1,\dots,s_n$ Spalten von A\\
bilden ONB $\Rightarrow (s_i|s_j)=\left\lbrace\begin{array}{ll}
1 & \text{für } i=j\\
0 & \text{für } i\neq j
\end{array}\right.$\\
$\Rightarrow A^TA=E_n$

\item
folgt aus (i)

\item
$||Av||^2 = (Av|Av) = (Av)^T\cdot Av = v^TA^TAv \stackrel{(i)}{=} v^TE_nv = v^Tv = (v|v) = ||v||^2$

\item
$1=\mathrm{det}E_n \stackrel{(i)}{=} \mathrm{det}(A^TA) = \mathrm{det}(A^T)\cdot \mathrm{det}(A) = \mathrm{det}(A)\cdot \mathrm{det}(A) = (\mathrm{det}(A))^2$
\end{enumerate}
% 11.1 - 11.27
 % 11.1 - 11.27
 % % % % % % % % % % % % % % % % % % % % % % % % % % % % %
 
% % % %
% 11  %
% % % %
\section{Mehrdimensionale Analysis}
siehe Blatt im Moodle


% % % % % % % % %
% % % 11.13   % %
% % % % % % % % %
\setcounter{subsection}{12}
\subsection{Beispiel}
\begin{enumerate}
	\item
	$f:\mathbb{R}^2\rightarrow\mathbb{R}$\\
	$f(x,y)=e^x+y^2$\\
	$\underbrace{f'(x,y}_{\text{Jacobimatrix}}=\begin{pmatrix}\frac{\partial f}{\partial x}(x,y) & \frac{\partial f}{\partial y}(x,y)\end{pmatrix}= \begin{pmatrix}e^x & 2y\end{pmatrix}$\\
	$f'(0,0)=\begin{pmatrix}1 & 0\end{pmatrix}$\\
	$\nabla f(x,y)=\begin{pmatrix}e^x \\ 2y\end{pmatrix}$
	
	\item
	$f:\mathbb{R}^3\rightarrow \mathbb{R}^2, \ \ f(x,y,z)=\begin{pmatrix}x+y \\ x*y*z\end{pmatrix}$\\
	$f'(x,y,z)=\begin{pmatrix}1 & 1 & 0 \\ yz & xz & xy\end{pmatrix}$\\
	$f'(2,0,1)=\begin{pmatrix}1 & 1 & 0 \\ 0 &  2 & 0 \end{pmatrix}$
\end{enumerate}

\setcounter{subsection}{21}
% % % % % % % % %
% % % 11.22   % %
% % % % % % % % %
\subsection{Definition: (total)differenzierbar, affin-linear}
$D\subseteq \mathbb{R}^n$ offen, $a\in D$\\
$f:D\rightarrow \mathbb{R}^m$ heißt \textbf{(total)differenzierbar in a} wenn f in a partiell differenzierbar ist und geschrieben werden kann als $f(x)=f(a)+f'(a)(x-a)+\epsilon(x)$ mit $\epsilon: D\rightarrow \mathbb{R}^m$ mit $\lim\limits_{x\rightarrow a} \frac{||\epsilon(x)||}{||x-a||}=0$ (d.h. $\epsilon$ wird klein nahe a)\\
($n=m=1$, erhalte Definition der Differenzierbarkeit aus Mathe II)\\
Anschaulich:\\
f kann in der nähe von a durch die \textbf{affin-lineare} Abbildung $x\mapsto \underbrace{f(a)}_{\text{konst.}}+\underbrace{\underbrace{f'(a)}_{\text{Matrix}}(x-a)}_{\text{linear}}$ beschrieben werden

% % % % % % % % %
% % % 11.23   % %
% % % % % % % % %
\subsection{Definition: Richtungsableitung}
$D\subseteq \mathbb{R}^n$ offen, $f:D\rightarrow \mathbb{R}, \ a\in D$\\
$v\in \mathbb{R}^n$ mit $||v||=1$\\
f heißt \textbf{in a differenzierbar in Richtung v} wenn $\lim\limits_{h\rightarrow 0} \frac{f(a+hv)-f(a)}{h}$ ex.\\
Der Grenzwert heißt dann \textbf{Richtungsabtleitung} von f in Richtung v im Punkt a.\\
Bez.: $\frac{\partial f}{\partial v}(a)$

% % % % % % % % %
% % % 11.24   % %
% % % % % % % % %
\subsection{Bemerkung}
$\frac{\partial f}{\partial x_j}$ ist die Richtungsableitung von f in Richtung $e_j=\begin{pmatrix}0 \\ \vdots \\ 0 \\ 1 \\ 0 \\ \vdots \\ 0\end{pmatrix}$ (1 an Stelle j)

% % % % % % % % %
% % % 11.25   % %
% % % % % % % % %
\subsection{Satz}
sei $f: D\subseteq \mathbb{R}^n\rightarrow \mathbb{R}$ (total)differenzeirbar in $a\in D$\\
Dann ex. in a alle Richtungsableitungen und für alle $v\in\mathbb{R}^n$ mit $||v||=1$ gilt:
\[\frac{\partial f}{\partial v}(a)=f'(a)*v\]
Die Richtungsableitug stellt den Anstieg von f an der Stelle a in Richtung v dar.

\textbf{Beweis:}\\
f differenzierbar, d.h. $f(x)=f(a)+f'(a)*(x-a)+\epsilon(x)$\\
$\stackrel{x=a+h*v}{\Rightarrow} f(a+h*v)=f(a)+f*(a)*(a+h*v-a)+\epsilon(a+h*v)$\\
$\stackrel{h\neq 0}{\Rightarrow} \frac{f(a+hv-f(a)}{h}=\frac{f'(a)*(hv)}{h} + \frac{\epsilon (a+hv)}{h}$\\
$\frac{\partial f}{\partial v}=\lim\limits_{h\rightarrow 0}\frac{f(a+hv)-f(a)}{h}=f'(a)*v$\\
\hspace*{13cm}$\square$

% % % % % % % % %
% % % 11.26   % %
% % % % % % % % %
\subsection{Beispiel}
\begin{enumerate}
	\item
	Anstieg von $f(x,y)=x^2+y^2$ im Punkt $a=\begin{pmatrix}1 \\ 1 \end{pmatrix}$ in Richtung $v=\begin{pmatrix}\sin \alpha \\ \cos \alpha\end{pmatrix} \ \ (||v||=1)$\\
	$\frac{\partial f}{\partial v}(1,1)=f'(1,1)*\begin{pmatrix}\sin \alpha \\ \cos \alpha\end{pmatrix}=\begin{pmatrix}2 & 2\end{pmatrix}*\begin{pmatrix}\sin \alpha \\ \cos \alpha	\end{pmatrix}=2*\sin \alpha+ 2*\cos\alpha$
	
	\item
	$f(x,y)=2x^2+y^2, \ \ a=\begin{pmatrix}\frac{1}{2} \\ \frac{1}{2} \\ \frac{3}{4}\end{pmatrix}$ Punkt auf ''Gebirge''\\
	\textbf{gesucht:} Richtung $v=\begin{pmatrix}v_1 \\ v_2\end{pmatrix}$, in der die Tangente an den Graph von f die Steigung $\frac{3}{\sqrt{2}}$ hat.\\
	$\frac{\partial f}{\partial v}\left(\frac{1}{2},\frac{1}{2}\right)=f'\left(\frac{1}{2},\frac{1}{2}\right) = \begin{pmatrix}2 & 1\end{pmatrix}\begin{pmatrix}v_1 \\ v_2\end{pmatrix}=2v_1+v_2 \stackrel{!}{=} \frac{3}{\sqrt{2}}$ und $||v||=1$, d.h. $v_1^2+v_2^2=1$\\
	Gleichungssystem lösen...\\
	ergibt $v=\begin{pmatrix}\frac{1}{\sqrt{2}} \\ \frac{1}{\sqrt{2}}\end{pmatrix}$ und $v=\begin{pmatrix}\frac{7}{5\sqrt{2}} \\ \frac{1}{5 \sqrt{2}}\end{pmatrix}$
\end{enumerate}

% % % % % % % % %
% % % 11.27   % %
% % % % % % % % %
\subsection{Bemerkung}
Es gilt: $\frac{\partial f}{\partial v}(a)=f'(a)*v=(\nabla f)^T*v = ||\nabla f(a)||*||v||*\cos \alpha$, $\alpha$: Winkel zw. $\nabla f$ und v\\
$\Rightarrow$ Richtungsableitung: $\frac{\partial f}{\partial v}(a)$ ist am größten, wenn $\cos \alpha = 1$, also $\alpha=0$  ist, d.h. wenn der Richtungsvektor v in Richtung des Gradienten zeigt.\\
Der Gradient zeigt also immer in die Richtung des steilsten Anstiegs der Funktion.

% 12.1 - 12.
 % 12.1 - 12.7
 % % % % % % % % % % % % % % % % % % % % % % % % % % % % %
 
% % % %
% 12  %
% % % %
\vspace*{0.5cm}
\hspace*{1cm}
\emph{Jetzt wieder 1-dimensionale Analysis:}
\section{Taylorpolynome und Taylorreihe}

% % % % % % % % %
% % % 12.1   % %
% % % % % % % % %
\subsection{Definition}
$I\subseteq \mathbb{R}$ Intervall, $x_0\in I, \ f:I\rightarrow\mathbb{R}$
\begin{enumerate}
	\item
	$f^{(0)}:=f$\\
	$f^{(1)}=f'$, falls f diffbar auf I\\
	$\vdots$\\
	$f^{(n)}=(f^{(n-1)})'$, falls $f^{(n-1)}$ diffbar auf I
	
	(f \textbf{n-mal differenzierbar}, $f^{(n)}$ \textbf{n-te Ableitung}
	
	\item
	f heißt \textbf{unendlich oft differenzierbar}, falls f n-mal diffbar $\forall n\in\mathbb{N}$.\\
	(Bez. auch $f^{(1)}=f',f^{(2)}=f'',\dots)$
\end{enumerate}

% % % % % % % % %
% % % 12.2    % %
% % % % % % % % %
\subsection{Beispiel}
\begin{enumerate}
	\item
	$f(x)=x^2 \ \ \infty$ oft diffbar\\
	$f'(x)=2x, \ f''(x)=2, \ f^{(n)}=0 \forall n\ge 3$
	
	\item
	$f(x)=e^x$\\
	$f^{(n)}(x)=e^x \forall n\in\mathbb{N}_0$
	
	\item
	$f(x)=\left\lbrace\begin{array}{ll}
	\frac{1}{2}x^2 & x\ge n\\
	-\frac{1}{2}x^2 & x< 0
	\end{array}\right.$\\
	$f'(x)=\left\lbrace\begin{array}{ll}
	x & x\ge 0\\
	-x & x<0
	\end{array}\right. = |x|$, nicht diffbar in $x=0$
	
	\item
	$f:\mathbb{R}^+\rightarrow\mathbb{R}^+, \ \ f(x)=x^\alpha \ \ (\alpha\in\mathbb{R})$\\
	$f'(x)=\alpha\cdot x^{\alpha-1}$\\
	$f^{(n)}=\alpha\cdot (\alpha-1)\cdot \dots\cdot (\alpha-n+1)\cdot x^{\alpha-n} = n!\underbrace{\begin{pmatrix}\alpha \\ n\end{pmatrix}}_{\text{binom.}}\cdot x^{\alpha-n} \ \ \forall n\in\mathbb{N}_0$
\end{enumerate}

% % % % % % % % %
% % % 12.3    % %
% % % % % % % % %
\subsection{Motivation}
Polynome sind besonders einfach zu handhaben.\\
Wir wollen komplizierte Funktonen möglichst gut mittels Polynome beschreiben / annähern.\\
Damit zwei Funktionen ''ähnlich`` sind, sollten nicht nur ihre Funktionswerte in einigen Punkten übereinstimmen, sondern möglichst auch ihre Ableitung in diesen Punkten.


\subsection*{}
\textbf{gegeben:} Funktion $f: \ I\rightarrow \mathbb{R}, \ x_o\in I$\\
\textbf{gesucht:} Polynom $T_n(x)$ vom Grad n, das f gut annähert, insbesondere an der Stelle $x_0$.\\
Wie muss $T_n$ aussehen?
\begin{enumerate}
	\item[für $n=0$:]
	(Grad 0, d.h. $T_0(x)$ ist Gerade)\\
	$T_0(x)=f(x_0)$ (dann wenigstens Übereinstimmung in $x_0$):\\
	$T_0(x_0)f(x_0)$
	
	\item[für $n=1$:]
	$T_1(x)=f(x_0)+f'(x_0)\cdot (x-x_0)$ Polynome vom Grad 1 $\surd$\\
	$T_1(x_0)=f(x_0)+f'(x_0)$ (Übereinstimmung in $x_o$) $\surd$\\
	$T_1'(x)=f'(x_0)$\\
	$\Rightarrow T_1'(x_0)=f'(x_0)$ (Übereinstimmung der 1. Ableitung in $x_0$)
	
	\item[für $n=2$:]
	$t_2(x)=f(x_0)+f'(x_0)\cdot (x-x_0)+\frac{1}{2}\cdot f''(x_0)\cdot (x-x_0)^2$ Polynom vom Grad 2 $\surd$\\
	$T_2(x_0)=f(x_0)+0+0$ (Übereinstimmung in $x_0$) $\surd$\\
	$T_2'(x)=f'(x_0)+2\cdot \frac{1}{2}\cdot f**(x_0)\cdot (x-x_o)^1=f'(x_0)+f''(x_0)(x-x_0)$\\
	$T_2''(x)=f''(x_0)$\\
	$T_2'(x_0)=f'(x_0)$\\
	$T_2''(x_0)=f''(x_0)$ $T-2$ und f stimmen in 1. und 2. Ableitung an der Stelle $x_0$ überein.
\end{enumerate}

% % % % % % % % %
% % % 12.4    % %
% % % % % % % % %
\subsection{Definition: Taylorpolynom}
$F:I\rightarrow \mathbb{R}$ n-mal differenzierbar auf I, $x_0\in I$\\
Dann heißt \[T_n(x):=\sum_{k=0}^n \frac{f^{(k)}(x_0)}{x-x_0}^k\] das \textbf{n-te Taylorplynome von f}, entwickelt um den Punkt $x_0\in I$.\\
oben für $n=0,1,2$ gesehen:\\
Für $T_n(x)$ gilt: $T_n(x_n)=f(x_n)$ und $T_n^{(k)}(x_0)=f^{(k)}(x_0)$ für $k=1\dots n$

$T_n(x)$ nähert also f an. Wie gut?

% % % % % % % % %
% % % 12.5    % %
% % % % % % % % %
\subsection{Satz: Formel von Taylor mit Lagrange-Restglied}
$f:I\rightarrow \mathbb{R}$ (n-1)-mal differenzierbar auf I, $x_0\in I$\\
Sei $R_n(x):=f(x)-T_n(x)$\\
der Fehler zwischen f un dem n-ten Taylorpolynom von f entwickelt um den Punkt $x_0$. (''Restglied``)\\
Dann gibt es zu jedem $x\in I$ eine Stelle $\xi$ zwischen $x_0$ und $x$, so dass \[R_n(x)=\frac{f^{(n+1)}(\xi)}{n+1}\cdot (x-x_0)^{n+1}\]
(Merkregel: (n+1)-ter Term von $t_{n+1}(x)$ mit $\xi$ statt $x_0$)\\
also ist f darstellbar durch das n-te Taylorpolynom mittels \[f(x)=\underbrace{\sum_{k=0}^n \frac{f^{(k)}(x_0)}{k!}(x-x_0)^k}_{\text{Polynom vom Grad n}} + \underbrace{\frac{f^{(n+1)}(\xi)}{(n+1)!}}_{R_n(x)}(x-x_0)^{n+1}\]
(Taylorentwicklung von f an der Stelle $X_0$)

\textbf{Beweis:}\\
Sei $g(x)=(x-x_)^{n+1}$\\
Es gilt $R_n^{(k)}(x_0)=0$ und $g^{(k)}(x_0)=0 \ \forall k=0\dots n$\\
$\frac{R(x)}{g(x)}=\frac{R(x)-R(x_0)}{g(x)-g(x_0)} \stackrel{*}{=} \frac{R'(\xi_1)}{g'(\xi_1)}$ für ein $\xi_1$ zwischen $x$ und $x_0$ (*: 2. Mittelwertsatz aus Mathe II)\\
$=\frac{R'(\xi_1)-R'(x_0)}{g'(\xi_1)-g'(x_0)} \stackrel{*}{=} \frac{R''(\xi_2)}{g''(\xi_2)}$ für ein $\xi_2$ zw. $\xi_1$ und $x_0$\\
$=\dots = \frac{R^{(n+1)}(\xi_{n+1})}{g^{(n+1)}(\xi_{n+1})} = \frac{f^{(n+1)}(\xi_{n+1})}{(n+1)!}$ für ein $\xi_{n+1}$ zwischen $\xi_n$ und $x_0$\\
setze $\xi=\xi_{n+1}$, Behauptung folgt

% % % % % % % % %
% % % 12.6    % %
% % % % % % % % %
\subsection{Bemerkung}
\begin{enumerate}
	\item
	Der Satz besagt:\\
	$f(x)$ kann bis auf $R_n(x)$ als Polynom n-ten Grades dargestellt werden.\\
	Je größer n, desto besser sollte diese Annäherung sein. Insbesondere ist interessant: gilt $R_n(x)\rightarrow 0$ für $n\rightarrow \infty$?
	
	\item
	Es gibt auch andere Darstellungen des Restglieds, z.B: mit Integral.
\end{enumerate}

% % % % % % % % %
% % % 12.7    % %
% % % % % % % % %
\subsection{Beispiel}
\begin{enumerate}
	\item
	$f(x)=e, \ x_0=0$\\
	$f^{(k)}=e^x \ \forall k\in \mathbb{N}_0$\\
	$f^{(k)}(x_0)=e^0=1$\\
	$\Rightarrow T_n(x)=\sum_{k=0}^n \frac{f^{(k)}(0)}{k!}(x-0)^k = \sum_{k=0}^n \frac{1}{k!}x^k$\\
	$\Rightarrow \underbrace{e^x}{f(x)}=\underbrace{\sum_{k=0}^n \frac{x^k}{k!}}_{T_n(x)}+\underbrace{\frac{e^\xi}{(n+1)!}\cdot x^{n+1}}_{R_n(x)}$ ($\xi$ zwischen 0 und x)\\
	$e^\xi$ ist beschränkt durch $e^0$ oder $e^x$, $\frac{x^{n+1}}{(n+1)!}\rightarrow 0$ für $n\rightarrow \infty$
	
	Also:\\
	%
	\[e^x=\sum_{k=0}^\infty \frac{x^k}{k!} \ \ \forall x\in\mathbb{R}\]
	% 
\end{enumerate}
\end{document}

