\documentclass[a4paper, 11pt, twosite, german, titlepage]{article}

% Packages 
\usepackage[german]{babel}
\usepackage[utf8]{inputenc}
\usepackage{textcomp}
\usepackage{lmodern}
\usepackage[T1]{fontenc}
\usepackage{amsmath}
\usepackage{amssymb}
\usepackage{amsthm}
\usepackage{wasysym} 
\usepackage{paralist}
\usepackage{varioref}
\usepackage{footnpag} % fussnoten seitenweise
\usepackage{fancyhdr}
\usepackage{newtxtext,newtxmath}
\usepackage[margin=3.5cm]{geometry}
\usepackage{polynom}
\usepackage[pdftex, pdftitle={Mathe III WiSe1415}, pdfauthor={getenv}]{hyperref}
\usepackage{pgf, tikz}


% % % % % % % % % % % % % %
\title{Mathematik III - Wintersemester 14/15}
%\author{names}
% % % % % % % % % % % % % % 

% % STYLE % %
\pagestyle{fancy}
\renewcommand{\sectionmark}[1]{%
\markboth {#1}{}}
\fancyhead[L]{Mathe III WiSe1415} 
\fancyhead[R]{\leftmark}	
% % % % % % % 
\renewcommand{\labelenumi}{\alph{enumi})}
% % % % % % %


% % % % % % % % % % % % % % %
% Selbstdefinierte Befehle  %
% bitte so eindeutig, wie   %
% moeglich halten!			%
% % % % % % % % % % % % % % %
\newcommand{\N}{\mathbb{N}}
\newcommand{\Z}{\mathbb{Z}}
\newcommand{\Q}{\mathbb{Q}}
\newcommand{\R}{\mathbb{R}}
\newcommand{\C}{\mathbb{C}}
% % Kuerzel % % % % % % % % % 
\newcommand{\gdw}{\Leftrightarrow}
% % Symbole % % % % % % % % %
\newcommand{\zerovec}{\mathcal{O}} %Nullvektor
\newcommand{\bigdot}{\,{\bullet}\,} %Verknuepfungsmal
%
\newcommand{\id}{\mathrm{ id }}
\newcommand{\Abb}{\mathrm{ Abb }}
\newcommand{\ggT}{\mathrm{ ggT }}
\renewcommand{\mod}{\;\mathrm{ mod }\;}
% % "Konstrukte" % % % % % %
\newcommand{\aset}[1]{\left\{#1\right\}} % Menge
\newcommand{\abs}[1]{\left|#1\right|} % Betrag
\newcommand{\without}[1]{\backslash \{#1\}} 
% % Vektoren / Matrizen % % %
\newcommand{\vct}[1]{\left(\begin{array}{c}
#1 \end{array}\right)} % Vektor - benutze \\ fuer neue Spalte
\newcommand{\aMatrix}[2]{\arraycolsep=1.8pt\def\arraystretch{0.8}\left(\begin{array}{#1}
#2 \end{array}\right)} % Matrix - erwartet als erstes Argument so viele c's, wie die Matrix Spalten hat, also etwa \aMatrix{cc}{1 & 2 \\ 3 & 4} fuer 2x2-Matrix
\newcommand{\sPer}[3]{\aMatrix{ccc}{1 & 2 & 3 \\ #1 & #2 & #3}}
% % Textstuff % % % % % % % %
\newcommand{\textunderset}[2]{\begin{tabular}[t]{@{}l@{}}
#1 \\ #2 
\end{tabular}}
 % Vorsicht, zweites Element unten!
 \newcommand{\mathunderset}[2]{\begin{array}[t]{@{}l@{}}
 #1 \\ #2 
 \end{array}}
  % Vorsicht, zweites Element unten!
% % % % % % % % % % % % % % %


% !! muss direkt vor \begin{document} bleiben !!
\usepackage[pdftex]{hyperref}
\hypersetup{pdftitle={Mathematik 2 SS14}, pdfnewwindow, colorlinks, linkcolor=black}

\begin{document}

\maketitle
\tableofcontents
\newpage

% Unterdrueckt Einrueckungen nach Absatz - gehoert direkt vor den Inhalt
\setlength{\parindent}{0pt}
\setlength{\parskip}{1ex plus 0.5ex minus 0.2ex}



% % % % % % % % % % % % %
% 		INHALT			%
% Hier geht es los :))	%
% % % % % % % % % % % % %

\section[Algebraische Strukturen mit einer Verknüpfung]{Algebraische Strukturen mit einer Verknüpfung \\ HALBGRUPPEN, MONOIDE, GRUPPEN}

\subsection{Definition}

Sei $X \neq \varnothing$ eine Menge.

Eine \emph{Verknüpfung} oder (abstrakte) Multiplikation auf $X$ ist eine Abbildung
\[
{\begin{array}{lc@{}l}
	\bigdot:& X \times X 	&\rightarrow X \\
			& (a, b)		&\mapsto	 a \bigdot b
		\end{array}}
\]
$\underset{(ab)}{a\bigdot b}$ heißt \emph{Produkt} von $a$ und $b$, muss aber mit der üblichen Multiplikation von Zahlen nichts zu tun haben.

Beschreibung bei endlichen Mengen oft durch Multiplikationstafeln.

\subsection{Beispiel}

\begin{enumerate}
	
	\item
	$X = \aset{a, b}$ 
	\quad
	$\begin{array}{c|cc}
		\bigdot	& a	& b \\
		\hline 
		a		& b	& b \\
		b		& a	& a
	\end{array}$
	
	$(a \bigdot a) \bigdot a = b \bigdot a = a \\
	 a \bigdot (a \bigdot a) = a \bigdot b = b$
	\qquad  $\rightarrow$ nicht assoziativ
	
	
	\item
	$X = \Z^- \; (= \aset{0, -1, -2, \dots})$
	
	Die normale Multiplikation ist auf $\Z^-$ keine Verknüpfung!
	\\(zum Beispiel ist $(-2) \cdot (-3) = 6 \notin  \Z^-$)
	\\ Aber auf $X = \N, X = \Z$ oder $X = \aset{1}, X = \aset{0, 1}$
	
\end{enumerate}

\subsection{Definition}

Sei $H \neq \varnothing$ eine Menge mit Verknüpfung.

$(H, \bigdot)$ heißt \emph{Halbgruppe}, falls gilt:
\[ \tag{Assoziativgesetz (AG)}
\forall a, b, c \in H \::\: (a \bigdot b) \bigdot c = a \bigdot (b \bigdot c)
\]

\subsection{Bemerkung}

AG sagt aus: bei endlichen Produkten ist die Klammerung irrelevant, z.B.

$(a \cdot b) \cdot (c \cdot d) = ((a \cdot b) \cdot c) \cdot d = (a \cdot (b \cdot c)) \cdot d$ \:(usw.)

Deshalb werden Klammern meistens weggelassen.

Die Reihenfolge der Elemente ist i.A. relevant!

\subsection{Beispiel} \label{bspHalbgruppe}

\begin{enumerate}
	
	\item
	$(\N, \bigdot), (\Z, \bigdot), (\Q, \bigdot), (\R, \bigdot)$ \footnote{$\bigdot$ normale Multiplikation}
	sind Halbgruppen.
	
	Ebenso $(\N, +), (\Z, +), (\Q, +), (\R, +)$ \footnote{$+$ normale Addition} 
	
	
	\item
	$(\Q \without{0}, :)$ \footnote{$:$ normale Division} ist \emph{keine} Halbgruppe, denn z.B.
	$\begin{array}{ccc}
	(12 : 6) : 2	&=& 1 \\
	12 : (6 : 2)	&=& 4
	\end{array}$
	
	\item
	vgl. Vorlesung Theoretische Informatik
	
	$A \neq \varnothing$ endliche Menge (''Alphabet'')
	
	$A^+ = \cup_{n \in N} A^n$ = Menge aller endlichen Wörter über $A$
	\\(z.B. $A=\aset{a, b}$, dann ist z.B. $\underset{aab}{\underbrace{(a, a, b)}} \in A^3$)
	
	Verknüpfung: Konkatenation (Hintereinanderschreiben)
	\\z.B. $aab \bigdot abab = aababab$
	
	$A^* = A^+ \cup \aset{\lambda}$ 
	\quad$\lambda$ (oder $\epsilon$) ist das leere Wort 
	
	Es gilt: 
	$\lambda \cdot w = w \cdot \lambda = w \;\forall w \in A^*$
	
	$(A^+, \bigdot), (A^*, \bigdot)$ \emph{Worthalbgruppe} über $A$
	
	
	\item
	$M \neq \varnothing$ Menge, Abb($M,M$): Menge aller Abbildungen $M \to M$
	mit $\circ$ (Komposition) ist Halbgruppe.
	
	\item (WICHTIG)
	
	$n \in \N,\; \Z_n = \aset{0, 1, \dots, n-1}$
	
	Verknüpfung:
	$\begin{array}{c@{\;:\;}r@{\;:=\;}r}
		\oplus	& a \oplus b	& (a+b) \mod{n} \\
		\odot	& a \odot b		& (a\cdot b)\mod{n}
	\end{array}$
	
	$(\Z_n, \oplus), (\Z_n, \odot)$ sind Halbgruppen.
	
\end{enumerate}










 %1.6 - 1.16
 % % % % % % % % % % % % % % % % % % % % % % % % % % % % %
 
 % % % 1.6
 \subsection[Definition: kommutative Halbgruppe]{Definition}
 
 Eine Halbgruppe $(H, \bigdot)$ heißt \emph{kommutativ}, falls gilt:
 %
 \[ \tag{Kommutativgesetz, KG}
 \forall a, b \in H\::\: a \cdot b = b \cdot a \]
 
 % % % 1.7
 \subsection{Beispiel}
 
 Beispiele \ref{bspHalbgruppe} a), e) sind kommutative Halbgruppen.
 \\ (hallo $\neq$ ollah, ab $\neq$ ba, Worthalbgruppe nicht kommutativ) 
 
 % % % 1.8
 \subsection[Definition: Unterhalbgruppe]{Definition}
 
 Sei $(H, \bigdot)$ Halbgruppe, $\varnothing \neq U \subseteq H$
 
 $U$ heißt \emph{Unterhalbgruppe} von $H$, falls $u \cdot v \in U \; \forall u, v \in U$ gilt.
 
 $(U, \odot)$ ist dann selbst Halbgruppe.
 
 % % % 1.9 
 \subsection{Beispiel}
 
 $(\Z, +)$ Halbgruppe
 
 $G =$ Menge aller gerade ganzen Zahlen $\subseteq \Z$
 \\$(G, +)$ ist Unterhalbgruppe von $(\Z, +)$
 
 $U =$ Menge aller ungerade Zahlen $\subseteq \Z$
 \\$(U, +)$ ist \underline{keine} Unterhalbgruppe!
 
 % % % 1.10
 \subsection[Lemma: Eins eindeutig]{Lemma} \label{nullelemEindeutig}
 
 \emph{Eindeutigkeit des neutralen Elements:}
 
 Sei $(H, \bigdot)$ Halbgruppe, $e_1, e_2 \in H$ mit 
 $^{(*)} e_1 \cdot x = x \cdot e_1 = x$ und $^{(**)} e_2 \cdot x = x \cdot e_2 = x \; \forall x \in H$

 Dann ist $e_1 = e_2$
 
 \begin{proof}
 $e_1 \stackrel{(**)}{=} e_1 \cdot e_2 \stackrel{(*)}{=} e_2$
 \end{proof} 
 
 % % % 1.11
 \subsection[Definition: Monoid]{Definition}
 
 Eine Halbgruppe $(H, \bigdot)$ heißt \emph{Monoid}, falls $e \in H$ existiert mit $e\cdot x = x \cdot e = x \; \forall x \in H$
 
 $e$ heißt \emph{neutrales Element} / Einselement / Eins in $H$.
 \\Schreibweise: $(H, \bigdot, e)$
 
 Für \textunderset{additive}{multiplikative} Verknüpfung oft \textunderset{0}{1} für $e$ (Nullelement)
 
 Nach \ref{nullelemEindeutig} ist das neutrale Element eindeutig!
 
 % % %  1.12
 \subsection{Beispiele}
 
 \begin{enumerate}

	\item
	$(\N, \bigdot)$ Monoid mit $e=1$
	\\$(\N, +)$ kein Monoid
	\\$(\N_0, +)$ Monoid mit $e=0$
	\\$(\Z, +), (\Q, +), (\R, +)$ Monoide mit $e=0$
	\\$(\Z, \bigdot), (\N_0, \bigdot), (\Q, \bigdot), (\R, \bigdot)$ Monoide mit $e=1$
	
	\item
	$(\Abb(M, M), \circ)$ Monoid, $e = \id$
	
	\item
	$(\Z_n, \oplus)$ Monoid, $e=0$
	\\ $(\Z_n, \odot)$ Monoid, $e=1$
	
	\item
	$(A^*, \bigdot)$ Monoid, $e= \lambda$ (hallo$\lambda$ = $\lambda$hallo = hallo)

 \end{enumerate}
 
 % % % 1.13
 \subsection[Definition: Untermonoid]{Definition}
 
 Sei $(M, \bigdot, e)$ Monoid. Eine Teilmenge $\varnothing \neq U \subseteq M$ heißt \emph{Untermonoid} von $M$, falls $U$ mit $\bigdot$ selbst ein Monoid mit neutralem Element $e$ ist (also $e \in U$)

% % %  1.14
 \subsection[Lemma: Inverses eindeutig]{Lemma} \label{neutElemEindeutig}
 
  \emph{Eindeutigkeit des inversen Elements:}
  
 Sei $(H, \bigdot, e)$ Monoid und es gebe zu jedem Element $h \in H$ Elemente $x, y \in H$ mit ${h \cdot x \stackrel{(*)}{=} e \stackrel{(**)}{=} y \cdot h}$.
 
 Dann ist $x = y$
 
 \begin{proof}
 	$y = y \cdot e \stackrel{(*)}{=} y \cdot (h \cdot x) \stackrel{(\text{AG})}{=} (y \cdot h) \cdot x \stackrel{(**)}{=} e \cdot x = x$
 \end{proof}
 
 % % % 1.15
 \subsection[Definition: Gruppe, Inverse, Ordnung]{Definition}
 
 {\renewcommand{\labelenumi}{(\roman{enumi})}
 \begin{enumerate}
 
 	\item
 	$(H, \bigdot, e)$ Monoid, $h \in H$
 	
 	Falls ein $x \in H$ existiert mit $hx = xh = e$, so nennt man $h$ \emph{invertierbar} und $x$ das \emph{Inverse} zu $h$, bez. $h^{-1}$
 	(bei additiven Verknüpfungen oft auch $-h$)
 	
 	Nach \ref{neutElemEindeutig} ist $h^{-1}$ eindeutig bestimmt!
 	
 	Es gilt: $e$ ist immer invertierbar, $e^{-1} = e$
 	
 	\item
 	Ein Monoid $(G, \bigdot, e)$ heißt \emph{Gruppe}, falls \underline{jedes} Element in $G$ invertierbar ist.
 	
 	\item
 	Für eine endliche Gruppe $G$ heißt die Anzahl der Elemente in $G$ die \emph{Ordnung} von $G$, $\abs{G}$
 	
 \end{enumerate}
 }
 
 % % % 1.16
 \subsection{Bemerkung}
 
 $(H, \bigdot, e)$ Monoid.
 
 Sei $G$ die Menge aller invertierbaren Elemente von $H$, dann ist $(G, \bigdot, e)$ eine Gruppe.
 
 Es gilt: $e$ invertierbar $(e^{-1} = e)$
 
 und falls $g$ invertierbar, dann ist auch $g^{-1}$ invertierbar: $(g^{-1})^{-1} = g$
 
 falls $g, h$ invertierbar, dann auch $g \cdot h$: \quad
 $(g \cdot h)^{-1} = h^{-1} \cdot g^{-1} $
 
 
 
 
 
 
 
 %1.17 - 1.19
 % % % % % % % % % % % % % % % % % % % % % % % % % % % % %

% % % 1.17
\subsection{Beispiele}

\begin{enumerate}
	
	\item
	$(\N_0, +, 0)$ ist keine Gruppe aber $(\Z, +, 0), (\Q, +, 0), (\R, +, 0)$ sind Gruppen.
	
	\item
	$(\Z, \bigdot, 1)$ ist keine Gruppe.
	\\ Die Menge der invertierbaren Elemente ist $\aset{1, -1}$, diese bilden eine Gruppe.
	
	\item
	$(\Q, \bigdot, 1)$ ist keine Gruppe, aber $(\Q \without{0}, \bigdot, 1), (\R \without{0}, \bigdot, 1)$ sind Gruppen.
	
	\item
	$A^*$ ist keine Gruppe, nur $\lambda$ ist invertierbar.
	
\end{enumerate}

% % % 1.18
\subsection{Beispiele}
\label{gruppenbeispiele}
\begin{enumerate}
	
	\item
	$(\Z_n, \oplus, 0)$ ist Gruppe (was ist das Inverse zu $x \in \Z_n$? Siehe PÜ1, A9)
	
	\item
	Sei $n \geq 2$. $(Z_n, \odot, 1)$ ist Monoid aber keine Gruppe. 
	
	Wann ist ein Element aus $Z_n$ invertierbar bezüglich $\odot$?
	
	$\begin{array}{lcl}
	z \in \Z_n \text{ invertierbar} 
		&\gdw&
		\exists x \in \Z_n : z \odot x = 1 \\
		&\gdw&
		 \exists x \in \Z: (z \cdot x) \mod{n} = 1 \\
		&\gdw&
		\exists x, q \in \Z: z \cdot x = q \cdot n + 1 \\
		&\gdw&
		\exists x, q \in \Z: z \cdot x + (-q \cdot n) = 1 \\
		&\stackrel{\text{Mathe I}}{\gdw}&
		\ggT(z, n) = 1 
	\end{array}$
	
	also sind nur zu $n$ teilerfremde Elemente invertierbar!
	
	(vgl. $(Z_6, 0, 1)$: $0, 2, 3, 4$ nicht invertierbar, $1, 5$ invertierbar)
	
	\underline{Bezeichnung:}
	\[Z_n^* = \aset{z \in \Z_n \;|\; \ggT(z, n) = 1}\]
	%
	ist Gruppe bezüglich $\odot$ (vgl. Bemerkung \ref{invbarkeitMonoid}) mit Ordnung $\abs{Z_n^*}=\varphi(n)$ (''phi von $n$'', Eulersche $\varphi$-Funktion) = Anzahl aller $z \in \N$, die teilerfremd zu $n$ sind und $1 \leq z \leq n$.
	
	$\varphi(3) = 2, \varphi(4)=2, \varphi(7) = 6$

	Wie berechnet man das Inverse von $z \in \Z_n^*$?
	
	Mathe I, Erweiterter Euklidischer Algorithmus (WHK, S. 80/81) liefert zu $z$ und $n$ $(\ggT(z, n) = 1)$ Zahlen $s, t \in \Z$ mit 
	
	$\begin{array}{cl}
	& z \cdot s + n \cdot t = 1 \\
	\Rightarrow & (z \cdot s) \mod n = 1 \\
	\Rightarrow & (z^{-1}) = s \mod n
	\end{array}$
	
	Beispiel:
	
	$n=8$: $(\Z_8, \odot)$, $z=5$ ist invertierbar, $\ggT(8, 5) = 1$
	
	EEA: $5 \cdot (-3) + 8 \cdot 2 = 1 \Rightarrow z^{-1} = -3 \mod 8 \Rightarrow z^{-1} = 5$


	\item
	$\Abb(M, M)$: invertierbare Elemente sind genau die \emph{bijektiven} Abbildungen auf $M, \mathrm{Bij}(M)$ (Mathe I)
	
	Speziell: $M = \aset{1, 2, \dots, n}$, dann heißt $\mathrm{Bij}(M)$ die symmetrische Gruppe von Grad $n$, $S_n$
	
	$\abs{S_n} = n!$, Elemente heißen Permutationen.
	
	\underline{Bsp:} $n=2$
	\[S_2= \aset{
	\underbrace{\aMatrix{cc}{
	1 & 2 \\ 
	1 & 2}}_{\id},
	\aMatrix{cc}{
	1 & 2 \\
	2 & 1}}\]
	%
	$n=3$
	%
	\[S_3 = \aset{
	\underbrace{\sPer{1}{2}{3}}_{\id},
	\sPer{1}{3}{2},
	\sPer{3}{2}{1},
	\sPer{2}{1}{3},
	\sPer{2}{3}{1},
	\sPer{3}{1}{2}
	}\]
	
	$\pi = \sPer{1}{3}{2}, \varrho = \sPer{3}{1}{2} \in S_3$
	
	$\pi \circ \varrho = \sPer{2}{1}{3},\; \varrho \circ \pi = \sPer{3}{2}{1}$ (nicht kommutativ!)
	
	$\pi^{-1} = \sPer{1}{3}{2} = \pi,\; \varrho^{-1}= \sPer{2}{3}{1}$
\end{enumerate}

% % % 1.19
\subsection[Satz: Gleichungen lösen in Gruppen]{Satz (Gleichungen lösen in Gruppen)} \label{gLösenGruppen}

	Sei $G$ Gruppe, $a, b \in G$
	
	{\renewcommand{\labelenumi}{(\roman{enumi})}
	\begin{enumerate}
		\item
		Es gibt genau ein $x \in G$ mit $ax = b$ (nämlich $x = a^{-1}b$)
		\item
		Es gibt genau ein $y \in G$ mit $ya = b$ (nämlich $y=ba^{-1}$)
		\item
		Ist $\underset{ya=yb}{ax=bx}$ für ein $\underset{y \in G}{x \in G}$, dann gilt $a=b$ (Kürzungsregel)
	\end{enumerate}
	
	\begin{proof}
	\begin{enumerate}
		\item
		\begin{itemize}
			\item
			$x=a^{-1}$ ist Lösung (prüfe $ax=b$):
			
			$a \cdot \underbrace{a^{-1}b}_{x} \stackrel{\text{AG}}{=}(a \cdot a^{-1}) \cdot b = e \cdot b = b$
			
			\item Es gibt genau eine Lösung:
			
			Es gelte $ax=b$
			\\ $\Rightarrow x = ex = (a^{-1}a)x\stackrel{\text{AG}}{=}a^{-1}(ax) = a^{-1}b$
		\end{itemize}
		
		\item
		analog
		
		\item
		Multipliziere von \textunderset{rechts}{links} mit $\underset{y^{-1}}{x^{-1}}$
	\end{enumerate}
	\end{proof}
	
	
	}

 
 \subsection{Beispiel}
 $
 \aMatrix{ccc}{
 	1 & 2 & 3 \\
 	2 & 1 & 3
 	}\circ
 	x =
 \aMatrix{ccc}{
 	1 & 2 & 3 \\
 	3 & 1 & 2
 	}
 	$
 - Was ist $x$?
 
 $ a \cdot x = b \Leftrightarrow x=a^{-1} \cdot b$
 
 $
 \lambda =
  \aMatrix{ccc}{
  	1 & 2 & 3 \\
  	2 & 1 & 3
  }^{-1}\circ
 \aMatrix{ccc}{
 	1 & 2 & 3 \\
 	3 & 1 & 2
 } = 
  \aMatrix{ccc}{
  	1 & 2 & 3 \\
  	2 & 1 & 3
  }\circ
  \aMatrix{ccc}{
  	1 & 2 & 3 \\
  	3 & 1 & 2
  } =
  \aMatrix{ccc}{
  	1 & 2 & 3 \\
  	3 & 2 & 1
  }
  $

\subsection{Definition}

$(G,\cdot)$ Gruppe, $\varnothing \neq U \subseteq G$ Teilmenge.

$U$ heißt \emph{Untergruppe} von $G$ ($U\leqslant G$), falls $u$ bzgl. $\cdot$ selbst eine Gruppe ist.

Insbesondere gilt dann:
$\forall u,v \in U$ ist $u \cdot v \in  U$.\\
$e$ von $G$ ist auch neutrales Element von $u$.\\
Inversen in $U$ sind die gleichen wie in $G$.

Angenommen $e$ neutrales Element in $G$, aber $f$ neutrales Element in $U$, $f^{-1}$ Inverses von $f$ in $G$.\\
Dann ist $f^{-1} \cdot f= f \cdot f^{-1} = e$ und $f\cdot f = f$.

$\Rightarrow f = e \cdot f = (f^{-1} \cdot f) \cdot f = f^{-1} \cdot (f \cdot f) = f^{-1} \cdot f = e$

\subsection{Beispiele}
\begin{enumerate}
	\item
	$(\Z, +) \leqslant (\Q, +) \leqslant (\R, +)$
	
	\item
	$(\{-1,1\}, \cdot) \leqslant (\Q \backslash \{0\}, \cdot) \leqslant (\R \backslash \{0\}, \cdot)$
	
	\item
	$(e, \cdot)$ ist Untergruppe jeder beliebigen Gruppe mit Verkn"upfung $\cdot$ und neutralem Element $e$.
	
	\item
	$\pi =
	\aMatrix{ccc}{
		1 & 2 & 3 \\
		2 & 1 & 3
		}
	\in S_3
	$,
	$\pi = \pi^{-1}, \pi^{-1} \circ \pi = \text{id} =
	\aMatrix{ccc}{
		1 & 2 & 3 \\
		1 & 2 & 3
		}
	$
	
	$\Rightarrow (\pi, \text{id})\leqslant S_3$
	
	\subsection{Satz und Definition}
	$G$ Gruppe, $U \leqslant G$
	\begin{enumerate}
		\item
		Durch $x \sim y \Leftrightarrow x \cdot y^{-1} \in U$\\
		TODO "Das muss unter die obere Zeile: bei additiver Verkn"upfung: $x + (-y) \in U (x-y \in U)$\\
		wird auf $G$ eine "Aquivalenzrelation definiert
		
		\underline{Beweis}:
		
		$\sim$ ist reflexiv: $x \sim x$ gilt $\forall x \in G$, denn $x \cdot x^{-1}= e \in U \; \checkmark$
		
		$\sim$ ist symmetrisch: $x \sim y \Rightarrow y \sim x$\\
		 Sei $x \sim y$, also $x \cdot y^{-1} \in U$ (zzg.: $y \sim x$, also $y \cdot x^{-1} \in U$) dann ist $y \cdot x^{-1} = (x \cdot y^{-1})^{-1} \in U$, da auch $x \sim y \Leftrightarrow x \cdot y^{-1} \in U$.
		 
		$\sim$ ist transitiv: $x \sim y, y \sim z \Rightarrow x \sim z$\\
		Sei $x \sim y$, also $x \cdot y^{-1} \in U$ und $y \sim z$, also $y \cdot z^{-1} \in U$ (zzg.: $x \sim z$, d.h. $x\cdot z^{-1} \in U$)
		
		$x \cdot z^{-1} = (\underbrace{x \cdot y^{-1}}_{\in U}) \cdot (\underbrace{y \cdot z^{-1}}_{\in U}) \in U$, also $x \sim z$.
		
		\item
		F"ur $x \in G$ ist $Ux = \{u \cdot x | u \in U\}$ die "Aquivalenzklasse von $x$ bzgl. $\sim$ und heißt Rechtsnebenklasse von $U$ in $G$.
		
		Also (Eigenschaften von "Aquivalenzklassen siehe Mathe I):
		\begin{enumerate}
			\item
			$Ux = Uy \Leftrightarrow x \sim y$, also $x \cdot y^{-1} \in U$
			\item
			$x,y \in G$, dann ist entweder $Ux = Uy$ oder $Ux \cap Uy = \varnothing $
		\end{enumerate}
		
		Beweis:
		\begin{enumerate}
			\item
			Seit $x \sim y \Rightarrow y \sim x \Rightarrow y \cdot x^{-1} \in U \Rightarrow y=y(x^{-1} \cdot x) = \underbrace{(y \cdot x^{-1})}_{\in U}x \in Ux$
			\item
			Sei $y \in Ux$, dann zeige: $x \sim y$ \\
			$y \in Ux \Rightarrow y = u \cdot x$ f"ur ein $u \in U\\
			\Rightarrow x \cdot y^{-1} = x \cdot (ux)^{-1} = x \cdot x^{-1} \cdot u^{-1} = u^{-1} \in U$\\
			Es wurde gezeigt, dass $x \sim y$ gilt.
		\end{enumerate}
	\end{enumerate}
\end{enumerate}

\subsection{Beispiel}

$G = (\Z, +), 3 \Z = \{\dots, -3,0,3,6,\dots\}\\
U = (3 \Z, +) \leqslant G$ ("UA, Blatt 2)\\
Inverses zu y in $(\Z,+)$ ist $-y$.\\
$x \sim y \Leftrightarrow \underbrace{x \cdot y ^{-1} \in U}_{\text{bzw.:}\, x-y \in U}$ 

$ x=0 : U+0 = \{u+0 | u \in U\}= \{\dots, -3,0,3,6,\dots\}\\
   x=1 : U+1 = \{u+1 | u \in U\} = \{\dots\}$
 %1.25 - 1.31
 % % % % % % % % % % % % % % % % % % % % % % % % % % % % %

\subsection[Lemma: Mächtigkeit von Untergruppen]{Lemma}
\label{maechtigkeituntergruppen}
$G$ Gruppe, $U$ endliche Untergruppe von $G$, $x \in G$

Dann ist $\abs{U} = \abs{Ux}$

\subsubsection*{Beweis}

$\begin{array}[b]{rrcl}
\Abb \; \varphi: 	& U &\to& Ux \\
				& u &\mapsto& ux
\end{array}$
\\ist surjektiv
und injektiv 
(falls $u_1x=u_2x$, dann ist $u_1 = u_2$ (Satz \ref{gLösenGruppen} (iii), Kürzungsregel))

Also ist $\varphi$ bijektiv, also $U, Ux$ gleich mächtig.

\subsection[Theorem: Satz von Lagrange]{Theorem (Satz von Lagrange)} \label{lagrange}

$G$ endliche Gruppe, $U \leqslant G$

Dann gilt $\abs{U}$ ist Teiler von $\abs{G}$ und $q = \frac{\abs{G}}{\abs{U}}$ ist die Anzahl der Rechtsnebenklassen von $U$ in $G$

\subsubsection*{Beweis}

Seien $Ux_1, \dots, Ux_q$ die $q$ verschiedenen Rechtsnebenklassen von $U$ in $G$

\[\begin{split}
\text{Mathe I \& \ref{rechtsnebenklassen}}\Rightarrow G = \bigcup_{i=1}^{q}Ux_i \text{ (disjunkte Vereinigung der Äquivalenzklassen)}
\\\Rightarrow \abs{G} = \sum_{i=1}^{q}\underbrace{\abs{Ux_i}}_{\abs{U}} \stackrel{\ref{maechtigkeituntergruppen}}{=} q \cdot \abs{U} \end{split}\]


\subsection[Definition: Potenzen]{Definition}

$(G, \bigdot, e)$ Gruppe, $a \in G$

Definiere
$\begin{array}[t]{lcll}
a^0	&:=& e \\
a^1 &:=& a \\
a^m &:=& a^{m-1} \cdot a & \text{für } m \in \N \\
a^m &:=& (a^{m})^{-1} 	& \text{für } m \in \Z^-
\end{array}$

(Potenzen von $a$)

Bei additiver Schreibweise:
$\begin{array}[t]{lcl}
0 \cdot a &=& e \\
1 \cdot a &=& a \\
m \cdot a &=& 
\begin{cases}
(m-1) \cdot a + a & \text{für } m \in \N \\
(-m) \cdot (-a) 	& \text{für } m \in \Z^-
\end{cases}
\end{array}$

\subsection[Satz: Potenzgesetze]{Satz}
\label{potenzgesetze}
$G, a$ wie oben

{\renewcommand{\labelenumi}{(\roman{enumi})}
\begin{enumerate}
	\item
	$(a^{-1})^m = (a^m)^{-1} = a^{-m} \quad \forall m \in \Z$
	
	\item
	$a^m \cdot a^n = a^{m+n} \quad \forall m, n \in \Z$
	
	\item
	$(a^m)^n=a^{m \cdot n} \quad \forall m, n \in \Z$
\end{enumerate}

\subsubsection*{Beweis}
\begin{enumerate}
	\item
	$m \in \N: (a^{-1})^m \cdot a^m = \underbrace{a^{-1} \cdot a^{-1}\cdot \dots \cdot a^{-1}}_{m \text{ mal}} \cdot \underbrace{a \cdot \dots \cdot a \cdot a}_{m \text{ mal}} = e$
	
	$\Rightarrow (a^{-1})^m = (a^m)^{-1}$ (Inverses von $a^m$)
	
	nach Definition ist $a^{-m} = (a^{-1})^m$
	\\ $\Rightarrow $ (i) gilt $\forall m \in \N$
	
	$m = 0: \; e = e = e \checkmark$
	
	$m \in \Z^-$: dann ist $-m \in \N$
	\\ Wende den bewiesenen Teil an auf $a^{-1}$ statt $a$ und $-m$ statt $m$, Behauptung folgt.
	
	\item[(ii), (iii)]
	per Induktion und mit (i) \qed
\end{enumerate}}

\subsection[Satz und Definition: Ordnung, zyklische Gruppe]{Satz und Definition}

$G$ endliche Gruppe, $g \in G$

{\renewcommand{\labelenumi}{(\roman{enumi})}
\begin{enumerate}
	\item
	Es existiert eine kleinste natürliche Zahl $n$ mit $g^n = e$, diese heißt die \emph{Ordnung} $\mathrm{o}(g)$ von $G$ 
	
	\item
	Die Menge $\aset{g^0 = e, g^1 = g, g^2, \dots, g^{n-1}}$ ist eine Untergruppe von $G$, die von $g$ erzeugte zyklische Gruppe $<g>$
	
	Es gilt $\mathrm{o}(g) = \abs{<g>} = n \text{ teilt } \abs{G}$
	
	\item
	$g^{\abs{G}} = e$
	
	Bemerkung: Eine endliche Gruppe heißt \emph{zyklisch}, falls sie von einem Element erzeugt werden kann.
\end{enumerate}

\subsubsection*{Beweis}

\begin{enumerate}
	\item
	$G$ endlich $\Rightarrow \exists i, j \in \N, i > j$ mit $g^i=g^j$ \quad(Schubfachschluss \emph{-Editor})
	
	Dann ist $g^{i-j} \stackrel{\ref{potenzgesetze} ii)}{=} g^i \cdot g^{-j} \stackrel{\ref{potenzgesetze}}{=}\underbrace{g^i}_{=g^j} \cdot (g^j)^{-1} = e$
	
	\item
	Das Produkt zweier Elemente aus $<g>$ liegt wieder in $<g>$
	
	Neutrales Element ist $g^0 = e$
	
	Inverses Element zu $g^i$ ist $(g^i)^{-1} = g^{n-i}$
	
	$\Rightarrow <g> \leqslant G$
	
	\item
	Satz von Lagrange (\ref{lagrange}): $n = \mathrm{o}(g)= \abs{<g>} \;\Big|\; \abs{G}$
	
	Also ist $\abs{G} = n \cdot k$ für ein $k \in \N$
	
	$g^{\abs{G}} = g^{n \cdot k} = (g^{n})^k = e^k = e$ 
\end{enumerate}} \qed


\subsection{Beispiel}

$(\Z_3 \without{0}, \odot, 1)$

\begin{itemize}

	\item[$g=1$:]
	$<1> = \aset{g^0 = 1^0 = 1},\; \mathrm{o}(1) = 1$
	
	\item[$g=2$:]
	$<2> = \aset{g^0 = 1, g^1 = 2},\; \mathrm{o}(2)=2$

\end{itemize}

$(\Z_5 \without{0}, \odot, 1)$

\begin{itemize}
	\item[$g=2$:]
	$<2> = \aset{2^0 = 1, 2^1 = 2, 2^2 = 4, 2^3 = 3}, \; \mathrm{o}(2)=4$
\end{itemize}

\subsection{Korollar}

{\renewcommand{\labelenumi}{(\roman{enumi})}
\begin{enumerate}
	\item
	\underline{Satz von Euler}
	
	Sei $n\in \N, a \in \Z, \ggT(a, n) = 1$	
	\\Dann ist \[a^{\varphi(n)} \equiv 1 (\mod n)\]
	
	\item
	\underline{Kleiner Satz von Fermat}
	
	Ist $p$ eine Primzahl, $a \in \Z, p \nmid a$, dann gilt
	\[a^{p-1} \equiv 1 (\mod p)\]
	
\end{enumerate}}










 
 \subsection{Beweis}
 \begin{enumerate}
 \item
 Wir k"onnen annehmen, dass $1 \leq a < n $ (denn $a^{\varphi(n)} \mod{n}= (a \mod{n})^{\varphi(n)})$\\
 wegen ggT$(a,n)=1$ ist $a \in \Z^*_n$, das ist eine endliche Gruppe.
 
 $\stackrel{\ref{zyklischeGruppe}(iii)}{\Rightarrow} a^{|\Z^*_n|}=1(=e) \hspace{1.5cm} a \odot a \odot \dots$\\
 $\Rightarrow a^{\varphi(n)} \equiv 1 (\mod{n}) \hspace{0.7cm} a \cdot a \cdot \dots $ 
 \item
 Folgt aus (i) $(n=p,\; \varphi (p) = -1)$

\end{enumerate}
\section{Algebraische Strukturen mit 2 Verkn"upfungen: Ringe und K"orper}

\subsection[Definition: Ring]{Definition} \label{ring}
Sei $R \neq \varnothing$ eine Menge mit zwei Verkn"upfungen $+$ und $\bigdot$.
{\renewcommand{\labelenumi}{(\roman{enumi})}
\begin{enumerate}
	\item
	Wir nennen $(R, +, \cdot)$ einen \emph{Ring}, falls gilt:
	%TODO 1), 2) etc
	{\renewcommand{\labelenumi}{\arabic{enumi})}\begin{enumerate}
		\item
		$(R,+)$ ist eine abelsche Gruppe (Eselsbr"ucke: KAIN)\\
		Das neutrale Element bezeichnen wir hier mit $0$, das zu $a \in \R$ Inverse mit $-a$ (schreibe auch $a-b$ f"ur $a+(-b)$.
		\item
		$(R,\cdot)$ ist eine Halbgruppe.
		\item
		Es gelten die Distributivgesetze:\\
		$a\cdot (b+c) = (a \cdot b) + (a \cdot c) = ab + ac\\
		(a+b) \cdot c - (a \cdot c) + (b \cdot c) = ac = bc$ \qquad $\forall a, b, c \in R$
	\end{enumerate}}
	\item
	Ein Ring $(R,+, \cdot)$ heißt \emph{kommutativ} falls $\cdot$ ebenfalls kommutativ ist, also falls ${\forall a,b \in \R: a \cdot b = b \cdot a}$
	\item
	Ein Ring $(R,+, \cdot)$ heißt \emph{Ring mit Eins}, falls $(R, \cdot)$ ein Monoid ist mit neutralen Element $1\neq 0$ \;($\forall a \in R: a \cdot 1 = 1 \cdot a = a$).
	\item
	Ist $(R, +, \cdot)$ Ring mit Eins, dann heißen die bez"uglich $\cdot$ invertierbaren Elemente \emph{Einheiten}. Das zu $a$ bez"ugliche $\cdot$
	invertierbare Element bezeichnen wir mit $a^{-1}$.\\ $R^* :=$ Menge der Einheiten in $R$.
\end{enumerate}}
\subsection{Beispiel}
\begin{enumerate}
	\item
	($\Z, +, \cdot$) ist kommutativer Ring mit Eins (1)\\
	$\Z^* = \{1,-1\}$,
	$(\Q, +, \cdot), (\R, +, \cdot)$ ebenso\\
	$\Q^*=\Q \backslash \{0\}, \R^* = \R \backslash \{0\}$.
	\item
	$(2\Z, +, \cdot)$ ist ein kommutativer Ring ohne Eins
	\item
	trivialer Ring $(\{0\},+, \cdot)$ ohne Eins
	\item
	$n \in \N, n \geq 2,  (\Z_n, \oplus, \odot)$ kommutativer Ring mit Eins
	\item
	$(\R^n, \underbrace{+\; ,\; \cdot}_{\text{Komponentenweise}})$; allgemein: $R_1, \dots , R_n$ Ringe, dann $R_1, \times \cdots \times R_n$ Ring.
	\item
	$M_n (\R)$ - Menge aller $n \times n$-Matrizen  "uber $\R$, mit Matrixaddition und -multiplikation ist Ring mit Eins (=$E_n$), nicht kommutativ f"ur $ n \geq 2$.
\end{enumerate}
\subsection[Satz: Rechnen mit Ringen]{Satz (Rechnen mit Ringen)}
Sei $(R, +, \cdot)$ ein Ring, $a,b,c \in R$. Dann gilt:
{\renewcommand{\labelenumi}{(\roman{enumi})}\begin{enumerate}
	\item
	$a \cdot 0 = 0 \cdot a = 0$
	\item 
	$(-a)\cdot b = a \cdot (-b) = -(a \cdot b)$
	\item
	$(-a) \cdot (-b) = a \cdot b$
\end{enumerate}

\subsubsection*{Beweis}

\begin{enumerate}
	\item
	$a \cdot 0 = a \cdot (0+0) \underset{\ref{ring}(3)}{=}a \cdot 0 + a \cdot 0$\\
	addiere $-(a \cdot 0)$ (Inverses von $a \cdot 0$) auf beiden Seiten,  erhalte $0=a \cdot 0$\\
	Analog $0 \cdot a = 0$
	\item
	$(-a)\cdot b + a \cdot b \underset{\ref{ring}(3)}{=} (-a+a) \cdot b = 0 \cdot b \overset{(i)}{=}0$\\
	also ist $(-a \cdot b)$ Inverses zu $a \cdot b$, also $=-(a \cdot b)$.\\
	Analog $a \cdot (-b) = -(a \cdot b)$
	\item
	$(-a) \cdot (-b) \underset{(ii)}{=} -(a \cdot (-b)) \underset{(ii)}{=}-(-(a \cdot b)) = a \cdot b   $
\end{enumerate} } \qed
 %2.5 - 2.11
 % % % % % % % % % % % % % % % % % % % % % % % % % % % % %



% % %  2.5
\subsection[Definition: Körper]{Definition}

Ein kommutativer Ring $(K, +, \cdot)$ heißt \emph{Körper}, wenn jedes Element $0 \neq x \in K$ eine Einheit ist, also wenn
%
\[K^* = K \without{0}\]

% % % 2.6
\subsection{Beispiele}

\begin{enumerate}
	\item
	$(\Q, +, \cdot), (\R, +, \cdot)$ sind Körper. $(\Z, +, \cdot)$ ist kein Körper.
	
	\item
	vgl. Beispiel \ref{gruppenbeispiele} b)
	\[\Z_n^* = \aset{z \in \Z_n \;|\; \ggT(z, n) = 1}\]
	ist Gruppe bezüglich $\odot$
	
	$\Rightarrow (\Z_n, \oplus, \odot)$ ist genau dann ein Körper, wenn $n$ eine Primzahl ist.
\end{enumerate}

% % % 2.7
\subsection[Satz: Rechnen im Körper, Nullteilerfreiheit]{Satz (Rechnen im Körper, Nullteilerfreiheit)}

Sei $(K, +, \cdot)$ ein Körper, $a, b \in K$

Dann gilt 
\[a \cdot b= 0 \gdw a = 0 \text{ oder } b = 0 \]
%
Gegenbeispiel: $(\Z_6, \oplus, \odot)$ ist kein Körper. 
Hier gilt $2 \odot 3 = 0$, aber weder $2=0$, noch $3=0$

\subsubsection*{Beweis}
\begin{itemize}

	\item[''$\Leftarrow$'':]
	klar: $0 \cdot b = 0$ oder $a \cdot 0 = 0$ \;\;(Satz \ref*{rechnenmitringen} (i), Rechenregeln für Ringe)
	
	\item[''$\Rightarrow$'':]
	Sei $a \cdot b = 0$.
	Angenommen $a \neq 0$ (d.h. $a$ hat Inverses)
	
	Dann ist 
	$\begin{array}[t]{l@{}c@{}l}
	b	&=& 1 \cdot b = (a^{-1} \cdot a) \cdot b \\
		&=& a^{-1} \cdot (a \cdot b) \\
		&=& a^{-1} \cdot 0 \\
		&\stackrel{\ref{rechnenmitringen}(i)}{=}& 0
	\end{array}$
	
\end{itemize} \qed

% % %  2.8
\subsection[Definition: Homomorphismus, Isomorphismus]{Definition}

Seien $(R, +, \cdot)$ und $(\tilde{R}, \boxplus, \boxdot)$ Ringe.

{\renewcommand{\labelenumi}{(\roman{enumi})}
\begin{enumerate}
	\item
	$\varphi: R \to \tilde{R}$ heißt (Ring-)\emph{Homomorphismus}, falls gilt:
	\[\varphi(\underbrace{x+y}_{\in R}) = \underbrace{\varphi(x)}_{\in \tilde{R}} \boxplus \underbrace{\varphi(y)}_{\in \tilde{R}}
	\text{\;\; und \;\;}
	\varphi(x \cdot y) = \varphi(x) \boxdot \varphi(y) \quad \forall x, y \in R\]
	
\end{enumerate}}

% % % 2.9
\subsection{Beispiel}

$\begin{array}[b]{r@{\;}c@{\;}l}
	\varphi(\Z, +, \cdot) &\to&  (\Z_n, \oplus, \odot) \\
	x	&\mapsto& x \mod n 
\end{array}$
ist Ringhomomorphismus (kein Isomorphismus), da $\varphi$ nicht injektiv ist, z.B. $n=5: \varphi(1)= \varphi(6) = \varphi(11) \dots$

% % % 2.10
\subsection[Satz: Chinesischer Restsatz]{Satz (Chinesischer Restsatz)} \label{chin.restsatz}

Seien $m_1, \dots, m_n \in \N$ paarweise teilerfremd,
$M := m_1 \cdot \ldots \cdot m_n, \;\; a_1, \dots, a_n \in \Z$

Dann existiert ein $x$, $0 \leq x < M$ mit 

$\begin{array}{lcll}
x &\equiv& a_1	& (\mod m_1) \\
x &\equiv& a_2	& (\mod m_2) \\
\dots \\
x &\equiv& a_n	& (\mod m_n)
\end{array}$

\subsubsection*{Beweis}
Für jedes $i \in \aset{1, \dots, n}$ sind die Zahlen $m_i$ und $M_i := \frac{M}{m_i}$ teilerfremd.

$\Rightarrow$ EEA liefert $s_i$ und $t_i \in \Z$ mit $t_i \cdot m_i + s_i \cdot M_i = 1$

Setze $e_i := s_i \cdot M_i$, dann gilt:
\[\begin{array}{ll}
	e_i \equiv 1	& (\mod m_i) \\
	e_i \equiv 0 	& (\mod m_j) \;\; (j \neq i)
\end{array}\]
%
Die Zahl $x := \sum_{i=1}^{n}a_ie_i (\mod M)$ ist dann die Lösung der simultanen Kongruenz. \qed

% % % 2.11
\subsection{Beispiel}

\begin{enumerate}

	\item
	Finde $0 \leq x < 60$ mit 
	$x \equiv \begin{cases}
	2 & (\mod 3) \\
	3 & (\mod 4) \\
	2 & (\mod 5)
	\end{cases}$
	
	$M = 3 \cdot 4 \cdot 5 = 60$
	
	\[\begin{array}{lr@{\quad\Rightarrow\;}l}
	M_1 = \frac{60}{3} = 20 & 7 \cdot 3 + (-1) \cdot 20 = 1	& e_1 = -20 \\
	M_2 = \frac{60}{4} = 15 & 4 \cdot 4 + (-1) \cdot 15 = 1	& e_2 = -15 \\
	M_3 = \frac{60}{5} = 12	& 5 \cdot 5 + (-2) \cdot 12 = 1	& e_3 = -24
	\end{array}\]
	
	$x = (2 \cdot (-20) + 3 \cdot (-15) + 2 \cdot (-24)) \mod 60 = 47$
	
	\item
	Was ist $2^{1000} \mod \underbrace{1155}_{3 \cdot 5 \cdot 7 \cdot 11}$
	
	\begin{enumerate}
		\item
		Berechne $2^{1000} \mod 3, 5, 7, 11$
		
		$\begin{array}{lcrcl}
		2^{1000} \mod 3 &=& (-1)^{1000} \mod 3 &=& 1 \\
		2^{1000} \mod 5 &=& 4^{500} \mod 5 = (-1)^{500} \mod 5 &=& 1 \\
		2^{1000} \mod 7 &=& 2^{3 \cdot 333+1} \mod 7 = (8^{333} \cdot 2) \mod 7 = (1 \cdot 2) \mod 7 &=& 2 \\
		2^{1000} \mod 11 &=& 2^{5 \cdot 200} \mod 11 = 32^{200} \mod 11 = (-1)^{200} \mod 11 &=& 1
		\end{array}$
		
		\item
		Suche $0 \leq x < 1155$ mit
		$x \equiv \begin{cases}
			1 & (\mod 3) \\
			1 & (\mod 5) \\
			2 & (\mod 7) \\
			1 & (\mod 11)
			\end{cases}$ 
			
		Der chinesische Restsatz liefert $x = 331$
	\end{enumerate}
	
\end{enumerate}

























 %2.12 - 2.19
 % % % % % % % % % % % % % % % % % % % % % % % % % % % % %

\subsection{Bemerkung} \label{chinsatzeindeutig}
Man kann auch zeigen, dass die L"osung $x$ aus Satz \ref{chin.restsatz} eindeutig ist:

Durch 
$\begin{array}[t]{lr@{\;}c@{\;}l}
\psi:&	\Z_M &\to&  Z_{m_1} \times \Z_{m_2} \times \dots \times \Z_{m_n} \\
&	x	&\mapsto& (x \mod{m_1}, \dots, x \mod{m_n})
\end{array}$

wird ein Ringisomophismus definiert:

$\psi$ ist surjektiv (zu jedem n-Tupel aus $\Z_{m_1}\times \dots \times \Z_{m_n}$ gibt es eine L"osung $x$, siehe Restsatz) und es gilt:

$\underbrace{|\Z_M|}_{M}=\underbrace{|\Z_{m_1}\times \dots \times \Z_{m_n}|}_{m_1\cdot \dots \cdot m_n = M}$

also ist $\psi$ bijektiv, also auch injektiv, also ist L"osung $x$ eindeutig.

\subsection[Korollar: Phi-Funktion berechnen]{Korollar}
$M=m_1\cdot \dots \cdot m_n$, $m_i$ paarweise teilerfremd.\\
Dann ist $\varphi(M)=\varphi(m_1)\cdot \dots \varphi(m_n),$ insbesondere:

$n=p^{a_1}_1 \cdot \dots \cdot p^{a_k}_k$ ($p_i$ Primzahlen, $a_1>0, p_i \neq p_j$ f"ur $i\neq j$)

\subsubsection*{Beweis}

Nach \ref{chinsatzeindeutig} ist $\Z_M \cong \Z_{m_1} \times \dots \times \Z_{m_n}$ mittels $\psi$\\
$\Rightarrow x$ Einheit $\Leftrightarrow \psi(x) = ( x\mod{m_1}, \dots, x \mod{m_n})$ Einheit
\\ $\Leftrightarrow x \mod{m_i}$ Einheit $\forall i = 1 \dots n$\\
$\Rightarrow \varphi(M) = \varphi(m_1) \cdot \dots \cdot \varphi(m_n)$\\
$\varphi(p^a)\underbrace{=}_{\text{"Uberlegen}}p^a - p^{a-1} = p^{a-1}(p-1)$

\subsection[Definition: Polynom]{Definition}
Sei $K$ K"orper mit Nullelement $0$ und Einselement 1:
{\renewcommand{\labelenumi}{(\roman{enumi})}\begin{enumerate}
	\item
	Ein \emph{Polynom "uber K} ist Ausdruck $f=a_0x^0+a_1x^1+\dots + a_nx^n$, $n\in \N_0, a_i\in K$.\\
	$a_i$ heißen \emph{Koeffizienten} des Polynoms.
	\begin{enumerate}
		\item
		Ist $a_i=0$, so kann man $0 \cdot x^i$ bei der Beschreibung weglassen.
		\item
		Statt $a_0x^0$ schreibt auch $a_0$
		\item
		Sind alle $a_i=0$, so schreibt man $f=0$, das Nullpolynom.
		\item
		Ist $a_i =1$, so schreibt man $x^i$ statt $1 \cdot x^i$
		\item
		Die Reihenfolge der $a_ix^i$ kann ver"andert werden, ohne dass das Polynom sich ver"andert ($x^4+2x^3 + 3= 2x^3 +  3 + x^4$)
	\end{enumerate} 
	\item
	Zwei Polynome $f$ und $g$ sind \emph{gleich}, wenn ($f=0$ und $g=0$) oder (${f=a_0 + a_1x^1+ \dots + a_nx^n},\\ {g=b_0+b_1x^1+ \dots + b_mx^m}, {a_n \neq  0, b_m \neq 0}$ und $n=m$, ${a_i = b_i}$ f"ur $i=0, \dots , n$) gilt.
	\item
	Die	Menge aller Polynome "uber $K$ bezeichnet man als $K[x]$
\end{enumerate} }

\subsection{Beispiel}
\begin{enumerate}
	\item
	$\underbrace{f}_{f(x)} = 3x^2 + \frac{1}{2}x -1 \in \Q [x] \land f \in \R [x]$
	\item
	$g= x^3 + x^2 +1 \in \Z_2[x] $ 
\end{enumerate}
Wir wollen in $K[x]$ wie in einem Ring rechnen k"onnen. Wir brauchen dazu $+$ und $\cdot$ f"ur Polynome.

\subsection[Satz und Definition: Polynomring]{Satz und Definition}
$K$ K"orper, dann wird $K[x]$ zu einem kommutativen Ring mit Eins durch folgende Verkn"upfungen:\\
$f= \underbrace{\sum_{i=0}^{n} a_ix^i}_{\text{z.B. }x+2}, \;\;\;g=\underbrace{\sum_{j=0}^{m} b_j x^j}_{x^3 + 2x + 1}$,

dann \[f + g = \underbrace{\sum_{i=0}^{\max(m,n)}(a_i+b_i)x^i}_{x^3 + 3x +3}\]

\[f \cdot g = {\sum_{i=0}^{n+m} c_ix^i}\] 
\[\text{mit } c_i= a_0b_i+a_1b_{i-1}+ \dots + a_ib_0 = \sum_{j=0}^{i} a_jb_{i-j}  \tag{Faltungsprodukt}\]
(setze $a_i$ mit $i>n$ bzw. $b_j$ mit $j>m$ gleich 0)
\begin{itemize}
	\item
	Einselement: $f=1 \;(a_0=1, a_j = 0 \;\;$f"ur $ j\geq 1)$
	\item
	Nullelement: $f= 0$
\end{itemize}
$K[x]$ heißt der \emph{Polynomring} in einer Variablen "uber $K$.\\
Beweis: Ringeigenschaften nachrechnen.

\subsection{Bemerkung}
Die +-Zeichen in der Beschreibung der Polynome entsprechen der Ring-Addition der \emph{Monome} $a_0, ax, a_2x^2, \dots, a_nx^n$

\subsection{Beispiel}

\begin{enumerate}
	\item 
	in $\Q[x], \R[x]$ Addition, Multiplikation klar
	
	\item
	in $\Z_3[x]$:
	$f = 2x^3 + 2x + 1$,
	$g = 2x^3 + x$
	
	$\begin{array}{lcl}
	f + g &=& x^3 + 1 \\
	f \cdot g &=& (2x^3 + 2x + 1)(2x^3 + x) \\
	&=& x^6 + 2x^4 + x^4 + 2x^2 + 2x^3 + x \\
	&=& x^6 + 2x^3 + 2x^2 + x
	\end{array}$
	
	\item
	in $\Z_2[x]$:
	$f = x^2 + 1$,
	$g = x + 1$
	
	$\begin{array}{lcl}
	f+g &=& x^2 + x \\
	f+f &=& 0 \\
	g \cdot g &=& x^2 + 1
	\end{array}$
	
	
\end{enumerate}


\subsection[Definition: Grad eines Polynoms]{Definition}

Sei $0 \neq f \in K[x]$

$f=a_0 + a_1x + \dots + a_nx^n$ mit $a_n \neq 0$

Dann heißt $n$ der \emph{Grad} von $f$ Grad($f$)

Grad($0$) $:= - \infty$ \\
Grad($f$) $=0$ \quad \emph{für konstante Polynome $\neq 0$}

















% % % 2.20
\subsection{Satz}
\label{sub:satz}

$K$ Körper, $f,g \in K[x]$

Dann ist $\Grad(f\cdot g) = \Grad(f) + \Grad(g)$

(Konvention: $-\infty + (-\infty) = -\infty + n = -\infty$)

\subsubsection*{Beweis}

	
	Stimmt für $f = 0$ oder $g = 0$
	
	\begin{align*}
		f &= a_0 + a_1x^1 + \ldots +  a_nx^n \quad &\text{mit} \quad  a_n \neq 0\\
		g &= b_0 + b_1x^1 + \ldots +  b_mx^m \quad &\text{mit} \quad  b_m \neq 0\\
		f \cdot g &= (\ldots) \cdot (\ldots) = \ldots + \underbrace{(a_nb_n)}_ {\mathclap{\substack{\neq 0, \\ \text{(siehe Satz 2.7
						Nullteilerfreiheit in Körpern)}}}}\cdot x^{n+m}
	\end{align*}
	
	
	Höhere Potenzen mit Koeffizienten $\neq 0$ gibt es nicht
	
	$\Rightarrow \Grad(f \cdot g) = n + m$

% % % 2.21
\subsection{Korollar}

$K$ Körper, dann
$K[x]^* = \{f \in K[x]\; |\; \Grad(f) = 0\}$,

d.h. nur die konstanten Polynome $\neq 0$ sind in $K[x]$ bez"uglich $\cdot$ invertierbar.

\[\underbrace{f}_{\mathclap{\Grad \;n}} \cdot \underbrace{f^{-1}}_{\text{müsste Grad $-n$ haben}} = \underbrace{1}_{\mathclap{\Grad \;0}} \leftarrow \text{geht nicht}\]

% % % 2.22
\subsection{Definition}


Sei $b \in K$

 $\varphi_b: K[x] \rightarrow K$, 
$f := \sum\limits_{i = 0}^n a_ix^i \mapsto f(b) := \sum\limits_{i = 0}^n a_ib^i$

ist ein surjektiver Ringhomomorphismus, der sogenannte \emph{Auswertungshomomorphismus} an der Stelle $b$.

(setze $b$ für $x$ ein)


% % % 2.23
\subsection{Definition}

$K$ Körper, $f, g \in K[x]$

$f$ \underline{teilt} $g$, $f|g$, falls ein $q \in K[x]$ existiert mit $g = q \cdot f$

(Nach \ref{sub:satz} ist dann $\Grad(f) \leq \Grad(g)$, falls $g \neq 0$)

% % % 2.24
\subsection{Definition (Division mit Rest)}

$K$ Körper, $0 \neq f \in K[x],\; g \in K[x]$

Dann existieren eindeutig bestimmte Polynome $q, r \in K[x]$ mit
$g = q \cdot f + r \; \text{und} \;\\ \Grad(r) < \Grad(f)$.

Bezeichnung:
\begin{align*}
	 r &=: g \mod f\\
	 q &=: g\; \text{div}\;  f
\end{align*}

\subsubsection*{Beweis}
Vgl. Mathe I für $\mathbb{Z}$, siehe z.B. WHK Satz 4.69

% % % 2.25
\subsection{Beispiel}
\label{sub:2.25}

\begin{enumerate}[a)]
	\item
	\begin{align*}
	g &= x^4+2x^3-x+2 \quad \in \mathbb{Q}[x]\\
	f &= 3x^2-1 \quad \in \mathbb{Q}[x]
	\end{align*}
	
	Rechne:
		
	\polylongdiv[style=C, div=:]{x^4+2x^3+0x^2-x+2}{3x^2-1}
	\item
	\[g = x^4+x^2+1 \quad \quad f=x^2+x \quad \in \mathbb{Z}_2[x]\]
	Rechne:
	\[(x^4+x^2+1) : x^2+x= \underbrace{x^2+x}_{q}\]
	
	
\end{enumerate}


% % % 2.26
\subsection{Korollar}

$K$ Körper, $a \in K$

$f \in K[x]$ ist genau dann durch $(x - a)$ teilbar, wenn $f(a) = 0$ ist (d.h. $a$ ist \underline{Nullstelle} von $f$).

\subsubsection*{Beweis}

"$\Rightarrow$" $\quad$ sei $f$ durch $(x - a)$ teilbar, d.h.
\[f = q \cdot (x - a) \Rightarrow f(a) = q(a) \cdot (\underbrace{a-a}_{0}) = 0 \quad q \in K\]

"$\Leftarrow$" $\quad$ Division mit Rest: $f = q(x - a) + r$, wobei
$\Grad(r) < \underbrace{\Grad(x-a)}_{1}$

$\Rightarrow r$ ist konstantes Polynom ($\Grad \; 0$) oder Nullpolynom $(\Grad \;(-\infty)$) also $r \in K$

\[0 = f(a) = q(a) \cdot 0 + r \Rightarrow r = 0 \quad \]
\qed

% % % 2.27
\subsection{Definition}

$K$ Körper

\begin{enumerate}[(i)]
	\item
	Ein Polynom dessen höchster von $0$ verschiedener Koeffizient gleich $1$ ist, heißt normiert.
	
	\item
	$g, h \in K[x]$, nicht beide $0$ 
	
	$f \in K[x]$ heißt \emph{größter gemeinsamer Teiler} von $g$ und $h$ ($f=\ggT(g,h)$), falls
	$f$ normiertes Polynom von maximalem Grad ist, das $g$ und $h$ teilt.
	
	\item
	$g, h \in K[x] \;\backslash\; \{0\}$ beide nicht $0$
	
	$f \in K[x]$ heißt \emph{kleinstes gemeinsames Vielfaches} von $g$ und $h$\\ ($f = kgV(g,h)$), falls
	$f$ normiertes Polynom von kleinstem Grad ist, das von $g$ und $h$ geteilt wird.
\end{enumerate}


% % % 2.28
\subsection{Bemerkung}

\begin{enumerate}[a)]
	\item
	$f = \sum\limits_{i = 0}^n a_ix^i, a_n \neq 0$, dann ist $a_n^{-1}f = x^n + \ldots\;$ normiertes Polynom.
	
	(z.B.: $f = 3x^2+x+7 \in \mathbb{R}[x]$)
	
	dann $\frac{1}{3}f = x^2+\frac{x}{3} + \frac{7}{3}$ normiert.
	
	In $\mathbb{Z}_{11}[x]: \underbrace{4}_{\mathclap{\text{Inverses von 3, denn $3\cdot 4 = 12 \equiv 1 \pmod{11}$}}}f = x^2+4x_6$ normiert.
	
	\item
	$\kgV(g,h)$ existiert und ist eindeutig:
	
	\begin{align*}
	\text{sei}\; f_1 &= \kgV(g,h), \; f_2 = \kgV(g,h)\\
	&\Rightarrow g,h | f_1, \quad g,h | f_2\\
	&\Rightarrow g,h | (f_1-f_2)
	\end{align*}
	
	\item
	$\ggT(g,h)$ existiert. Beweis Eindeutigkeit wie in $\mathbb{Z}$ (Mathe I), folgt aus.
\end{enumerate}


% % % 2.29
\subsection{Satz (von Bezout)}
\label{sub:satz_von_bezout_}

$K$ Körper, $g, h \in K[x]$, nicht beide $0$.

Dann existieren $s, t \in K[x]$, sodass
\[f = s \cdot g + t\cdot h\]

ein $\ggT$ von $g$ und $h$ ist.

(Beweis: EEA in $K[x]$, später)

 %2.30 - 2.25
 % % % % % % % % % % % % % % % % % % % % % % % % % % % % %

% % % 2.30
\subsection{Satz} Euklidischer Algorithmus in $K[x] \rightarrow$ siehe ,,Blatt''

% % %  2.31
\subsection{Satz} EEA in $K[x] \rightarrow$ siehe ,,Blatt''

% % % 2.32
\subsection{Beispiel}
$g=x^4+x^3 + 2x^2+1,  h= x^3+2x^2+2 \in \Z_3 [x]$\\
\dots TBD \dots

% % % 2.33
\subsection{Definition}
$k$ K"orper. Ein Polynom $p\in K[x]$, Grad$(p)\geq 1$ (d.h. $p\neq 0$, $p$ nicht konst., also keine Einheit) heißt \emph{irreduzibel}, falls golt: \\
Ist $p=f \cdot g$ ($f,g\in K[x]$), so ist Grad($f)= 0$ oder Grad($g)=0$ (d.h. $f$ oder $g$ ist konst. Polynom).

Bemerkung: $p= a \cdot a^{-1} \cdot p$ f"ur $a \in K \backslash \{0\}$ geht immer.

% % % 2.34
\subsection{Beispiel}
\begin{enumerate}
	\item
	$ax+b$ ($a \neq 0$) ist irreduzibel in $K[x]$ f"ur jeden K"orper $K$
	\item
	$x^2-2 \in \Q[x]$ ist irreduzibel:\\
	angenommen nicht, dann $(x^2-2) = (ax+b)(cx+d)$ mit $a,b,c \in \Q \land a,c \neq 0$\\
	$(ax+b)$ hat Nullstelle $-\frac{b}{a}$, also m"usste auch $(x^2-2)$ Nullstelle $\underbrace{-\frac{b}{a}}_{\in \Q}$ haben.
	Nullstellen von $(x^2-2)$ sind aber nur $\sqrt{2}$ und $-\sqrt{2}$, beide nicht in $\Q$ !
	\item
	$x^2-2 \in \R[x]$ ist nicht irreduzibel.\\
	$x^2-2 = \underbrace{(x+ \sqrt{2})}_{\in \R[x]} \cdot \underbrace{(x-\sqrt{2})}_{\in \R[x]}$
	\item
	$x^2+1 \in \R[x]$ ist irreduzibel
	\item
	$x^2+1 \in \Z_5[x]$ ist nicht irreduzibel:\\
	$(x^2 +1) = (x+2) \cdot (x+3) = (x^2 + 3x +2x +1) = (x^2+1)$\\
	$2 \Rightarrow (x^2+1)$ ist teilbar durch $(x-2) \hat{=} (x+3)$
\end{enumerate}

% % % 2.35
\subsection{Abschlussbemerkung}
\begin{enumerate}
	\item
	Irreduzibel Polynome in $K[x]$ entsprechen den Primzahlen in $\Z$. Man kann zeigen:
	$f = \sum_{i=0}^{n} a_i x^i \in K[x]$, $a_n \neq 0, n \geq 1$.\\
	Dann existieren eindeutig bestimmte irreduzibel Polynome $p_1, \dots, p_e$ und nat"urlichen Zahlen $m_1, \dots , m_e \in \N$ mit $f= a_n \cdot p_1^{m_1} \cdot \dots \cdot p_e^{m_e}$
	\item
	geg: Primzahl $p$, dann gibt es K"orper mit $p$ Elementen:\\
	$(\Z_p, \oplus, \odot)$
	
	Man kann zeigen: zu jeder Primzahlpotenz $p^a$ gibt es K"orper mit $p^a$ Elementen, diesen konstruiert man "uber irreduzible Polynome in $\Z_p[x]$. 
\end{enumerate}


\end{document}

