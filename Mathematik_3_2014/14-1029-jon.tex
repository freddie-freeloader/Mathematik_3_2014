 
 \subsection{Beweis}
 \begin{enumerate}
 \item
 Wir k"onnen annehmen, dass $1 \leq a < n $ (denn $a^{\varphi(n)} \mod{n}= (a \mod{n})^{\varphi(n)})$\\
 wegen ggT$(a,n)=1$ ist $a \in \Z^*_n$, das ist eine endliche Gruppe.
 
 $\stackrel{\ref{zyklischeGruppe}(iii)}{\Rightarrow} a^{|\Z^*_n|}=1(=e) \hspace{1.5cm} a \odot a \odot \dots$\\
 $\Rightarrow a^{\varphi(n)} \equiv 1 (\mod{n}) \hspace{0.7cm} a \cdot a \cdot \dots $ 
 \item
 Folgt aus (i) $(n=p,\; \varphi (p) = -1)$

\end{enumerate}
\section{Algebraische Strukturen mit 2 Verkn"upfungen: Ringe und K"orper}

\subsection[Definition: Ring]{Definition} \label{ring}
Sei $R \neq \varnothing$ eine Menge mit zwei Verkn"upfungen $+$ und $\bigdot$.
{\renewcommand{\labelenumi}{(\roman{enumi})}
\begin{enumerate}
	\item
	Wir nennen $(R, +, \cdot)$ einen \emph{Ring}, falls gilt:
	%TODO 1), 2) etc
	{\renewcommand{\labelenumi}{\arabic{enumi})}\begin{enumerate}
		\item
		$(R,+)$ ist eine abelsche Gruppe (Eselsbr"ucke: KAIN)\\
		Das neutrale Element bezeichnen wir hier mit $0$, das zu $a \in \R$ Inverse mit $-a$ (schreibe auch $a-b$ f"ur $a+(-b)$.
		\item
		$(R,\cdot)$ ist eine Halbgruppe.
		\item
		Es gelten die Distributivgesetze:\\
		$a\cdot (b+c) = (a \cdot b) + (a \cdot c) = ab + ac\\
		(a+b) \cdot c - (a \cdot c) + (b \cdot c) = ac = bc$ \qquad $\forall a, b, c \in R$
	\end{enumerate}}
	\item
	Ein Ring $(R,+, \cdot)$ heißt \emph{kommutativ} falls $\cdot$ ebenfalls kommutativ ist, also falls ${\forall a,b \in \R: a \cdot b = b \cdot a}$
	\item
	Ein Ring $(R,+, \cdot)$ heißt \emph{Ring mit Eins}, falls $(R, \cdot)$ ein Monoid ist mit neutralen Element $1\neq 0$ \;($\forall a \in R: a \cdot 1 = 1 \cdot a = a$).
	\item
	Ist $(R, +, \cdot)$ Ring mit Eins, dann heißen die bez"uglich $\cdot$ invertierbaren Elemente \emph{Einheiten}. Das zu $a$ bez"ugliche $\cdot$
	invertierbare Element bezeichnen wir mit $a^{-1}$.\\ $R^* :=$ Menge der Einheiten in $R$.
\end{enumerate}}
\subsection{Beispiel}
\begin{enumerate}
	\item
	($\Z, +, \cdot$) ist kommutativer Ring mit Eins (1)\\
	$\Z^* = \{1,-1\}$,
	$(\Q, +, \cdot), (\R, +, \cdot)$ ebenso\\
	$\Q^*=\Q \backslash \{0\}, \R^* = \R \backslash \{0\}$.
	\item
	$(2\Z, +, \cdot)$ ist ein kommutativer Ring ohne Eins
	\item
	trivialer Ring $(\{0\},+, \cdot)$ ohne Eins
	\item
	$n \in \N, n \geq 2,  (\Z_n, \oplus, \odot)$ kommutativer Ring mit Eins
	\item
	$(\R^n, \underbrace{+\; ,\; \cdot}_{\text{Komponentenweise}})$; allgemein: $R_1, \dots , R_n$ Ringe, dann $R_1, \times \cdots \times R_n$ Ring.
	\item
	$M_n (\R)$ - Menge aller $n \times n$-Matrizen  "uber $\R$, mit Matrixaddition und -multiplikation ist Ring mit Eins (=$E_n$), nicht kommutativ f"ur $ n \geq 2$.
\end{enumerate}
\subsection[Satz: Rechnen mit Ringen]{Satz (Rechnen mit Ringen)} \label{rechnenmitringen}
Sei $(R, +, \cdot)$ ein Ring, $a,b,c \in R$. Dann gilt:
{\renewcommand{\labelenumi}{(\roman{enumi})}\begin{enumerate}
	\item
	$a \cdot 0 = 0 \cdot a = 0$
	\item 
	$(-a)\cdot b = a \cdot (-b) = -(a \cdot b)$
	\item
	$(-a) \cdot (-b) = a \cdot b$
\end{enumerate}

\subsubsection*{Beweis}

\begin{enumerate}
	\item
	$a \cdot 0 = a \cdot (0+0) \underset{\ref{ring}(3)}{=}a \cdot 0 + a \cdot 0$\\
	addiere $-(a \cdot 0)$ (Inverses von $a \cdot 0$) auf beiden Seiten,  erhalte $0=a \cdot 0$\\
	Analog $0 \cdot a = 0$
	\item
	$(-a)\cdot b + a \cdot b \underset{\ref{ring}(3)}{=} (-a+a) \cdot b = 0 \cdot b \overset{(i)}{=}0$\\
	also ist $(-a \cdot b)$ Inverses zu $a \cdot b$, also $=-(a \cdot b)$.\\
	Analog $a \cdot (-b) = -(a \cdot b)$
	\item
	$(-a) \cdot (-b) \underset{(ii)}{=} -(a \cdot (-b)) \underset{(ii)}{=}-(-(a \cdot b)) = a \cdot b   $
\end{enumerate} } \qed

\subsection{Bemerkung}

\begin{enumerate}
	\item
	In jedem Ring mit Eins sind $1$ und $-1$ Einheiten (denn $(-1) \cdot (-1) = 1$, siehe \ref{rechnenmitringen}(iii))
	Es kann mehr geben (z.B. in $\Z_5$ usw.).
	Es kann auch $-1 = 1$ gelten (z.B. in $(\Z_2, \oplus, \odot)$)
	
	\item
	$0$ kann nach  \ref{rechnenmitringen}(i) nie Einheit sein (da $1 \neq 0$)
\end{enumerate}






