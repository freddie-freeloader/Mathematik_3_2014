 %8.3 - 8.9
 % % % % % % % % % % % % % % % % % % % % % % % % % % % % %
 
% % % 8.3
\subsection{Definition}

F"ur $A \in M_n(K)$ heißt $p_a(\lambda):=det(A-\lambda \cdot E_n)$ das \emph{charakteristische Polynom} von $A$.

% % % 8.4
\subsection{Beispiel}
$A=\begin{pmatrix}1 & 1 \\ -2 & 4\end{pmatrix} \in M_2(\mathbb{R}$\\
Eigenwerte, Eigenvektoren, Eig(A), $p_A(\lambda)$?

$A-\lambda\cdot E_2 = \begin{pmatrix}1 & 1 \\ -2 & 4\end{pmatrix}-\lambda\cdot \begin{pmatrix}1 & 0 \\ 0 & 1\end{pmatrix} = \begin{pmatrix}1-\lambda & 1\\ -2 & 4-\lambda\end{pmatrix}$
\begin{itemize}
	\item
	$p_A(\lambda)=\mathrm{det}\begin{pmatrix}1-\lambda & 1 \\ -2 & 4-\lambda\end{pmatrix} = (1-\lambda)(4-\lambda)-(1\cdot (-2)) = \lambda^2-5\lambda+6 = (\lambda-2)(\lambda-3)$
	
	\item
	Eigenwerte von A:\\
	$\lambda\in W$ von A $\stackrel{8.2}{\Leftrightarrow} p_A(\lambda)=0 \Leftrightarrow \lambda=2$ oder $\lambda=3$\\
	$\Rightarrow \lambda_1=2, \lambda_2=3$ Eigenwerte von A
	
	\item
	Eigenvektoren von A:\\
	x ist EV von A zum EW $\lambda_1 \Leftrightarrow x\neq \mathcal{O}$ und $(A-\lambda_1E_2)x=\mathcal{O}$
	
	also $\begin{pmatrix}1-2 & 1 \\ -2 & 4-2\end{pmatrix}x=\begin{pmatrix}0 \\ 0\end{pmatrix} \Leftrightarrow \begin{pmatrix}-1 & 1 \\ -2 & 2\end{pmatrix}\begin{pmatrix}x_1 \\ x_2\end{pmatrix}=\begin{pmatrix}0 \\ 0\end{pmatrix}$\\
	$\begin{pmatrix}-1 & 1 & | & 0 \\ -2 & 2 & | & 0\end{pmatrix}\Leftrightarrow\begin{pmatrix}-1 & 1 & | & 0 \\ 0 & 0 & | & 0\end{pmatrix}$\\
	$-x1+x2=0$ ($x_2$ ist freie Variable)\\
	$\Leftrightarrow x_1=x_2$\\
	Lösung $\left\lbrace \begin{pmatrix}x_1 \\ x_2\end{pmatrix}\in \mathbb{R}^2 | x_1=x_2\right\rbrace$ alternativ $=<\begin{pmatrix}1 \\ 1\end{pmatrix}>_{\mathbb{R}}$\\
	(oder wähle z.B. $x_2=1 \Rightarrow x_1=1$, also ist $\begin{pmatrix}1 \\ 1\end{pmatrix}$ Lösung, restliche Lösungen sind $<\begin{pmatrix}1 \\ 1\end{pmatrix}>_{\mathbb{R}}$)\\
	$\mathrm{Eig}_A(\lambda_1)=\mathrm{Ker}\begin{pmatrix}-1 & 1 \\ -2 & 2\end{pmatrix}=<\begin{pmatrix}1 \\ 1\end{pmatrix}>$\\
	x ist EV von A zum EW $\lambda_2 \Leftrightarrow x_\neq 0$ und $\begin{pmatrix}-2 & 1 \\ -2 & 1\end{pmatrix}x=\begin{pmatrix}0 \\ 0\end{pmatrix}$\\
	$\mathrm{Eig}_A(\lambda_2)=\mathrm{Ker}\begin{pmatrix}-2 & 1 \\ -2 & 1\end{pmatrix}=<\begin{pmatrix}1 \\ 2\end{pmatrix}>$\\
	zu Lösung von homogenen LGS: siehe Blatt im Moodle
\end{itemize}

% % % 8.5
\subsection{Anwendungen}
\begin{enumerate}
	\item
	Matrixpotenzen\\
	Berechne $A^{2015}=\underbrace{A\cdot A\cdot \ldots\cdot A}_{\text{2015 mal}}$ für $A=\begin{pmatrix}1 & 1 \\ -2 & 4\end{pmatrix}$ aus Bsp. 8.4\\
	Definiere $S:=\begin{pmatrix}1 & 1 \\ 1 & 2\end{pmatrix}, \ \begin{pmatrix}1 \\ 1\end{pmatrix}$ EV zu $\lambda_1, \ \ \begin{pmatrix}1 \\ 2\end{pmatrix}$ EV zu $\lambda_2$\\
	$S^{-1}\stackrel{7.10}{=}\frac{1}{\mathrm{det}S}\cdot \begin{pmatrix}2 & -1 \\ -1 & 1\end{pmatrix}=\begin{pmatrix}2 & -1 \\ -1 & 1\end{pmatrix}$\\
	dann ist $A=S\cdot \underbrace{\begin{pmatrix}2 & 0 \\ 0 & 3\end{pmatrix}}_{D}\cdot S^{-1}$, D=Diagonalmatrix (stimmt, nachrechnen!)\\
	$\Rightarrow A^{2015}=(SDS^{-1})^{2015} = (SD\underbrace{S^{-1})\cdot (S}_{E_2}DS^{-1})\cdot (SDS^{-1})\cdot \dots\cdot (SDS^{-1})$\\
	$=S\cdot D^{2015}\cdot S^{-1}$\\
	$=S\cdot \begin{pmatrix}2^{2015} & 0 \\ 0 & 3^{2015}\end{pmatrix}S^{-1}$
	
	Mit lin. Abb./Darstellungsmatr. ausgedrückt:\\
	$\varphi: \mathbb{R}^2\rightarrow\mathbb{R}^2$ mit $A=A_{\varphi}^{\mathcal{B}}=\begin{pmatrix}1 & 1 \\ -2 & 4\end{pmatrix}, \ \mathcal{B}$ kanon. Basis\\
	Bezügl. Basis $\mathcal{B}'=\left(\begin{pmatrix}1 \\ 1\end{pmatrix},\begin{pmatrix}1 \\ 2\end{pmatrix}\right)$ hat Darstellungsmatrix Diagonalgestalt $A_{\varphi}^{\mathcal{B}'}=\begin{pmatrix}2 & 0 \\ 0 & 3\end{pmatrix}$
	
	Bem.: nicht jede Darstellungsmatrix lässt sich auf Diagonalgestalt bringen, z.B. $A=\begin{pmatrix}0 & -1 \\ 1 & 0\end{pmatrix}\in M_2(\mathbb{R})$, Drehung um $90^{\circ}$\\
	$\mathrm{det}(A-\lambda E_2)=\mathrm{det}\begin{pmatrix}-\lambda & -1 \\ 1 & -\lambda\end{pmatrix}=\lambda^2+1$, keine nullstellen in $\mathbb{R}$, also keine reelen Eigenwerte!
	
	\item
	\begin{itemize}
		\item Physik: Schwingungen, Eigenfrequenz, Tacoma Narrows Bridge
		\item Googles PageRank-Algorithmus
		\item Eigenfaces / Zähne ...\\
		$\vdots$
	\end{itemize}
\end{enumerate}

% % % 8.6
\subsection{Bemerkung}
Für $A\in M_n(K)$ ist $p_A(\lambda)=\mathrm{det}(A-\lambda E_n)=\mathrm{det}\begin{pmatrix}a_{11}-\lambda & a_{12} & \dots & a_{1m} \\ a_{21} & a_{22}-\lambda & \dots & \dots \\ \dots & \dots & \dots & \dots \\ a_{n1} & \dots & \dots & a_{nm}-\lambda\end{pmatrix}$\\
ein Polynom vom Grad n (folgt aus Def. der Det.)\\
Nullstellen von $p_A(\lambda)$ sind $\in W$ von A\\
$\Rightarrow K=\mathbb{R}: \le n$ Eigenwerte\\
$K\in\mathbb{C}: $ genau n Eigenwerte (nicht notwendig verschieden)

% % % 8.7
\subsection{Definition: diagonalisierbar}
\begin{enumerate}
	\item
	Eine Matrix $A\in M_n(K)$ heißt \textbf{diagonalisierbar}, wenn eine invertierbare Matrix $S\in M_n(K)$ existiert, so dass $A=SDS^{-1}$ gilt, wobei $D=\begin{pmatrix}\lambda_1 & \dots & 0 \\ \dots & \dots & \dots  \\ 0 & \dots & \lambda_n\end{pmatrix}$ Diagonalmatrix ist. (die $\lambda_i$ sind dann gerade die Eigenwerte von A, siehe 8.8)\\
	(Bem.: Dann gilt auch $D=S^{-1}AS$)
	
	\item
	für lin. Abb:\\
	Eine lin. Abb. $\varphi: V\rightarrow V$ heißt \textbf{diagonalisierbar}, falls $V$ eine Basis $\mathcal{B}$ aus Eigenvektoren (zur zugehörigen Darstellungsmatrix) besitzt, d.h. $A_\varphi^\mathcal{B}$ ist Diagonalmatrix.
\end{enumerate}
Ist jede Matrix diagonalisierbar? 

% % % 8.8
\subsection{Satz: Spektralsatz}
\begin{enumerate}
	\item
	$A\in M_n(K)$ ist diagonalisierbar $\Leftrightarrow$ Es gibt $n$ l.u. Eigenvektoren von A
	
	\item
	Besitzt A n verschiedene Eigenwerte, so ist A diagonalisierbar.
\end{enumerate}

\textbf{Beweis:}
\begin{enumerate}
	\item
	A diagonalisierbar, d,h, $\exists S$ invertierbar mit\\
	$S^{-1}AS=\begin{pmatrix}\lambda_1 & \dots & 0 \\ \dots & \dots & \dots  \\ 0 & \dots & \lambda_n\end{pmatrix} \Leftrightarrow  AS=S\cdot \begin{pmatrix}\lambda_1 & \dots & 0 \\ \dots & \dots & \dots  \\ 0 & \dots & \lambda_n\end{pmatrix}$
	
	Sei $S=(s_1,\dots,s_n)$ (s=Spalten) Für die i-te Spalte $s_i$ von S gilt dann $A_{si}=\lambda_i\cdot s_i \ \ (i=1,\dots,n)$\\
	Also ist $s_i$ Eigenvektor zum EW $\lambda_i$ von A\\
	S ist invertierbar $\Leftrightarrow$ Spalten $s_1,\dots,s_n$ l.u. (Satz 6.8)
	
	\item
	zeige per Induktion, dass die zugehörigen Eigenvektoren linear unabhängig sind, Behauptung folgt dann aus (i)\\
	\hspace*{13cm}$\square$
\end{enumerate}

% % % 8.9
\subsection{Bemerkung zu 8.8 (ii)}
Es gib auch diagonalisierbare Matrizen, die nicht n verschiedene Eigenwerte haben!\\
z.B. $E_n$ ist bereits in Diagonalform\\
$E_n=\begin{pmatrix}1 & \dots & 0\\ 0 & \ldots & 0\\ 0 & \dots & 1\end{pmatrix}=\underbrace{\begin{pmatrix}1 & \dots & 0\\ 0 & \ddots & 0\\ 0 & \dots & 1\end{pmatrix}}_{S}\underbrace{\begin{pmatrix}1 & \dots & 0\\ 0 & \ddots & 0\\ 0 & \dots & 1\end{pmatrix}}_{D}\underbrace{\begin{pmatrix}1 & \dots & 0\\ 0 & \ddots & 0\\ 0 & \dots & 1\end{pmatrix}}_{S^{-1}}$\\
aber alle n Ew sind 1 (mit lin. Abb. ausgedrückt: $\mathrm{id}_v$ ist diagonalisierbar, $A_{\mathrm{id}_v}^B$ hat n gleiche EW)