% % % 8.3
\subsection{Definition}

F"ur $A \in M_n(K)$ heißt $p_a(\lambda):=det(A-\lambda \cdot E_n)$ das \emph{charakteristische Polynom} von $A$.

% % % 8.4
\subsection{Beispiel}

$A=\aMatrix{cc}{1 & 1 \\ 2 & 4} \in M_2(\R)$

Eigenwert, Eigenvektor, $\Eig(A)$, $p_a(A)$?

$A-\lambda \cdot E_2 = \aMatrix{cc}{1 & 1 \\ -2 & 5}-\lambda \aMatrix{cc}{1 & 0\\0 & 1}=\aMatrix{cc}{1-\lambda & 1 \\ -2 & 4- \lambda}$

\begin{itemize}
    \item
    $p_A(\lambda)= \det\aMatrix{cc}{1-\lambda & 1 \\ -2 & 4 - \lambda} = (1-\lambda)(4-\lambda) - (1 \cdot (-2)) =  \lambda^2 - 5 \lambda + 6 = (\lambda - 2) (\lambda - 3)$
    \item

    \item 
    Eigenwerte von $A$:

    $x$ ist ein EV von $A$ zum EW $\lambda \Leftrightarrow x \neq \zerovec \& (A - \lambda_1 E_2) x = \zerovec$
\end{itemize}
