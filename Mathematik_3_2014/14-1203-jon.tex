 %6.1 - 6.4
 % % % % % % % % % % % % % % % % % % % % % % % % % % % % %
 
 
 
% % %  6
\section{Matrizen und lineare Abbildungen}

% % %  6.1
\subsection{Definition}

Seien $V,W$ endlich dimensionale VR mit geordneter Basis

\[\mathcal{B} = (v_1,\cdots,v_n) \text{ von } V\] und
\[\mathcal{C} = (w_1,\cdots,w_n) \text{ von } W\] Sei
\[\varphi:V\rightarrow W \text{ lineare Abbildung}\]
\\
Stelle die Bilder $\underbrace{\varphi(v_1)}_{\in W}, \ldots, \underbrace{\varphi(v_n)}_{\in W}$ bzgl. Basis $C$ dar:

\begin{align*}
\varphi(v_1) &= a_{11} \cdot w_1 + \cdots + a_{11}w_m\\
\vdots\\
\varphi(v_n) &= a_{1n} \cdot w_1 + \cdots + a_{n1}w_m\\
\end{align*}

Dann heißt die $m\times n$ Matrix
\[\left. A_\varphi^{\mathcal{B},\mathcal{C}} :=
\begin{pmatrix}
a_{11} \cdots a_{1n}\\
\vdots \hspace{1cm} \vdots\\
a_{m1}\cdots a_{m,n}
\end{pmatrix}\right. \;\text{(Spalte $i$ enthält Koordinaten von $\varphi(v_i)$ bzgl. $\mathcal{C}$)}\]

die \emph{Darstellungsmatrix} von $\varphi$ bzgl. der Basen $\mathcal{B}$ und $\mathcal{C}$ \\(Schreibweise f"ur den Fall $\mathcal{B} =
\mathcal{C}$, dann auch  $A_\varphi^\mathcal{B}$)

Bemerkung: $\varphi$ durch $A_\varphi^{\mathcal{B},\mathcal{C}}$ eindeutig bestimmt, vgl. \ref{kern}

% % %  6.2
\subsection{Beispiel}

\begin{enumerate}[a)]
	\item
	\[V=W=\mathbb{R}^2, \quad \mathcal{B}=\mathcal{\color{red}C\color{black}}=(e_1,e_2) =
	\bigg ( \color{red}\weirdvct{1}{0}\color{black},\color{red}\weirdvct{0}{1}\color{black}\bigg )\]
	\[\varphi: V \rightarrow V, \quad v \mapsto 2v\]
	
	\[A_\varphi^{\mathcal{B},\mathcal{C}} = A_\varphi^{\mathcal{B}} = \text{?}\]
	\begin{align*}
	\begin{rcases}
	\varphi\left(\weirdvct{1}{0}\right) &= \weirdvct{2}{0} = \underline{2} \color{red}\weirdvct{1}{0}\color{black} +
	\underline{0} \color{red}\weirdvct{0}{1}\color{black}\\
	\varphi\left(\weirdvct{0}{1}\right) &= \weirdvct{0}{2} = \underline{0} \color{red}\weirdvct{1}{0}\color{black} +
	\underline{2} \color{red}\weirdvct{0}{1}\color{black}\\
	\end{rcases}
	A_\varphi^\mathcal{B} = \weirdvct{2\quad0}{0\quad2}
	\end{align*}
	
	andere Basis $\mathcal{D} = \left( \weirdvct{1}{2},\weirdvct{0}{2}\right) \quad
	A_\varphi^{\mathcal{B},\color{green}\mathcal{D}\color{black}}$
	
	\begin{align*}
	\begin{rcases}
	\varphi\left(\weirdvct{1}{0}\right) &= \weirdvct{2}{0} = \underline{2} \color{green}\weirdvct{1}{2}\color{black} -
	\underline{2} \color{green}\weirdvct{0}{2}\color{black}\\
	\varphi\left(\weirdvct{0}{1}\right) &= \weirdvct{0}{2} = \underline{0} \color{green}\weirdvct{1}{2}\color{black} +
	\underline{1} \color{green}\weirdvct{0}{2}\color{black}\\
	\end{rcases}
	A_\varphi^{\mathcal{B},\color{green}\mathcal{D}\color{black}} = \weirdvct{2\quad0}{-2\quad1}
	\end{align*}
	
	\item
	$V = W$ mit $\dim V = n,\;\; \mathcal{B}$ bel. Basis,$\;\; \varphi = id_V,$ dann ist: \[A_\varphi^\mathcal{B} = E_n\]
	
	\item
	$V = W = \mathbb{R}^2,\; \mathcal{B} = \mathcal{C}= (e_1,e_2)$
		
	$\varphi$ Drehung um Nullpunkt um $\alpha$ gegen Uhrzeigersinn
	
	\[\Rightarrow A_\varphi^\mathcal{B} = \weirdvct{\cos \alpha \quad -\sin \alpha}{\sin \alpha\quad \cos \alpha}\]
	Vgl. Beispiel \ref{beispiel:drehung}
	
	\item
	$V=W=\mathbb{R}^2,\; \mathcal{B} = (e_1,e_2)$
	
	$\varphi:$ Spiegelung an der $\underbrace{\big<e_1\big>}_{\mathclap{x_1-\text{Achse}}}$, d.h. $\varphi:\weirdvct{x_1}{x_2}\mapsto \weirdvct{x_1}{-x_2}\quad
	A_\varphi^\mathcal{B} = \weirdvct{1\quad0}{0\quad-1}$
	
	andere Basis $\mathcal{B}' = \left( \weirdvct{1}{1},\weirdvct{1}{-1} \right)$
	
	$A_\varphi^{\mathcal{B},\mathcal{B}'} =$ ?
	
	\begin{align*}
	\varphi\left(\weirdvct{1}{0}\right) &= \weirdvct{1}{0} = a_{11}\weirdvct{1}{1} + a_{21}\weirdvct{1}{-1}\\
	\varphi\left(\weirdvct{0}{1}\right) &= \weirdvct{0}{-1} = a_{12}\weirdvct{1}{1} + a_{22}\weirdvct{1}{-1}\\
	\end{align*}
	
	$\Rightarrow$ LGS, ausrechnen, erhalte:
	\[A_\varphi^{\mathcal{B},\mathcal{B}'} = \weirdvct{\frac{1}{2}\quad \frac{1}{2}}{-\frac{1}{2}\quad \frac{1}{2}}\]
	
	\item
	andersherum:
	
	$V=W=\mathbb{R}, \quad \mathcal{B}=(e_1,e_2)$
	
	$A_\varphi^\mathcal{B} = \weirdvct{1\quad2}{3\quad4}$
	
	Was ist $\varphi\left(\weirdvct{7}{-5}\right)$
	
	$\varphi\left(\weirdvct{1}{0}\right) = \weirdvct{1}{3}$
	
	$\varphi\left(\weirdvct{0}{1}\right) = \weirdvct{2}{4} = \varphi\left(7 \weirdvct{1}{0} - 5 \weirdvct{0}{1}\right) = 7 \varphi\left(\weirdvct{1}{0}\right) -
	5 \varphi\left(\weirdvct{0}{1}\right) = 7 \weirdvct{1}{5} + (-5) \weirdvct{2}{4} = \weirdvct{-3}{1}$\\
	
	\underline{Gegeben:}
	
	Koordinaten eines Punktes bzgl. Basis $\mathcal{B}$ (z.B. Roboterkoordinaten), Abbildung $\varphi$\\
	
	\underline{Gegeben:}
	
	Koordinaten dieses Punktes bzgl. Basis $\mathcal{C}$ (Weltkoordinatensystem) $\rightarrow$ später
	
	Koordinaten des mit $\varphi$ abgebildeten Punktes bzgl. $\mathcal{C} \rightarrow$ jetzt
	
\end{enumerate}



% % %  6.3
\subsection{Satz}

$V,W,\mathcal{B},\mathcal{C},\varphi$ wie in 6.1\bigskip

Sei $v \in V,K_B(v)$ sei Koordinatenvektor von $v$ bzgl. $\mathcal{B}$ (enthält Koordinaten von $v$ bzgl. $\mathcal{B}$)

Dann lässt sich der Koordinatenvektor von $\varphi(b)$ bzgl. $\mathcal{C}$ berechnen als
\[K_\mathcal{C}(\varphi(V)) = A_\varphi^{\mathcal{B},\mathcal{C}} \cdot K_\mathcal{B}(v)\]

\emph{Beweis}: nacher

% % % 6.4
\subsection{Beispiel}

$\dim(V) = 3 \quad \mathcal{B} = (v_1,v_2,v_3) \quad \varphi: V \rightarrow W$\\
$\dim(W) = 2 \quad \mathcal{C} = (w_1,w_2)$

mit
\[A_\varphi^{\mathcal{B},\mathcal{C}} \weirdvct{1\quad1\quad-2}{2\quad0\quad3}\]

$v = \underline{5}\cdot v_1 \underline{-2} \cdot v_2 +\underline{4} \cdot v_3$, d.h. Koordinaten von $v$ bzgl. $\mathcal{B}$ sind $5,-2,4$

\[K_\mathcal{B} =
\aMatrix{c}
{5\\-2\\4}
\]

Was sind Koordinaten von $\varphi(v)$ in Basis $\mathcal{C}$?

\[
K_\mathcal{C} = \aMatrix{ccc}{1 & 1 & -2 \\ 2 & 0 & 3}\cdot
\aMatrix{c}{5\\-2\\4}= \weirdvct{-5}{22}
\]

d.h. $\varphi(v) = -5 \cdot w_1 + 22 \cdot w_2$\bigskip

\subsubsection*{Beweis}

\begin{align*}
A_\varphi^{\mathcal{B},\mathcal{C}} =
\aMatrix{c}{a_{11}\cdots a_{1n}\\\vdots\\a_{m1}\cdots a_{mn}},
\quad v = \sum\limits_{i=0}^n \lambda_i v_i, \quad K_\mathcal{B}(v) =
\aMatrix{c}{\lambda_1\\\vdots\\\lambda_n}
\end{align*}
\begin{align*}
A_\varphi^{\mathcal{B},\mathcal{C}} \cdot K_\mathcal{B}(v) =
\aMatrix{c}{\sum\limits_{i=1}^n a_{1i}\lambda_i\\\vdots\\\sum\limits_{i=1}^n a_{mi}\lambda_i}
\end{align*}
\begin{align*}
\varphi(v) &= \varphi\left(\sum\cdots\right)\\
&= \sum\limits_{i=1}^n \lambda_i \underbrace{\varphi(v_i)}_{\mathclap{\sum\limits_{k=1}^m a_{ki}w_k}}\\
&= \sum\limits_{k=1}^m \underbrace{\left(\sum\limits_{i=1}^n \lambda_i a_{ki}\right)}_{\mathclap{\text{Koordinaten von }
		\varphi(v) \text{ bzgl. } \mathcal{C}}} \cdot w_k\\
\end{align*}
\begin{align*}
K_\varphi(\varphi(v)) =
\aMatrix{c}{\sum\limits_{i=1}^n \lambda_i a_{1i}\\\vdots\\\sum\limits_{i=1}^n \lambda_i a_{mi}}
\end{align*}
\qed