 %6.1 - 6.15
 % % % % % % % % % % % % % % % % % % % % % % % % % % % % %
 
 
 
% % %  6
\section{Matrizen und lineare Abbildungen}

% % %  6.1
\subsection{Definition}

Seien $V,W$ endlich dimensionale VR mit geordneter Basis

\[\mathcal{B} = (v_1,\cdots,v_n) \text{ von } V\] und
\[\mathcal{C} = (w_1,\cdots,w_m) \text{ von } W\] Sei
\[\varphi:V\rightarrow W \text{ lineare Abbildung}\]
\\
Stelle die Bilder $\underbrace{\varphi(v_1)}_{\in W}, \ldots, \underbrace{\varphi(v_n)}_{\in W}$ bzgl. Basis $C$ dar:

\begin{align*}
\varphi(v_1) &= a_{11} \cdot w_1 + \cdots + a_{m1}w_m\\
\vdots\\
\varphi(v_n) &= a_{1n} \cdot w_1 + \cdots + a_{mn}w_m\\
\end{align*}

Dann heißt die $m\times n$ Matrix
\[\left. A_\varphi^{\mathcal{B},\mathcal{C}} :=
\begin{pmatrix}
a_{11} \cdots a_{1n}\\
\vdots \hspace{1cm} \vdots\\
a_{m1}\cdots a_{mn}
\end{pmatrix}\right. \;\text{(Spalte $i$ enthält Koordinaten von $\varphi(v_i)$ bzgl. $\mathcal{C}$)}\]

die \emph{Darstellungsmatrix} von $\varphi$ bzgl. der Basen $\mathcal{B}$ und $\mathcal{C}$ \\(Schreibweise f"ur den Fall $\mathcal{B} =
\mathcal{C}$, dann auch  $A_\varphi^\mathcal{B}$)

Bemerkung: $\varphi$ durch $A_\varphi^{\mathcal{B},\mathcal{C}}$ eindeutig bestimmt, vgl. \ref{kern}

% % %  6.2
\subsection{Beispiel}

\begin{enumerate}[a)]
	\item
	\[V=W=\mathbb{R}^2, \quad \mathcal{B}=\mathcal{\color{red}C\color{black}}=(e_1,e_2) =
	\bigg ( \color{red}\weirdvct{1}{0}\color{black},\color{red}\weirdvct{0}{1}\color{black}\bigg )\]
	\[\varphi: V \rightarrow V, \quad v \mapsto 2v\]
	
	\[A_\varphi^{\mathcal{B},\mathcal{C}} = A_\varphi^{\mathcal{B}} = \text{?}\]
	\begin{align*}
	\begin{rcases}
	\varphi\left(\weirdvct{1}{0}\right) &= \weirdvct{2}{0} = \underline{2} \color{red}\weirdvct{1}{0}\color{black} +
	\underline{0} \color{red}\weirdvct{0}{1}\color{black}\\
	\varphi\left(\weirdvct{0}{1}\right) &= \weirdvct{0}{2} = \underline{0} \color{red}\weirdvct{1}{0}\color{black} +
	\underline{2} \color{red}\weirdvct{0}{1}\color{black}\\
	\end{rcases}
	A_\varphi^\mathcal{B} = \weirdvct{2\quad0}{0\quad2}
	\end{align*}
	
	andere Basis $\mathcal{D} = \left( \weirdvct{1}{2},\weirdvct{0}{2}\right) \quad
	A_\varphi^{\mathcal{B},\color{green}\mathcal{D}\color{black}}$
	
	\begin{align*}
	\begin{rcases}
	\varphi\left(\weirdvct{1}{0}\right) &= \weirdvct{2}{0} = \underline{2} \color{green}\weirdvct{1}{2}\color{black} -
	\underline{2} \color{green}\weirdvct{0}{2}\color{black}\\
	\varphi\left(\weirdvct{0}{1}\right) &= \weirdvct{0}{2} = \underline{0} \color{green}\weirdvct{1}{2}\color{black} +
	\underline{1} \color{green}\weirdvct{0}{2}\color{black}\\
	\end{rcases}
	A_\varphi^{\mathcal{B},\color{green}\mathcal{D}\color{black}} = \weirdvct{2\quad0}{-2\quad1}
	\end{align*}
	
	\item
	$V = W$ mit $\dim V = n,\;\; \mathcal{B}$ bel. Basis,$\;\; \varphi = id_V,$ dann ist: \[A_\varphi^\mathcal{B} = E_n\]
	
	\item
	$V = W = \mathbb{R}^2,\; \mathcal{B} = \mathcal{C}= (e_1,e_2)$
		
	$\varphi$ Drehung um Nullpunkt um $\alpha$ gegen Uhrzeigersinn
	
	\[\Rightarrow A_\varphi^\mathcal{B} = \weirdvct{\cos \alpha \quad -\sin \alpha}{\sin \alpha\quad \cos \alpha}\]
	Vgl. Beispiel \ref{beispiel:drehung}
	
	\item
	$V=W=\mathbb{R}^2,\; \mathcal{B} = (e_1,e_2)$
	
	$\varphi:$ Spiegelung an der $\underbrace{\big<e_1\big>}_{\mathclap{x_1-\text{Achse}}}$, d.h. $\varphi:\weirdvct{x_1}{x_2}\mapsto \weirdvct{x_1}{-x_2}\quad
	A_\varphi^\mathcal{B} = \weirdvct{1\quad0}{0\quad-1}$
	
	andere Basis $\mathcal{B}' = \left( \weirdvct{1}{1},\weirdvct{1}{-1} \right)$
	
	$A_\varphi^{\mathcal{B},\mathcal{B}'} =$ ?
	
	\begin{align*}
	\varphi\left(\weirdvct{1}{0}\right) &= \weirdvct{1}{0} = a_{11}\weirdvct{1}{1} + a_{21}\weirdvct{1}{-1}\\
	\varphi\left(\weirdvct{0}{1}\right) &= \weirdvct{0}{-1} = a_{12}\weirdvct{1}{1} + a_{22}\weirdvct{1}{-1}\\
	\end{align*}
	
	$\Rightarrow$ LGS, ausrechnen, erhalte:
	\[A_\varphi^{\mathcal{B},\mathcal{B}'} = \weirdvct{\frac{1}{2}\quad \frac{1}{2}}{-\frac{1}{2}\quad \frac{1}{2}}\]
	
	\item
	andersherum:
	
	$V=W=\mathbb{R}, \quad \mathcal{B}=(e_1,e_2)$
	
	$A_\varphi^\mathcal{B} = \weirdvct{1\quad2}{3\quad4}$
	
	Was ist $\varphi\left(\weirdvct{7}{-5}\right)$
	
	$\varphi\left(\weirdvct{1}{0}\right) = \weirdvct{1}{3}$
	
	$\varphi\left(\weirdvct{0}{1}\right) = \weirdvct{2}{4} = \varphi\left(7 \weirdvct{1}{0} - 5 \weirdvct{0}{1}\right) = 7 \varphi\left(\weirdvct{1}{0}\right) -
	5 \varphi\left(\weirdvct{0}{1}\right) = 7 \weirdvct{1}{5} + (-5) \weirdvct{2}{4} = \weirdvct{-3}{1}$\\
	
	\underline{Gegeben:}
	
	Koordinaten eines Punktes bzgl. Basis $\mathcal{B}$ (z.B. Roboterkoordinaten), Abbildung $\varphi$\\
	
	\underline{Gegeben:}
	
	Koordinaten dieses Punktes bzgl. Basis $\mathcal{C}$ (Weltkoordinatensystem) $\rightarrow$ später
	
	Koordinaten des mit $\varphi$ abgebildeten Punktes bzgl. $\mathcal{C} \rightarrow$ jetzt
	
\end{enumerate}



% % %  6.3
\subsection{Satz}

$V,W,\mathcal{B},\mathcal{C},\varphi$ wie in 6.1\bigskip

Sei $v \in V,K_B(v)$ sei Koordinatenvektor von $v$ bzgl. $\mathcal{B}$ (enthält Koordinaten von $v$ bzgl. $\mathcal{B}$)

Dann lässt sich der Koordinatenvektor von $\varphi(b)$ bzgl. $\mathcal{C}$ berechnen als
\[K_\mathcal{C}(\varphi(V)) = A_\varphi^{\mathcal{B},\mathcal{C}} \cdot K_\mathcal{B}(v)\]

\emph{Beweis}: nacher

% % % 6.4
\subsection{Beispiel}

$\dim(V) = 3 \quad \mathcal{B} = (v_1,v_2,v_3) \quad \varphi: V \rightarrow W$\\
$\dim(W) = 2 \quad \mathcal{C} = (w_1,w_2)$

mit
\[A_\varphi^{\mathcal{B},\mathcal{C}} \aMatrix{ccc}{1 & 1 & -2 \\ 2 & 0 & 3}\]

$v = \underline{5}\cdot v_1 \underline{-2} \cdot v_2 +\underline{4} \cdot v_3$, d.h. Koordinaten von $v$ bzgl. $\mathcal{B}$ sind $5,-2,4$

\[K_\mathcal{B} =
\aMatrix{c}
{5\\-2\\4}
\]

Was sind Koordinaten von $\varphi(v)$ in Basis $\mathcal{C}$?

\[
K_\mathcal{C} = \aMatrix{ccc}{1 & 1 & -2 \\ 2 & 0 & 3}\cdot
\aMatrix{c}{5\\-2\\4}= \weirdvct{-5}{22}
\]

d.h. $\varphi(v) = -5 \cdot w_1 + 22 \cdot w_2$\bigskip

\subsubsection*{Beweis}

\begin{align*}
A_\varphi^{\mathcal{B},\mathcal{C}} =
\aMatrix{c}{a_{11}\cdots a_{1n}\\\vdots\\a_{m1}\cdots a_{mn}},
\quad v = \sum\limits_{i=0}^n \lambda_i v_i, \quad K_\mathcal{B}(v) =
\aMatrix{c}{\lambda_1\\\vdots\\\lambda_n}
\end{align*}
\begin{align*}
A_\varphi^{\mathcal{B},\mathcal{C}} \cdot K_\mathcal{B}(v) =
\aMatrix{c}{\sum\limits_{i=1}^n a_{1i}\lambda_i\\\vdots\\\sum\limits_{i=1}^n a_{mi}\lambda_i}
\end{align*}
\begin{align*}
\varphi(v) &= \varphi\left(\sum\cdots\right)\\
&= \sum\limits_{i=1}^n \lambda_i \underbrace{\varphi(v_i)}_{\mathclap{\sum\limits_{k=1}^m a_{ki}w_k}}\\
&= \sum\limits_{k=1}^m \underbrace{\left(\sum\limits_{i=1}^n \lambda_i a_{ki}\right)}_{\mathclap{\text{Koordinaten von }
		\varphi(v) \text{ bzgl. } \mathcal{C}}} \cdot w_k\\
\end{align*}
\begin{align*}
K_\varphi(\varphi(v)) =
\aMatrix{c}{\sum\limits_{i=1}^n \lambda_i a_{1i}\\\vdots\\\sum\limits_{i=1}^n \lambda_i a_{mi}}
\end{align*}
\qed

% % % 6.5
\subsection{Bemerkung / Korollar zu 6.3}
Der Koordinatenvektor kann als Bild der ''Koordinatenabbildung''\\
$K_B: V\rightarrow K^n$\\
$v=\sum_{i=0}^n \lambda v_i \mapsto \begin{pmatrix}\lambda_1 \\ \vdots \\ \lambda_n\end{pmatrix}$\\
aufgefasst werden, dann erhalte folgende Übersicht\\
(dim V=n, Basis B) $\begin{array}{lcr}
V& \stackrel{\varphi}{\rightarrow} & W\\
\downarrow K_B & & \downarrow K_C\\
K^n & \stackrel{\rightarrow}{\text{Multiplikation mit } A_{\varphi}^{B,C}} & K_m (*)
\end{array}$ (dim W=m, Basis C)\\
(*): $\underbrace{K_{C}\varphi(v))}_{(*)}=A_{\varphi}^{B,C} K_B(v)$
Damit folgt:\\
jede lin. Abb $K^n\rightarrow K^m$ (K Körper) ist von der Form $\varphi(x)=Ax$ für ein $A \in M_{m,n}(K)$

\subsection*{Beweis:}
Benutze kanonische Basis von $K^n$ bzw. $K^m$. Dann stimmen Elemente von $K^n$ bzw. $K^m$ mit ihren Koodrdinatenvektoren bzgl. Basis überein, Beh. folgt aus 6.3

% % % 6.6
\subsection{Satz (Eigenschaften der Darstellungsmatrix)}
U, V, W VR mit Basen B, C, D\\
$\varphi_1,\varphi_2,\varphi: U\rightarrow V, \Psi: V\rightarrow W$
\begin{enumerate}
	\item
	$A_{\varphi_1+\varphi_2}^{B,C} = A_{\varphi_1}^{B,C} + A_{\varphi_2}^{B,C}$
	
	\item
	$A_{\lambda \varphi}^{B,C} = \lambda\cdot A_{\varphi}^{B,C} \ (\lambda\in K)$
	
	\item
	$A_{\Psi_0\varphi}^{B,D} = A_{\Psi}^{C,D}\cdot A_{\varphi}^{B,C}$
\end{enumerate}
(D.h.: Der Hintereinanderausführung von lin. Abb. entspricht das Matrixprodukt der Darstellungsmatrizen)

\subsection*{Beweis:}
Übungsaufgabe\\
\hspace*{13cm}$\square$


% % % 6.7
\textbf{Folgerung:}
\subsection{Satz:}
V ein K-VR, dim(V)=n, Basis B\\
$\varphi: V\rightarrow V$ lin. Abb. mit $A_{\varphi}^B$\\
Dann gilt:\\
$\varphi$ invertierbar (bij.) $\Rightarrow A_{\varphi}^B$ invertierbar und $A_{\varphi^{-1}}^B$ ist dann $=(A_{\varphi}^B)^{-1}$

\subsection*{Beweis:}
\begin{enumerate}
	\item[''$\Rightarrow$'']
	Sei $\varphi$ invertierbar, d.h. $\exists \varphi^{-1}$\\
	Dann ist $A_{\varphi}^B\cdot A_{\varphi^{-1}}^B \underbrace{=}_{(6.6)} A_{\varphi\circ\varphi^{-1}}^B = A_{id}^B = E_n$\\
	analog $A_{\varphi^{-1}}^B=A_{\varphi}^B$
	
	\item[''$\Leftarrow$'']
	Sei $A_{\varphi}^B$ invertierbar, d.h. $\exists Y$ mit $A_{\varphi}^B\cdot Y=Y\cdot A_{\varphi}^B = E_n$\\
	Dann ist Y Abbildungsmatrix für eine eindeutig bestimmte lineare Abbildung $\Psi: V\rightarrow V,  \ Y=A_{\Psi}^B$\\
	$\stackrel{(6.6)}{\Rightarrow} A_{\varphi\circ\Psi}^B = A_{\varphi}^B\cdot A_{\Psi}^B = E_n$\\
	d.h. $\varphi\circ\Psi=\Psi\circ\varphi=idv \Rightarrow \varphi$ ist bij. (invertierbar.)\\
	\hspace*{13cm}$\square$
	
	\textbf{Fragen:} wann ist eine Matrix (lineare Abbildung) invertierbar?\\
	Wie berechnet man inverse?
\end{enumerate}

% % % 6.8
\subsection{Satz:}
$A\in M_{n,n}(K)$\\
Dann gilt: A ist invertierbar $\Leftrightarrow \underbrace{\mathrm{rg}(A)=n}_{\text{d.h. alle Zeilen/Spalten sind l.u.}}$

\subsection*{Beweis:}
Betrachte $\varphi: K^n\rightarrow K^n$ mit $\varphi(x)=Ax$\\
Dann ist $A = A_{\varphi}^B$ (B Basis von $K^n$)\\
A invertierbar $\stackrel{(6.7)}{\Leftrightarrow} \varphi$ invertierbar (bij.)\\
A invertierbar $\stackrel{(5.12)}{\Leftrightarrow} \varphi$ surjektiv\\
A invertierbar $\Leftrightarrow \mathrm{rg}(\varphi)=n$\\
A invertierbar $\stackrel{(5.13)}{\Leftrightarrow} \mathrm{rg}(A)=n$\\
\hspace*{13cm}$\square$

% % % 6.9
\subsection{Berechnung von Inversen}
$\rightarrow$ Blatt (Gauß) + Bsp.

\textbf{Gesehen:} Darstellungsmatrix hängt von der Wahl der Basen ab. Wie ändert sie sich, wenn man Basen ändert? Dieser Vorgang wird als Basistransformation bezeichnet.

% % % 6.10
\textbf{Dazu:}
\subsection{Definition/Satz:}
Sei V ein VR, $B=(v_1,\dots,v_n)$ und $B'=(v_1',\dots,v_n')$ Basis von V\\
Schreiben $v_i'$ als LK der Basisvektoren von B$(i=1\dots n)$, also\\
$v_1'=s_{11}v_1+\dots+s_{n1}v_n$\\
$v_n'=s_{1n}v_1+\dots+s_{nn}v_n$\\
Dann heißt $S_{B,B'}=\begin{pmatrix}s_{11} & s_{12} & \dots & s_{1n} \\ \vdots & \vdots & \dots & \vdots \\ s_{n1} & s_{n2} & \dots & s_{nn}\end{pmatrix}$ Basiswechselmatrix\\
Ihre Spalten sind die Koordinatenvektoren der Basisvektoren von $B'$ bzgl. B

Analog:\\
Stelle $v_k$ als LK der Basisvektoren von $B'$ das ($v_k=\sum_{l=1}^n t_{lk}v_l')$\\
erhalte so $S_{B',B} (=(t_{lk})_{l,k=1\dots n})$\\
Es gilt $(S_{B,B'})^{-1}=(S_{B',B})$\\
(nachrechnen: $S_{B,B'}\cdot S_{B',B}=E_n$)

% % % 6.11
\subsection{Satz: Koordinaten umrechnen}
V mit $B,B'$ wie in 6.10, $v\in V$

Dann ist $K_{B'}\cdot (v)=S_{B',B}\cdot K_B(v)$

\subsection*{Beweis:} $v=\sum_{k=1}^n \lambda_k\cdot \underbrace{v_k}_{\sum_{l=1}^n t_{lk}v_l'}$, also $K_B(v)=\begin{pmatrix}\lambda_1 \\ \vdots \\ \lambda_n\end{pmatrix}$\\
also $V=\sum_{l=1}^n\underbrace{\left(\sum_{k=1}^n \lambda_k t_{lk}\right)}_{\text{neue Koordinaten (bzgl. B')}}\cdot \underbrace{v_l'}_{\in B'}$

% % % 6.12
\subsection{Beispiel}
$V=\mathbb{R}^2, \ B=(e_1,e_2), \ B'=\left(v_1=\begin{pmatrix}1 \\ 1\end{pmatrix}, v_2=\begin{pmatrix}1 \\ -2\end{pmatrix}\right)$\\
$S_{B,B'}=\begin{pmatrix}1 & 1 \\ 1 & -2 \end{pmatrix}, \ \ v_1=1\cdot e_1+1\cdot e_2, \ v_2=1\cdot e_1-2\cdot e_2$\\
$S_{B',B}=(S_{B,B'})^{-1}=(\dots$ Gauß $\dots)=\begin{pmatrix}\frac{2}{3} & \frac{1}{3} \\ \frac{1}{3} & -\frac{1}{3}\end{pmatrix}$

$i=\begin{pmatrix}3 \\ 6\end{pmatrix} \ K_B(u)=\begin{pmatrix}3 \\ 6\end{pmatrix} \ (u=3\cdot e_1+6\cdot 3_2)$\\
Koordinaten von u in Basis B'?\\
$K_{B'}(u)=S_{B',B}\cdot K_B(u)=\begin{pmatrix}\frac{2}{3} & \frac{1}{3} \\ \frac{1}{3} & -\frac{1}{3}\end{pmatrix}\cdot \begin{pmatrix}3 \\ 6\end{pmatrix}=\frac{4}{-1}$

(also ist $u=4\cdot v_1-1\cdot v_2$)

Mit der Basiswechselmatrix kann man auch Darstellungsmatrizen umrechnen:

% % % 6.13
\subsection{Satz: Darstellungsmatrizen umrechnen}
$\varphi: V\rightarrow W$ lin. Abb.\\
$B,B'$ Basen von V, $C,C'$ Basen von W

$\Rightarrow A_{\varphi}^{B',C'}=S_{C',C}A_{\varphi}^{B,C}S_{B,B'}$

\subsection*{Beweis:} Sei $v\in V$\\
nach 6.3:\\
$A_{\varphi}^{B',C'}\cdot K_{B'}(v)=K_{C'}(\varphi(v))$\\
Koordinatenwechsel nach C (6.11): $= S_{C',C}\cdot K_{\varphi}(v))$\\
6.3: $=S_{C',C}\cdot A_{\varphi}^{B,C}\cdot K_B(v)$\\
Koordinatenwechsel nach B' (6.11): $S_{C',C}\cdot A_{\varphi}^{B,C}\cdot s_{B,B'}\cdot K_{B'}(v)$\\
\hspace*{13cm}$\square$

% % % 6.14
\subsection{Korollar}
$\varphi: V\rightarrow V$ lin. Abb.\\
$B,B'$ Basen von V. $S:=S_{B,B'}$\\
$\Rightarrow A_{\varphi}^{B'} = S^{-1}A_{\varphi}^B S$

% % % 6.15
\subsection{Beispiel}
$V, B, B'$ wie in 6.12\\
$\varphi:$ Spiegelung an der $x_1$-Achse\\
$\Rightarrow A_{\varphi}^B = \begin{pmatrix}1 & 0 \\ 0 & -1\end{pmatrix}$\\
$A_{\varphi}^{B'} \stackrel{=}{6.14} \begin{pmatrix}\frac{2}{3} & \frac{1}{3} \\ \frac{1}{3} & -\frac{1}{3}\end{pmatrix}\begin{pmatrix}1 & 0 \\ 0 & -1\end{pmatrix}\begin{pmatrix}1 & 1\\ 1 & -2\end{pmatrix} = \begin{pmatrix}\frac{1}{3} & \frac{4}{3} \\ \frac{2}{3} & -\frac{1}{3}\end{pmatrix}$