% % % 5.9
\subsection{Satz}

$V,W$ $K$-VR, $\dim(V)=n$

$\{v_1, \dots, v_n\}$ Basis von $V$ \\
$w_1,\dots, w_2 $ Vektoren aus $W$ (nicht notwendig verschieden)

Dann $\exists!$ lineare Abbildung



$\varphi: V \rightarrow W \;$mit$\; \varphi(v_i) = w_i \; (i=1,...,n)$

und zwar: 

$\left.
\begin{array}{l}
\varphi: V \rightarrow W\\
v=\sum_{i=1}^{n} \lambda_i v_i \mapsto \sum_{i=i}^{n} \lambda_i w_i
\end{array}
\right\rbrace$\huge*\normalsize\\

D.h.: wenn man weiß, wie die Basisvektoren abgebildet werden, dann kennt man die lineare Abbildung vollständig.

% % % 
\subsubsection*{Beweis}
Für $\varphi$ aus * gilt:
\begin{itemize}
	\item $\varphi$ ist linear
	\item $\varphi(v_i) = w_i$\\
	$\varphi(v_1)=\varphi(1\cdot v_1 + 0 \cdot v_2 + ... + 0 \cdot v_n) = 1 \cdot w_1 + 0 \cdot w_2 + ... + 0 \cdot w_n = 1 \cdot w_1 = w_1$ usw.
	\item $\varphi$ ist eindeutig.
	
	Angenommen $\exists \;\psi: V \rightarrow W$ lin. Abb. mit $\psi(v_i) = w_i \; \forall i=1...n$
	
	Dann ist $\psi(\sum_{i=1}^{n} \lambda_i v_i) = \sum_{i=1}^{n} \lambda_i(\psi(v_i)) = \sum_{i=1}^{n} \lambda_i w_i = \varphi(\sum_{i=1}^{n} \lambda_i v_i) $
	\qed
\end{itemize}

% % % 5.10
\subsection{Beispiel}

$V = \R^2, \varphi$ Drehung um Winkel $\alpha\; (0 \leq \alpha < 2 \pi)$ um Nullpunkt gegen den Uhrzeigersinn.

$\varphi$ ist lin. Abb.:\\
$\varphi(\alpha_1 + \alpha_2) = \varphi(\alpha_1) + \varphi(\alpha_2)$\\
$\varphi(\lambda \alpha) = \lambda \varphi(\alpha)$\\

$\varphi: e_1 = \vct{1\\0} \mapsto \vct{\cos \alpha\\\sin \alpha}$

$e_2 = \vct{0\\1} \mapsto \vct{-\sin \alpha\\\cos \alpha}$

allg. Vektor $x = \vct{x_1\\x_2} = x_1 \cdot \vct{1\\0} + x_2 \cdot \vct{0\\1}$
\begin{align*}
\varphi: x \mapsto& \;x_1 \cdot \varphi(e_1) + x_2 \cdot \varphi(e_2)\\
=&\; x_1 \cdot \vct{\cos \alpha \\ \sin \alpha} + x_2 \cdot 
\vct{-\sin \alpha \\ \cos \alpha}\\
=& \vct{x_1 \cdot \cos \alpha - x_2 \cdot \sin \alpha \\
	x_1 \cdot \sin \alpha + x_2 \cdot \cos \alpha}\\
=&\; A \cdot x
\end{align*}
 mit $A = \begin{bmatrix}
\cos \alpha & \sin \alpha\\
-\sin \alpha & \cos \alpha
\end{bmatrix}$

% % % 5.11
\subsection{Satz (Dimensionsformel)}
\label{dimensionsformel}

$V$ endl. dim. $K$-VR, $\varphi: V \rightarrow W$ lin. Abb.

Dann gilt: \\$\dim(V) = \dim(\ker(\varphi)) +  \underbrace{\rg(\varphi)}_{\mathclap{\dim(\varphi(V))}}$

% % %
\subsubsection*{Beweis}
Sei $u_1,...,u_k$ Basis von ker($\varphi$)

Ergänze zu Basis $u_1,...,u_n$ von V (Mathe 2, Basisergänzungssatz)

Setze $U:= \left< u_{k+1},..., u_n \right>$

Dann ist $\ker(\varphi) \cap U = \{ \zerovec \}$, \\d.h. kein Element außer $\zerovec$ liegt in U, \\also hat die Abb. $\varphi|_U$ den 

$\ker(\varphi|_U) = \{ \zerovec \}$, 

ist damit nach Satz 5.7 (ii) injektiv. \\
Deshalb ist dim(U) = dum($\varphi(U)$). 

Außerdem ist $\varphi(U) = \varphi(V)$\\
$\Rightarrow \dim(V) = \dim(\ker(\varphi)) + \underbrace{\dim(U)}_{\mathclap{\dim(\varphi(U)) = \dim(\varphi(V)) = \rg(\varphi)}}$

\qed
% % % 5.12
\subsection{Korollar}
$V,\; W$ endlich. dim. $K$-VR mit $\dim V = \dim W, \; \\\varphi: V \rightarrow W$ lin. Abb. 

Dann sind folgende Aussagen äquivalent:
\begin{enumerate}[(i)]
	\item $\varphi$ ist surjektiv
	\item $\varphi$ ist injektiv
	\item $\varphi$ ist bijektiv
\end{enumerate}

% % % 
\subsubsection*{Beweis}
$\dim V = \dim W = n$\\

Nach \ref{dimensionsformel} gilt:\\
$n = \dim(\ker(\varphi)) + \rg(\varphi)$

Also: $\underbrace{\rg(\varphi)=n}_{\mathclap{\varphi \text{ surjektiv}}} \Leftrightarrow \underbrace{\dim(\ker(\varphi))}_{\mathclap{\varphi \text{ injektiv (Satz~\ref{kern}})}} = 0\\
\Rightarrow$ Beh.
\qed

% % % 5.13
\subsection{Zusammenhang lin. Abb. und hom. LGS, Matrizen, Rang}
\begin{itemize}
	\item homogenes LGS: $A \in M_{m,n}(K)$ gesucht: \\
	Menge aller $x \in K^n$ mit $Ax = \zerovec$
	\item lin. Abb. dazu:\\
	 $\varphi: K^n \rightarrow K^m, x \mapsto Ax$
\end{itemize}
Dann ist der Lösungsraum des homogenen LGS = $\ker(\varphi)$