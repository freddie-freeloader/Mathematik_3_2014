\subsection{Bemerkung} \label{chinsatzeindeutig}
Man kann auch zeigen, dass die L"osung $x$ aus Satz \ref{chin.restsatz} eindeutig ist:

Durch 
$\begin{array}[t]{lr@{\;}c@{\;}l}
\psi:&	\Z_M &\to&  Z_{m_1} \times \Z_{m_2} \times \dots \times \Z_{m_n} \\
&	x	&\mapsto& (x \mod{m_1}, \dots, x \mod{m_n})
\end{array}$

wird ein Ringisomophismus definiert:

$\psi$ ist surjektiv (zu jedem n-Tupel aus $\Z_{m_1}\times \dots \times \Z_{m_n}$ gibt es eine L"osung $x$, siehe Restsatz) und es gilt:

$\underbrace{|\Z_M|}_{M}=\underbrace{|\Z_{m_1}\times \dots \times \Z_{m_n}|}_{m_1\cdot \dots \cdot m_n = M}$

also ist $\psi$ bijektiv, also auch injektiv, also ist L"osung $x$ eindeutig.

\subsection[Korollar: Phi-Funktion berechnen]{Korollar}
$M=m_1\cdot \dots \cdot m_n$, $m_i$ paarweise teilerfremd.\\
Dann ist $\varphi(M)=\varphi(m_1)\cdot \dots \varphi(m_n),$ insbesondere:

$n=p^{a_1}_1 \cdot \dots \cdot p^{a_k}_k$ ($p_i$ Primzahlen, $a_1>0, p_i \neq p_j$ f"ur $i\neq j$)

\subsubsection*{Beweis}

Nach \ref{chinsatzeindeutig} ist $\Z_M \cong \Z_{m_1} \times \dots \times \Z_{m_n}$ mittels $\psi$\\
$\Rightarrow x$ Einheit $\Leftrightarrow \psi(x) = ( x\mod{m_1}, \dots, x \mod{m_n})$ Einheit
\\ $\Leftrightarrow x \mod{m_i}$ Einheit $\forall i = 1 \dots n$\\
$\Rightarrow \varphi(M) = \varphi(m_1) \cdot \dots \cdot \varphi(m_n)$\\
$\varphi(p^a)\underbrace{=}_{\text{"Uberlegen}}p^a - p^{a-1} = p^{a-1}(p-1)$

\subsection[Definition: Polynom]{Definition}
Sei $K$ K"orper mit Nullelement $0$ und Einselement 1:
{\renewcommand{\labelenumi}{(\roman{enumi})}\begin{enumerate}
	\item
	Ein \emph{Polynom "uber K} ist Ausdruck $f=a_0x^0+a_1x^1+\dots + a_nx^n$, $n\in \N_0, a_i\in K$.\\
	$a_i$ heißen \emph{Koeffizienten} des Polynoms.
	\begin{enumerate}
		\item
		Ist $a_i=0$, so kann man $0 \cdot x^i$ bei der Beschreibung weglassen.
		\item
		Statt $a_0x^0$ schreibt auch $a_0$
		\item
		Sind alle $a_i=0$, so schreibt man $f=0$, das Nullpolynom.
		\item
		Ist $a_i =1$, so schreibt man $x^i$ statt $1 \cdot x^i$
		\item
		Die Reihenfolge der $a_ix^i$ kann ver"andert werden, ohne dass das Polynom sich ver"andert ($x^4+2x^3 + 3= 2x^3 +  3 + x^4$)
	\end{enumerate} 
	\item
	Zwei Polynome $f$ und $g$ sind \emph{gleich}, wenn ($f=0$ und $g=0$) oder (${f=a_0 + a_1x^1+ \dots + a_nx^n},\\ {g=b_0+b_1x^1+ \dots + b_mx^m}, {a_n \neq  0, b_m \neq 0}$ und $n=m$, ${a_i = b_i}$ f"ur $i=0, \dots , n$) gilt.
	\item
	Die	Menge aller Polynome "uber $K$ bezeichnet man als $K[x]$
\end{enumerate} }

\subsection{Beispiel}
\begin{enumerate}
	\item
	$\underbrace{f}_{f(x)} = 3x^2 + \frac{1}{2}x -1 \in \Q [x] \land f \in \R [x]$
	\item
	$g= x^3 + x^2 +1 \in \Z_2[x] $ 
\end{enumerate}
Wir wollen in $K[x]$ wie in einem Ring rechnen k"onnen. Wir brauchen dazu $+$ und $\cdot$ f"ur Polynome.

\subsection[Satz und Definition: Polynomring]{Satz und Definition}
$K$ K"orper, dann wird $K[x]$ zu einem kommutativen Ring mit Eins durch folgende Verkn"upfungen:\\
$f= \underbrace{\sum_{i=0}^{n} a_ix^i}_{\text{z.B. }x+2}, \;\;\;g=\underbrace{\sum_{j=0}^{m} b_j x^j}_{x^3 + 2x + 1}$,

dann \[f + g = \underbrace{\sum_{i=0}^{\max(m,n)}(a_i+b_i)x^i}_{x^3 + 3x +3}\]

\[f \cdot g = {\sum_{i=0}^{n+m} c_ix^i}\] 
\[\text{mit } c_i= a_0b_i+a_1b_{i-1}+ \dots + a_ib_0 = \sum_{j=0}^{i} a_jb_{i-j}  \tag{Faltungsprodukt}\]
(setze $a_i$ mit $i>n$ bzw. $b_j$ mit $j>m$ gleich 0)
\begin{itemize}
	\item
	Einselement: $f=1 \;(a_0=1, a_j = 0 \;\;$f"ur $ j\geq 1)$
	\item
	Nullelement: $f= 0$
\end{itemize}
$K[x]$ heißt der \emph{Polynomring} in einer Variablen "uber $K$.\\
Beweis: Ringeigenschaften nachrechnen.

\subsection{Bemerkung}
Die +-Zeichen in der Beschreibung der Polynome entsprechen der Ring-Addition der \emph{Monome} $a_0, ax, a_2x^2, \dots, a_nx^n$

\subsection{Beispiel}

\begin{enumerate}
	\item 
	in $\Q[x], \R[x]$ Addition, Multiplikation klar
	
	\item
	in $\Z_3[x]$:
	$f = 2x^3 + 2x + 1$,
	$g = 2x^3 + x$
	
	$\begin{array}{lcl}
	f + g &=& x^3 + 1 \\
	f \cdot g &=& (2x^3 + 2x + 1)(2x^3 + x) \\
	&=& x^6 + 2x^4 + x^4 + 2x^2 + 2x^3 + x \\
	&=& x^6 + 2x^3 + 2x^2 + x
	\end{array}$
	
	\item
	in $\Z_2[x]$:
	$f = x^2 + 1$,
	$g = x + 1$
	
	$\begin{array}{lcl}
	f+g &=& x^2 + x \\
	f+f &=& 0 \\
	g \cdot g &=& x^2 + 1
	\end{array}$
	
	
\end{enumerate}


\subsection[Definition: Grad eines Polynoms]{Definition}

Sei $0 \neq f \in K[x]$

$f=a_0 + a_1x + \dots + a_nx^n$ mit $a_n \neq 0$

Dann heißt $n$ der \emph{Grad} von $f$ Grad($f$)

Grad($0$) $:= - \infty$ \\
Grad($f$) $=0$ \quad \emph{konstante Polynome $\neq 0$}
















