% % % 2.20
\subsection{Satz}
\label{sub:satz}

$K$ Körper, $f,g \in K[x]$

Dann ist $\Grad(f\cdot g) = \Grad(f) + \Grad(g)$

(Konvention: $-\infty + (-\infty) = -\infty + n = -\infty$)

\subsubsection*{Beweis}

	
	Stimmt für $f = 0$ oder $g = 0$
	
	\begin{align*}
		f &= a_0 + a_1x^1 + \ldots +  a_nx^n \quad &\text{mit} \quad  a_n \neq 0\\
		g &= b_0 + b_1x^1 + \ldots +  b_mx^m \quad &\text{mit} \quad  b_m \neq 0\\
		f \cdot g &= (\ldots) \cdot (\ldots) = \ldots + \underbrace{(a_nb_n)}_ {\mathclap{\substack{\neq 0, \\ \text{(siehe Satz 2.7
						Nullteilerfreiheit in Körpern)}}}}\cdot x^{n+m}
	\end{align*}
	
	
	Höhere Potenzen mit Koeffizienten $\neq 0$ gibt es nicht
	
	$\Rightarrow \Grad(f \cdot g) = n + m$

% % % 2.21
\subsection{Korollar}

$K$ Körper, dann
$K[x]^* = \{f \in K[x]\; |\; \Grad(f) = 0\}$,

d.h. nur die konstanten Polynome $\neq 0$ sind in $K[x]$ bez"uglich $\cdot$ invertierbar.

\[\underbrace{f}_{\mathclap{\Grad \;n}} \cdot \underbrace{f^{-1}}_{\text{müsste Grad $-n$ haben}} = \underbrace{1}_{\mathclap{\Grad \;0}} \leftarrow \text{geht nicht}\]

% % % 2.22
\subsection{Definition}


Sei $b \in K$

 $\varphi_b: K[x] \rightarrow K$, 
$f := \sum\limits_{i = 0}^n a_ix^i \mapsto f(b) := \sum\limits_{i = 0}^n a_ib^i$

ist ein surjektiver Ringhomomorphismus, der sogenannte \emph{Auswertungshomomorphismus} an der Stelle $b$.

(setze $b$ für $x$ ein)


% % % 2.23
\subsection{Definition}

$K$ Körper, $f, g \in K[x]$

$f$ \underline{teilt} $g$, $f|g$, falls ein $q \in K[x]$ existiert mit $g = q \cdot f$

(Nach \ref{sub:satz} ist dann $\Grad(f) \leq \Grad(g)$, falls $g \neq 0$)

% % % 2.24
\subsection{Definition (Division mit Rest)}

$K$ Körper, $0 \neq f \in K[x],\; g \in K[x]$

Dann existieren eindeutig bestimmte Polynome $q, r \in K[x]$ mit
$g = q \cdot f + r \; \text{und} \;\\ \Grad(r) < \Grad(f)$.

Bezeichnung:
\begin{align*}
	 r &=: g \mod f\\
	 q &=: g\; \text{div}\;  f
\end{align*}

\subsubsection*{Beweis}
Vgl. Mathe I für $\mathbb{Z}$, siehe z.B. WHK Satz 4.69

% % % 2.25
\subsection{Beispiel}
\label{sub:2.25}

\begin{enumerate}[a)]
	\item
	\begin{align*}
	g &= x^4+2x^3-x+2 \quad \in \mathbb{Q}[x]\\
	f &= 3x^2-1 \quad \in \mathbb{Q}[x]
	\end{align*}
	
	Rechne:
		
	\polylongdiv[style=C, div=:]{x^4+2x^3+0x^2-x+2}{3x^2-1}
	\item
	\[g = x^4+x^2+1 \quad \quad f=x^2+x \quad \in \mathbb{Z}_2[x]\]
	Rechne:
	\[(x^4+x^2+1) : x^2+x= \underbrace{x^2+x}_{q}\]
	
	
\end{enumerate}


% % % 2.26
\subsection{Korollar}

$K$ Körper, $a \in K$

$f \in K[x]$ ist genau dann durch $(x - a)$ teilbar, wenn $f(a) = 0$ ist (d.h. $a$ ist \underline{Nullstelle} von $f$).

\subsubsection*{Beweis}

"$\Rightarrow$" $\quad$ sei $f$ durch $(x - a)$ teilbar, d.h.
\[f = q \cdot (x - a) \Rightarrow f(a) = q(a) \cdot (\underbrace{a-a}_{0}) = 0 \quad q \in K\]

"$\Leftarrow$" $\quad$ Division mit Rest: $f = q(x - a) + r$, wobei
$\Grad(r) < \underbrace{\Grad(x-a)}_{1}$

$\Rightarrow r$ ist konstantes Polynom ($\Grad \; 0$) oder Nullpolynom $(\Grad \;(-\infty)$) also $r \in K$

\[0 = f(a) = q(a) \cdot 0 + r \Rightarrow r = 0 \quad \]
\qed

% % % 2.27
\subsection{Definition}

$K$ Körper

\begin{enumerate}[(i)]
	\item
	Ein Polynom dessen höchster von $0$ verschiedener Koeffizient gleich $1$ ist, heißt normiert.
	
	\item
	$g, h \in K[x]$, nicht beide $0$ 
	
	$f \in K[x]$ heißt \emph{größter gemeinsamer Teiler} von $g$ und $h$ ($f=\ggT(g,h)$), falls
	$f$ normiertes Polynom von maximalem Grad ist, das $g$ und $h$ teilt.
	
	\item
	$g, h \in K[x] \;\backslash\; \{0\}$ beide nicht $0$
	
	$f \in K[x]$ heißt \emph{kleinstes gemeinsames Vielfaches} von $g$ und $h$\\ ($f = kgV(g,h)$), falls
	$f$ normiertes Polynom von kleinstem Grad ist, das von $g$ und $h$ geteilt wird.
\end{enumerate}


% % % 2.28
\subsection{Bemerkung}

\begin{enumerate}[a)]
	\item
	$f = \sum\limits_{i = 0}^n a_ix^i, a_n \neq 0$, dann ist $a_n^{-1}f = x^n + \ldots\;$ normiertes Polynom.
	
	(z.B.: $f = 3x^2+x+7 \in \mathbb{R}[x]$)
	
	dann $\frac{1}{3}f = x^2+\frac{x}{3} + \frac{7}{3}$ normiert.
	
	In $\mathbb{Z}_{11}[x]: \underbrace{4}_{\mathclap{\text{Inverses von 3, denn $3\cdot 4 = 12 \equiv 1 \pmod{11}$}}}f = x^2+4x_6$ normiert.
	
	\item
	$\kgV(g,h)$ existiert und ist eindeutig:
	
	\begin{align*}
	\text{sei}\; f_1 &= \kgV(g,h), \; f_2 = \kgV(g,h)\\
	&\Rightarrow g,h | f_1, \quad g,h | f_2\\
	&\Rightarrow g,h | (f_1-f_2)
	\end{align*}
	
	\item
	$\ggT(g,h)$ existiert. Beweis Eindeutigkeit wie in $\mathbb{Z}$ (Mathe I), folgt aus.
\end{enumerate}


% % % 2.29
\subsection{Satz (von Bezout)}
\label{sub:satz_von_bezout_}

$K$ Körper, $g, h \in K[x]$, nicht beide $0$.

Dann existieren $s, t \in K[x]$, sodass
\[f = s \cdot g + t\cdot h\]

ein $\ggT$ von $g$ und $h$ ist.

(Beweis: EEA in $K[x]$, später)
