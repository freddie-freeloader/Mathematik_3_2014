% % % 2.20
\subsection{TBD}
\subsection{Satz}
\label{sub:satz}

$K$ Körper, $f,g \in K[x]$

Dann ist $Grad(f\cdot g) = Grad(f) + Grad(g)$

(Konvention: $-\infty + (-\infty) = -\infty + n = -\infty$)

\par\bigskip
\underline{Beweis}

\begin{itemize}
	\item
	stimmt für $f = 0$ oder $g = 0$
	
	\item
	\[f = a_0 + a_1x^1 + \ldots +  a_nx^n \quad \text{mit} \quad  a_n \neq 0\]
	\[g = b_0 + b_1x^1 + \ldots +  b_mx^m \quad \text{mit} \quad  b_m \neq 0\]
	\[f \cdot g = (\cdots) \cdot (\cdots) = \cdots + \underbrace{(a_nb_n)}_ {\mathclap{\substack{\neq 0, \\ \text{(siehe Satz 2.7
					Nullteilerfreiheit in Körpern)}}}}\cdot x^{n+m}\]
	
	Höhere Potenzen mit Koeffizienten $\neq 0$ gibt es nicht
	
	$\Rightarrow Grad(f \cdot g) = n + m$
\end{itemize}

% % % 2.21
\subsection{TBD}

% % % 2.22
\subsection{TBD}

% % % 2.23
\subsection{TBD}

% % % 2.24
\subsection{TBD}

% % % 2.25
\subsection{TBD}

% % % 2.26
\subsection{TBD}

% % % 2.27
\subsection{TBD}

% % % 2.28
\subsection{TBD}

% % % 2.29
\subsection{TBD}