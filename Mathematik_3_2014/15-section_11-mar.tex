 % 11.1 - 11.27
 % % % % % % % % % % % % % % % % % % % % % % % % % % % % %
 
% % % %
% 11  %
% % % %
\section{Mehrdimensionale Analysis}
\hspace*{0.5cm}
\hspace*{1cm}
\emph{Siehe Blatt im Moodle (11.1 bis 11.12)}


% % % % % % % % %
% % % 11.13   % %
% % % % % % % % %
\setcounter{subsection}{12}
\subsection{Beispiel}
\begin{enumerate}
	\item
	$f:\mathbb{R}^2\rightarrow\mathbb{R}$\\
	$f(x,y)=e^x+y^2$\\
	$\underbrace{f'(x,y}_{\text{Jacobimatrix}}=\begin{pmatrix}\frac{\partial f}{\partial x}(x,y) & \frac{\partial f}{\partial y}(x,y)\end{pmatrix}= \begin{pmatrix}e^x & 2y\end{pmatrix}$\\
	$f'(0,0)=\begin{pmatrix}1 & 0\end{pmatrix}$\\
	$\nabla f(x,y)=\begin{pmatrix}e^x \\ 2y\end{pmatrix}$
	
	\item
	$f:\mathbb{R}^3\rightarrow \mathbb{R}^2, \ \ f(x,y,z)=\begin{pmatrix}x+y \\ x\cdot y\cdot z\end{pmatrix}$\\
	$f'(x,y,z)=\begin{pmatrix}1 & 1 & 0 \\ yz & xz & xy\end{pmatrix}$\\
	$f'(2,0,1)=\begin{pmatrix}1 & 1 & 0 \\ 0 &  2 & 0 \end{pmatrix}$
\end{enumerate}

\hspace*{0.5cm}
\hspace*{1cm}
\emph{Siehe Blatt im Moodle (11.14 bis 11.21)}

\setcounter{subsection}{21}
% % % % % % % % %
% % % 11.22   % %
% % % % % % % % %
\subsection{Definition: (total)differenzierbar, affin-linear}
$D\subseteq \mathbb{R}^n$ offen, $a\in D$\\
$f:D\rightarrow \mathbb{R}^m$ heißt \textbf{(total)differenzierbar in a} wenn f in a partiell differenzierbar ist und geschrieben werden kann als $f(x)=f(a)+f'(a)(x-a)+\epsilon(x)$ mit $\epsilon: D\rightarrow \mathbb{R}^m$ mit $\lim\limits_{x\rightarrow a} \frac{||\epsilon(x)||}{||x-a||}=0$ (d.h. $\epsilon$ wird klein nahe a)\\
($n=m=1$, erhalte Definition der Differenzierbarkeit aus Mathe II)\\
Anschaulich:\\
f kann in der nähe von a durch die \textbf{affin-lineare} Abbildung $x\mapsto \underbrace{f(a)}_{\text{konst.}}+\underbrace{\underbrace{f'(a)}_{\text{Matrix}}(x-a)}_{\text{linear}}$ beschrieben werden

% % % % % % % % %
% % % 11.23   % %
% % % % % % % % %
\subsection{Definition: Richtungsableitung}
$D\subseteq \mathbb{R}^n$ offen, $f:D\rightarrow \mathbb{R}, \ a\in D$\\
$v\in \mathbb{R}^n$ mit $||v||=1$\\
f heißt \textbf{in a differenzierbar in Richtung v} wenn $\lim\limits_{h\rightarrow 0} \frac{f(a+hv)-f(a)}{h}$ ex.\\
Der Grenzwert heißt dann \textbf{Richtungsabtleitung} von f in Richtung v im Punkt a.\\
Bez.: $\frac{\partial f}{\partial v}(a)$

% % % % % % % % %
% % % 11.24   % %
% % % % % % % % %
\subsection{Bemerkung}
$\frac{\partial f}{\partial x_j}$ ist die Richtungsableitung von f in Richtung $e_j=\begin{pmatrix}0 \\ \vdots \\ 0 \\ 1 \\ 0 \\ \vdots \\ 0\end{pmatrix}$ (1 an Stelle j)

% % % % % % % % %
% % % 11.25   % %
% % % % % % % % %
\subsection{Satz}
sei $f: D\subseteq \mathbb{R}^n\rightarrow \mathbb{R}$ (total)differenzeirbar in $a\in D$\\
Dann ex. in a alle Richtungsableitungen und für alle $v\in\mathbb{R}^n$ mit $||v||=1$ gilt:
\[\frac{\partial f}{\partial v}(a)=f'(a)\cdot v\]
Die Richtungsableitug stellt den Anstieg von f an der Stelle a in Richtung v dar.

\textbf{Beweis:}\\
f differenzierbar, d.h. $f(x)=f(a)+f'(a)\cdot (x-a)+\epsilon(x)$\\
$\stackrel{x=a+h\cdot v}{\Rightarrow} f(a+h\cdot v)=f(a)+f\cdot (a)\cdot (a+h\cdot v-a)+\epsilon(a+h\cdot v)$\\
$\stackrel{h\neq 0}{\Rightarrow} \frac{f(a+hv-f(a)}{h}=\frac{f'(a)\cdot (hv)}{h} + \frac{\epsilon (a+hv)}{h}$\\
$\frac{\partial f}{\partial v}=\lim\limits_{h\rightarrow 0}\frac{f(a+hv)-f(a)}{h}=f'(a)\cdot v$\\
\hspace*{13cm}$\square$

% % % % % % % % %
% % % 11.26   % %
% % % % % % % % %
\subsection{Beispiel}
\begin{enumerate}
	\item
	Anstieg von $f(x,y)=x^2+y^2$ im Punkt $a=\begin{pmatrix}1 \\ 1 \end{pmatrix}$ in Richtung $v=\begin{pmatrix}\sin \alpha \\ \cos \alpha\end{pmatrix} \ \ (||v||=1)$\\
	$\frac{\partial f}{\partial v}(1,1)=f'(1,1)\cdot \begin{pmatrix}\sin \alpha \\ \cos \alpha\end{pmatrix}=\begin{pmatrix}2 & 2\end{pmatrix}\cdot \begin{pmatrix}\sin \alpha \\ \cos \alpha	\end{pmatrix}=2\cdot \sin \alpha+ 2\cdot \cos\alpha$
	
	\item
	$f(x,y)=2x^2+y^2, \ \ a=\begin{pmatrix}\frac{1}{2} \\ \frac{1}{2} \\ \frac{3}{4}\end{pmatrix}$ Punkt auf ''Gebirge''\\
	\textbf{gesucht:} Richtung $v=\begin{pmatrix}v_1 \\ v_2\end{pmatrix}$, in der die Tangente an den Graph von f die Steigung $\frac{3}{\sqrt{2}}$ hat.\\
	$\frac{\partial f}{\partial v}\left(\frac{1}{2},\frac{1}{2}\right)=f'\left(\frac{1}{2},\frac{1}{2}\right) = \begin{pmatrix}2 & 1\end{pmatrix}\begin{pmatrix}v_1 \\ v_2\end{pmatrix}=2v_1+v_2 \stackrel{!}{=} \frac{3}{\sqrt{2}}$ und $||v||=1$, d.h. $v_1^2+v_2^2=1$\\
	Gleichungssystem lösen...\\
	ergibt $v=\begin{pmatrix}\frac{1}{\sqrt{2}} \\ \frac{1}{\sqrt{2}}\end{pmatrix}$ und $v=\begin{pmatrix}\frac{7}{5\sqrt{2}} \\ \frac{1}{5 \sqrt{2}}\end{pmatrix}$
\end{enumerate}

% % % % % % % % %
% % % 11.27   % %
% % % % % % % % %
\subsection{Bemerkung}
Es gilt: $\frac{\partial f}{\partial v}(a)=f'(a)\cdot v=(\nabla f)^T\cdot v = ||\nabla f(a)||\cdot ||v||\cdot \cos \alpha$, $\alpha$: Winkel zw. $\nabla f$ und v\\
$\Rightarrow$ Richtungsableitung: $\frac{\partial f}{\partial v}(a)$ ist am größten, wenn $\cos \alpha = 1$, also $\alpha=0$  ist, d.h. wenn der Richtungsvektor v in Richtung des Gradienten zeigt.\\
Der Gradient zeigt also immer in die Richtung des steilsten Anstiegs der Funktion.
