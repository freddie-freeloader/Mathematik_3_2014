 %1.20 - 1.24
 % % % % % % % % % % % % % % % % % % % % % % % % % % % % %
 
 \subsection{Beispiel}
 $
 \aMatrix{ccc}{
 	1 & 2 & 3 \\
 	2 & 1 & 3
 	}\circ
 	x =
 \aMatrix{ccc}{
 	1 & 2 & 3 \\
 	3 & 1 & 2
 	}
 	$
 - Was ist $x$?
 
 $ a \cdot x = b \Leftrightarrow x=a^{-1} \cdot b$
 
 $
 x =
  \aMatrix{ccc}{
  	1 & 2 & 3 \\
  	2 & 1 & 3
  }^{-1}\circ
 \aMatrix{ccc}{
 	1 & 2 & 3 \\
 	3 & 1 & 2
 } = 
  \aMatrix{ccc}{
  	1 & 2 & 3 \\
  	2 & 1 & 3
  }\circ
  \aMatrix{ccc}{
  	1 & 2 & 3 \\
  	3 & 1 & 2
  } =
  \aMatrix{ccc}{
  	1 & 2 & 3 \\
  	3 & 2 & 1
  }
  $

\subsection[Definition: Untergruppe]{Definition}

$(G,\cdot)$ Gruppe, $\varnothing \neq U \subseteq G$ Teilmenge.

$U$ heißt \emph{Untergruppe} von $G$ ($U\leqslant G$), falls $U$ bzgl. $\cdot$ selbst eine Gruppe ist.

Insbesondere gilt dann:
$\forall u,v \in U$ ist $u \cdot v \in  U$.\\
$e$ von $G$ ist auch neutrales Element in $U$. (*)\\
Inversen in $U$ sind die gleichen wie in $G$.

(*) Angenommen $e$ ist neutrales Element in $G$, aber $f$ neutrales Element in $U$, $f^{-1}$ Inverses von $f$ in $G$.\\
Dann ist $f^{-1} \cdot f= f \cdot f^{-1} = e$ und $f\cdot f = f$.

$\Rightarrow f = e \cdot f = (f^{-1} \cdot f) \cdot f = f^{-1} \cdot (f \cdot f) = f^{-1} \cdot f = e$

\subsection{Beispiele}
\begin{enumerate}
	\item
	$(\Z, +) \leqslant (\Q, +) \leqslant (\R, +)$
	
	\item
	$(\{-1,1\}, \cdot) \leqslant (\Q \backslash \{0\}, \cdot) \leqslant (\R \backslash \{0\}, \cdot)$
	
	\item
	$(e, \cdot)$ ist Untergruppe jeder beliebigen Gruppe mit Verkn"upfung $\cdot$ und neutralem Element $e$.
	
	\item
	$\pi =
	\aMatrix{ccc}{
		1 & 2 & 3 \\
		2 & 1 & 3
		}
	\in S_3
	$,
	$\pi^{-1} = \pi, \pi^{-1} \circ \pi = \text{id} =
	\aMatrix{ccc}{
		1 & 2 & 3 \\
		1 & 2 & 3
		}
	$
	
	$\Rightarrow (\pi, \text{id})\leqslant S_3$
	
\end{enumerate}
	
	\subsection[Satz und Definition: Rechtsnebenklassen]{Satz und Definition}
	$G$ Gruppe, $U \leqslant G$
	{\renewcommand{\labelenumi}{(\roman{enumi})}
	\begin{enumerate}
		\item
		{Durch} $x \sim y \Leftrightarrow \mathunderset{x \cdot y^{-1} \in U}{x + (-y) \in U}$ \textunderset{}{(bei additiver Verknüpfung)}\\
		wird auf $G$ eine "Aquivalenzrelation definiert
		
		\subsubsection*{Beweis}
		
		$\sim$ ist reflexiv: $x \sim x$ gilt $\forall x \in G$, denn $x \cdot x^{-1}= e \in U \; \checkmark$
		
		$\sim$ ist symmetrisch: $x \sim y \Rightarrow y \sim x$\\
		 Sei $x \sim y$, also $x \cdot y^{-1} \in U$ (zzg.: $y \sim x$, also $y \cdot x^{-1} \in U$) \\dann ist $y \cdot x^{-1} = (x \cdot y^{-1})^{-1} \in U$, da auch $ x \cdot y^{-1} \in U$.
		 
		$\sim$ ist transitiv: $x \sim y, y \sim z \Rightarrow x \sim z$\\
		Sei $x \sim y$, also $x \cdot y^{-1} \in U$ und $y \sim z$, also $y \cdot z^{-1} \in U$ (zzg.: $x \sim z$, d.h. $x\cdot z^{-1} \in U$)
		
		$x \cdot z^{-1} = xez^{-1} = x(y^{-1}y)z^{-1} = (\underbrace{x \cdot y^{-1}}_{\in U}) \cdot (\underbrace{y \cdot z^{-1}}_{\in U}) \in U$, also $x \sim z$.
		\qed
		
		
		\item
		F"ur $x \in G$ ist $Ux = \{u \cdot x \;|\; u \in U\}$ die "Aquivalenzklasse von $x$ bzgl. $\sim$ und heißt \emph{Rechtsnebenklasse} von $U$ in $G$.
		
		Also (Eigenschaften von "Aquivalenzklassen siehe Mathe I):
		\begin{enumerate}
			\item
			$Ux = Uy \Leftrightarrow x \sim y$, also $x \cdot y^{-1} \in U$
			\item
			$x,y \in G$, dann ist entweder $Ux = Uy$ oder $Ux \cap Uy = \varnothing $
		\end{enumerate}
		
		\subsubsection*{Beweis}
		\begin{enumerate}
			\item
			Seit $x \sim y \Rightarrow y \sim x \Rightarrow y \cdot x^{-1} \in U \Rightarrow y=y(x^{-1} \cdot x) = \underbrace{(y \cdot x^{-1})}_{\in U}x \in Ux$
			\item
			Sei $y \in Ux$, dann zeige: $x \sim y$ \\
			$y \in Ux \Rightarrow y = u \cdot x$ f"ur ein $u \in U\\
			\Rightarrow x \cdot y^{-1} = x \cdot (ux)^{-1} = x \cdot x^{-1} \cdot u^{-1} = u^{-1} \in U$\\
			Es wurde gezeigt, dass $x \sim y$ gilt.
		\end{enumerate} \qed

\end{enumerate}}

\subsection{Beispiel}

$G = (\Z, +), 3 \Z = \{\dots, -3,0,3,6,\dots\}\\
U = (3 \Z, +) \leqslant G$ ("UA, Blatt 2)\\
Inverses zu y in $(\Z,+)$ ist $-y$.\\
$x \sim y \Leftrightarrow \underset{\text{bzw.:}\, x-y \in U}{x \cdot y ^{-1} \in U}$ 


$ x=0 : U+0 = \{u+0 \;|\; u \in U\}= \{\dots, -3,0,3,6,\dots\} = U = 3\Z$
   
$x=1 : U+1 = \{u+1 \;|\; u \in U\} = \{\dots, -2, 1, 4, 7, 10, \dots \} = 3\Z +1$

$x=2 : U+2 = \aset{u + 2 \;|\; u \in U} = \aset{\dots, -1, 2, 5, 8, 11, \dots} = 3 \Z + 2$

$x=3: U+3 = U+0 = 0$

\dots
   
   
   
   
   
   
   
   
   
   
   
   
   
   
   
   
   