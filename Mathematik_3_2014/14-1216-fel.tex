
 
 \begin{enumerate}
 	\item
 	\item
 	Entwicklung nach der j-ten Spalte:
 	
 	für $i \in \aset{1, \dots, n}$ gilt:
 	%
 	\[\det(A) = \sum_{j=1}^{n}(-1)^{i+j}a_{ij}\det(A_{ij})\]
 	%
 	$(-1)^{i+j} \rightsquigarrow \aMatrix{cccccc}{
 	+ & - & + & - & + & \dots \\
 	- & + & - & + & \dots \\
 	+ & - & + & \dots \\
 	\dots}$
 	
 	
 \end{enumerate}
 
 \subsection{Beispiel}
 
 
 $A = \aMatrix{ccc}{
 2 	& 1	& 1 \\
 -1	& 0 & 3 \\
 2	& 0 & 4}  \in M_3(\R)$


Mit Definition 7.2 %TODO Label
(Entwicklung nach der 1. Zeile):

$\det(A) = 2 \cdot \det \aMatrix{cc}{0 & 3 \\ 0 & 4} 
	- 1 \cdot \det \aMatrix{cc}{-1 & 3 \\ 2 & 4}
	+ 1 \cdot \det \aMatrix{cc}{-1 & 0 \\ 2 & 0}
  = 2 \cdot 0 - 1 \cdot (-10) + 1 \cdot 0 
  = 10$ 
	
oder: Entwicklung nach der 3. Zeile:

$\det(A) = 2 \cdot \aMatrix{cc}{1 & 1 \\ 0 & 3} 
	- 0 \cdot \det (\dots) 
	+ 4 \cdot \det \aMatrix{cc}{2 & 1 \\ -1 & 0}
  = 2 \cdot 3 - 0 + 4 \cdot 1 
  = 10$ 
  
oder (besser): Entwicklung nach der 2. Spalte:

$\det(A) = -1 \cdot \det \aMatrix{cc}{-1 & 3 \\ 2 & 4}
	+ 0 \cdot \det (\dots)
	- 0 \cdot \det (\dots)
  = -1 \cdot (-10)
  = 10$
  
Also: Es ist geschickt, nach einer Spalte oder Zeile zu entwickeln, in der viele Nullen stehen.

Falls es wenig Nullen gibt: Zuerst Gauß anwenden (Auchtung: det. ändert sich dabei eventuell!)

\subsection{Bemerkung}

Aus 7.4 %TODO Label
folgt $\det(A)=\det(A^T)$


\subsection{Satz (Eigenschaften der Determinanten)}

$A, B \in M_n(K), s_1, \dots, s_n$ Spalten von $A$,
$s_i' \in K^n, \lambda \in K$

\begin{enumerate}
	\item[(D1)]
	$\det(s_1, \dots, \underbrace{s_i+s_i'}_{i}, \dots, s_n)
		= \det(s_1, \dots, s_i, \dots, s_n) 
			+ \det(s_1, \dots, s_i', \dots, s_n)$
			
	\item[(D2)]
	Beim Vertauschen zweier Spalten von $A$ ändert sich das Vorzeichen von $\det(A)$
	
	\item[(D3)]
	$\det(s_1, \dots, \underbrace{\lambda \cdot s_i}_{i}, \dots, s_n) = \lambda \cdot \det(s_1, \dots, s_i, \dots, s_n)$
	
	(Beweis D1-D3 folgt aus 7.2 \& 7.4) %TODO Label
	
	\item[(D4)]
	$\det(\lambda \cdot A) = \det(\lambda s_1, \dots, \lambda s_n) \stackrel{(D3)}{=} \lambda^n \cdot \det(A)$
	
	\item[(D5)]
	Ist eine Spalte von $A$ gleich $\zerovec$, so ist $\det(A) = 0$
	
	\item[(D6)]
	Besitzt $A$ zwei identische Spalten, so ist $\det(A) = 0$
	\\ (Vertausche identische Spalten, erhalte Matrix $A' = A$.
	Nach (D2): $\det(A)=-\det(A')=-\det(A)$.
	Dies ist nur möglich, falls $\det(A)=0$ (oder in Körper mit $1+1=0$. Hier anders beweisen!))
	
	\item[(D7)]
	$\det(s_1, \dots, \underbrace{s_i + \lambda s_j}_i, \dots, s_n) = \det(A) \quad (i \neq j)$
	\\	mit D1, D3, D6
	
	\item[(D8)]
	$\det(A \cdot B) = \det(A)\cdot\det(B)$
	
	
\end{enumerate} 

Analog mit Zeilen statt Spalten!

Vorsicht: Im Allgemeinen ist $\det(A+B) \neq \det(A)+ \det(B)$


\subsection{Bemerkung / Beispiel}

Also: Erzeuge mit Gaußelimination viele Nulleinträge (! D2, D3: det ändert sich)
 
D7: det bleibt, entwickele nach guter Zeile / Spalte, oder bringe Matrix auf obere/untere $\triangle$-Form

z.B. 
\\$\det\aMatrix{ccc}{
0 & 1 & 2 \\
-2& 0 & 3 \\
2 & -2& 3}
	\stackrel{z_1\leftrightarrow z_2}{=}
	- \det\aMatrix{ccc}{
-2 & 0 & 3 \\
0  & 1 & 2 \\
0  & -2& 3}
	\stackrel{III=2\cdot II+III}{=}
	- \det \aMatrix{ccc}{
-2 & 0 & 3 \\
0  & 1 & 2 \\
0  & 0 & 7}
	= 
	- (-2) \cdot 1 \cdot 7 = 14$
 
 
\subsection{Satz (Charakterisietung invertierbarer Matrizen über det)}

$A \in M_n(K)$ ist invertierbar $\gdw \det(A) = 0$

In diesem Fall gilt:
\[\det(A^{-1}) = (\det(A))^{-1} \]

\subsubsection*{Beweis} 
\begin{itemize}
	\item[''$\Rightarrow$'':]
	
	Sei $A$ invertierbar, d.h. $\exists A^{-1}$ mit $A\cdot A^{-1} = A^{-1}\cdot A = E_n$
	
	$\Rightarrow \underbrace{\det(A\cdot A^{-1})}_{(D8): \det(A) \cdot \det(A^{-1})} = \det(E_n) = 1$
	
	 $\Rightarrow \det(A) \neq 0$ und $\det(A^{-1}) = (\det(A))^{-1}$
	 
	 \item[''$\Leftarrow$'':]
	 
	 Sei $A$ nicht invertierbar.
	 
	 $\Rightarrow \rg(A) < n
	 \\ \stackrel{6.8}{\Rightarrow}$ Spalten von $A$ sind l.a. %TODO Label
	 \\ d.h. $\exists i$ mit $s_i = \sum_{k=1\\k\neq i}^{n}\lambda_k s_k$ ($s_i$ als LK der anderen Spalten)
	 
	 $\det(A) \stackrel{(D7)}{=} \det(s_1, \dots, s_i- \sum\lambda_ks_k, \dots, s_n)
	 \\ = \det(s_1, \dots, \zerovec, \dots, s_n) 
	 \stackrel{(D5)}{=} 0$
	
	
\end{itemize} 

\subsection{Bemerkung}

Berechnung von $A^{-1}$ mittels 6.9 (Gauß mit $(A\;|\;E_n))$ oder auch mittels det, vgl. Übungsblatt 10, A1

$A = \aMatrix{cc}{a & b \\ c & d} \in M_2(\R) \Rightarrow A^{-1} = \frac{1}{\det A} \cdot \aMatrix{cc}{d & - b \\ - c & a}$



\section{Eigenwerte und Eigenvektoren}

\subsection[Definition (Eigenwert)]{Definition}

Sei $A \in M_n(K)$.

Ein Skalar $\lambda \in K$ heißt \emph{Eigenwert} von $A$, wenn es einen Vektor $\zerovec \neq x \in K^n$ gibt (''nichttrivial'', d.h. $\neq 0$) mit
%
\[A\cdot x = \lambda x\]
 
 
 
 
 
 
 
 
 
 
 
 
 
 
 
 
 