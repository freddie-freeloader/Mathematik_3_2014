 %1.17 - 1.19
 % % % % % % % % % % % % % % % % % % % % % % % % % % % % %

% % % 1.17
\subsection{Beispiele}

\begin{enumerate}
	
	\item
	$(\N_0, +, 0)$ ist keine Gruppe aber $(\Z, +, 0), (\Q, +, 0), (\R, +, 0)$ sind Gruppen.
	
	\item
	$(\Z, \bigdot, 1)$ ist keine Gruppe.
	\\ Die Menge der invertierbaren Elemente ist $\aset{1, -1}$, diese bilden eine Gruppe.
	
	\item
	$(\Q, \bigdot, 1)$ ist keine Gruppe, aber $(\Q \without{0}, \bigdot, 1), (\R \without{0}, \bigdot, 1)$ sind Gruppen.
	
	\item
	$A^*$ ist keine Gruppe, nur $\lambda$ ist invertierbar.
	
\end{enumerate}

% % % 1.18
\subsection{Beispiele}
\label{gruppenbeispiele}
\begin{enumerate}
	
	\item
	$(\Z_n, \oplus, 0)$ ist Gruppe (was ist das Inverse zu $x \in \Z_n$? Siehe PÜ1, A9)
	
	\item
	Sei $n \geq 2$. $(Z_n, \odot, 1)$ ist Monoid aber keine Gruppe. 
	
	Wann ist ein Element aus $Z_n$ invertierbar bezüglich $\odot$?
	
	$\begin{array}{lcl}
	z \in \Z_n \text{ invertierbar} 
		&\gdw&
		\exists x \in \Z_n : z \odot x = 1 \\
		&\gdw&
		 \exists x \in \Z: (z \cdot x) \mod{n} = 1 \\
		&\gdw&
		\exists x, q \in \Z: z \cdot x = q \cdot n + 1 \\
		&\gdw&
		\exists x, q \in \Z: z \cdot x + (-q \cdot n) = 1 \\
		&\stackrel{\text{Mathe I}}{\gdw}&
		\ggT(z, n) = 1 
	\end{array}$
	
	also sind nur zu $n$ teilerfremde Elemente invertierbar!
	
	(vgl. $(Z_6, 0, 1)$: $0, 2, 3, 4$ nicht invertierbar, $1, 5$ invertierbar)
	
	\underline{Bezeichnung:}
	\[Z_n^* = \aset{z \in \Z_n \;|\; \ggT(z, n) = 1}\]
	%
	ist Gruppe bezüglich $\odot$ (vgl. Bemerkung \ref{invbarkeitMonoid}) mit Ordnung $\abs{Z_n^*}=\varphi(n)$ (''phi von $n$'', Eulersche $\varphi$-Funktion) = Anzahl aller $z \in \N$, die teilerfremd zu $n$ sind und $1 \leq z \leq n$.
	
	$\varphi(3) = 2, \varphi(4)=2, \varphi(7) = 6$

	Wie berechnet man das Inverse von $z \in \Z_n^*$?
	
	Mathe I, Erweiterter Euklidischer Algorithmus (WHK, S. 80/81) liefert zu $z$ und $n$ $(\ggT(z, n) = 1)$ Zahlen $s, t \in \Z$ mit 
	
	$\begin{array}{cl}
	& z \cdot s + n \cdot t = 1 \\
	\Rightarrow & (z \cdot s) \mod n = 1 \\
	\Rightarrow & (z^{-1}) = s \mod n
	\end{array}$
	
	Beispiel:
	
	$n=8$: $(\Z_8, \odot)$, $z=5$ ist invertierbar, $\ggT(8, 5) = 1$
	
	EEA: $5 \cdot (-3) + 8 \cdot 2 = 1 \Rightarrow z^{-1} = -3 \mod 8 \Rightarrow z^{-1} = 5$


	\item
	$\Abb(M, M)$: invertierbare Elemente sind genau die \emph{bijektiven} Abbildungen auf $M, \mathrm{Bij}(M)$ (Mathe I)
	
	Speziell: $M = \aset{1, 2, \dots, n}$, dann heißt $\mathrm{Bij}(M)$ die symmetrische Gruppe von Grad $n$, $S_n$
	
	$\abs{S_n} = n!$, Elemente heißen Permutationen.
	
	\underline{Bsp:} $n=2$
	\[S_2= \aset{
	\underbrace{\aMatrix{cc}{
	1 & 2 \\ 
	1 & 2}}_{\id},
	\aMatrix{cc}{
	1 & 2 \\
	2 & 1}}\]
	%
	$n=3$
	%
	\[S_3 = \aset{
	\underbrace{\sPer{1}{2}{3}}_{\id},
	\sPer{1}{3}{2},
	\sPer{3}{2}{1},
	\sPer{2}{1}{3},
	\sPer{2}{3}{1},
	\sPer{3}{1}{2}
	}\]
	
	$\pi = \sPer{1}{3}{2}, \varrho = \sPer{3}{1}{2} \in S_3$
	
	$\pi \circ \varrho = \sPer{2}{1}{3},\; \varrho \circ \pi = \sPer{3}{2}{1}$ (nicht kommutativ!)
	
	$\pi^{-1} = \sPer{1}{3}{2} = \pi,\; \varrho^{-1}= \sPer{2}{3}{1}$
\end{enumerate}

% % % 1.19
\subsection[Satz: Gleichungen lösen in Gruppen]{Satz (Gleichungen lösen in Gruppen)} \label{gLösenGruppen}

	Sei $G$ Gruppe, $a, b \in G$
	
	{\renewcommand{\labelenumi}{(\roman{enumi})}
	\begin{enumerate}
		\item
		Es gibt genau ein $x \in G$ mit $ax = b$ (nämlich $x = a^{-1}b$)
		\item
		Es gibt genau ein $y \in G$ mit $ya = b$ (nämlich $y=ba^{-1}$)
		\item
		Ist $\underset{ya=yb}{ax=bx}$ für ein $\underset{y \in G}{x \in G}$, dann gilt $a=b$ (Kürzungsregel)
	\end{enumerate}
	
	\begin{proof}
	\begin{enumerate}
		\item
		\begin{itemize}
			\item
			$x=a^{-1}$ ist Lösung (prüfe $ax=b$):
			
			$a \cdot \underbrace{a^{-1}b}_{x} \stackrel{\text{AG}}{=}(a \cdot a^{-1}) \cdot b = e \cdot b = b$
			
			\item Es gibt genau eine Lösung:
			
			Es gelte $ax=b$
			\\ $\Rightarrow x = ex = (a^{-1}a)x\stackrel{\text{AG}}{=}a^{-1}(ax) = a^{-1}b$
		\end{itemize}
		
		\item
		analog
		
		\item
		Multipliziere von \textunderset{rechts}{links} mit $\underset{y^{-1}}{x^{-1}}$
	\end{enumerate}
	\end{proof}
	
	
	}
