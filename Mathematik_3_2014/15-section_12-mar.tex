 % 12.1 - 12.7
 % % % % % % % % % % % % % % % % % % % % % % % % % % % % %
 
% % % %
% 12  %
% % % %
\vspace*{0.5cm}
\hspace*{1cm}
\emph{Jetzt wieder 1-dimensionale Analysis:}
\section{Taylorpolynome und Taylorreihe}

% % % % % % % % %
% % % 12.1   % %
% % % % % % % % %
\subsection{Definition}
$I\subseteq \mathbb{R}$ Intervall, $x_0\in I, \ f:I\rightarrow\mathbb{R}$
\begin{enumerate}
	\item
	$f^{(0)}:=f$\\
	$f^{(1)}=f'$, falls f diffbar auf I\\
	$\vdots$\\
	$f^{(n)}=(f^{(n-1)})'$, falls $f^{(n-1)}$ diffbar auf I
	
	(f \textbf{n-mal differenzierbar}, $f^{(n)}$ \textbf{n-te Ableitung}
	
	\item
	f heißt \textbf{unendlich oft differenzierbar}, falls f n-mal diffbar $\forall n\in\mathbb{N}$.\\
	(Bez. auch $f^{(1)}=f',f^{(2)}=f'',\dots)$
\end{enumerate}

% % % % % % % % %
% % % 12.2    % %
% % % % % % % % %
\subsection{Beispiel}
\begin{enumerate}
	\item
	$f(x)=x^2 \ \ \infty$ oft diffbar\\
	$f'(x)=2x, \ f''(x)=2, \ f^{(n)}=0 \forall n\ge 3$
	
	\item
	$f(x)=e^x$\\
	$f^{(n)}(x)=e^x \forall n\in\mathbb{N}_0$
	
	\item
	$f(x)=\left\lbrace\begin{array}{ll}
	\frac{1}{2}x^2 & x\ge n\\
	-\frac{1}{2}x^2 & x< 0
	\end{array}\right.$\\
	$f'(x)=\left\lbrace\begin{array}{ll}
	x & x\ge 0\\
	-x & x<0
	\end{array}\right. = |x|$, nicht diffbar in $x=0$
	
	\item
	$f:\mathbb{R}^+\rightarrow\mathbb{R}^+, \ \ f(x)=x^\alpha \ \ (\alpha\in\mathbb{R})$\\
	$f'(x)=\alpha\cdot x^{\alpha-1}$\\
	$f^{(n)}=\alpha\cdot (\alpha-1)\cdot \dots\cdot (\alpha-n+1)\cdot x^{\alpha-n} = n!\underbrace{\begin{pmatrix}\alpha \\ n\end{pmatrix}}_{\text{binom.}}\cdot x^{\alpha-n} \ \ \forall n\in\mathbb{N}_0$
\end{enumerate}

% % % % % % % % %
% % % 12.3    % %
% % % % % % % % %
\subsection{Motivation}
Polynome sind besonders einfach zu handhaben.\\
Wir wollen komplizierte Funktionen möglichst gut mittels Polynome beschreiben / annähern.\\
Damit zwei Funktionen ,,ähnlich'' sind, sollten nicht nur ihre Funktionswerte in einigen Punkten übereinstimmen, sondern möglichst auch ihre Ableitung in diesen Punkten.


\subsection*{}
\textbf{gegeben:} Funktion $f: \ I\rightarrow \mathbb{R}, \ x_o\in I$\\
\textbf{gesucht:} Polynom $T_n(x)$ vom Grad n, das f gut annähert, insbesondere an der Stelle $x_0$.\\
Wie muss $T_n$ aussehen?
\begin{enumerate}
	\item[für $n=0$:]
	(Grad 0, d.h. $T_0(x)$ ist Gerade)\\
	$T_0(x)=f(x_0)$ (dann wenigstens Übereinstimmung in $x_0$):\\
	$T_0(x_0)f(x_0)$
	
	\item[für $n=1$:]
	$T_1(x)=f(x_0)+f'(x_0)\cdot (x-x_0)$ Polynome vom Grad 1 $\surd$\\
	$T_1(x_0)=f(x_0)+f'(x_0)$ (Übereinstimmung in $x_0$) $\surd$\\
	$T_1'(x)=f'(x_0)$\\
	$\Rightarrow T_1'(x_0)=f'(x_0)$ (Übereinstimmung der 1. Ableitung in $x_0$)
	
	\item[für $n=2$:]
	$t_2(x)=f(x_0)+f'(x_0)\cdot (x-x_0)+\frac{1}{2}\cdot f''(x_0)\cdot (x-x_0)^2$ Polynom vom Grad 2 $\surd$\\
	$T_2(x_0)=f(x_0)+0+0$ (Übereinstimmung in $x_0$) $\surd$\\
	$T_2'(x)=f'(x_0)+2\cdot \frac{1}{2}\cdot f''(x_0)\cdot (x-x_o)^1=f'(x_0)+f''(x_0)(x-x_0)$\\
	$T_2''(x)=f''(x_0)$\\
	$T_2'(x_0)=f'(x_0)$\\
	$T_2''(x_0)=f''(x_0)$ $T-2$ und f stimmen in 1. und 2. Ableitung an der Stelle $x_0$ überein.
\end{enumerate}

% % % % % % % % %
% % % 12.4    % %
% % % % % % % % %
\subsection{Definition: Taylorpolynom}
$F:I\rightarrow \mathbb{R}$ n-mal differenzierbar auf I, $x_0\in I$\\
Dann heißt \[T_n(x):=\sum_{k=0}^n \frac{f^{(k)}(x_0)}{k!}(x-x_0)^k\] das \textbf{n-te Taylorplynome von f}, entwickelt um den Punkt $x_0\in I$.\\
oben für $n=0,1,2$ gesehen:\\
Für $T_n(x)$ gilt: $T_n(x_n)=f(x_n)$ und $T_n^{(k)}(x_0)=f^{(k)}(x_0)$ für $k=1\dots n$

$T_n(x)$ nähert also f an. Wie gut?

% % % % % % % % %
% % % 12.5    % %
% % % % % % % % %
\subsection{Satz: Formel von Taylor mit Lagrange-Restglied}
$f:I\rightarrow \mathbb{R}$ (n-1)-mal differenzierbar auf I, $x_0\in I$\\
Sei $R_n(x):=f(x)-T_n(x)$\\
der Fehler zwischen f un dem n-ten Taylorpolynom von f entwickelt um den Punkt $x_0$. (''Restglied``)\\
Dann gibt es zu jedem $x\in I$ eine Stelle $\xi$ zwischen $x_0$ und $x$, so dass \[R_n(x)=\frac{f^{(n+1)}(\xi)}{n+1}\cdot (x-x_0)^{n+1}\]
(Merkregel: (n+1)-ter Term von $t_{n+1}(x)$ mit $\xi$ statt $x_0$)\\
also ist f darstellbar durch das n-te Taylorpolynom mittels \[f(x)=\underbrace{\sum_{k=0}^n \frac{f^{(k)}(x_0)}{k!}(x-x_0)^k}_{\text{Polynom vom Grad n}} + \underbrace{\frac{f^{(n+1)}(\xi)}{(n+1)!}}_{R_n(x)}(x-x_0)^{n+1}\]
(Taylorentwicklung von f an der Stelle $X_0$)

\textbf{Beweis:}\\
Sei $g(x)=(x-x_)^{n+1}$\\
Es gilt $R_n^{(k)}(x_0)=0$ und $g^{(k)}(x_0)=0 \ \forall k=0\dots n$\\
$\frac{R(x)}{g(x)}=\frac{R(x)-R(x_0)}{g(x)-g(x_0)} \stackrel{*}{=} \frac{R'(\xi_1)}{g'(\xi_1)}$ für ein $\xi_1$ zwischen $x$ und $x_0$ (*: 2. Mittelwertsatz aus Mathe II)\\
$=\frac{R'(\xi_1)-R'(x_0)}{g'(\xi_1)-g'(x_0)} \stackrel{*}{=} \frac{R''(\xi_2)}{g''(\xi_2)}$ für ein $\xi_2$ zw. $\xi_1$ und $x_0$\\
$=\dots = \frac{R^{(n+1)}(\xi_{n+1})}{g^{(n+1)}(\xi_{n+1})} = \frac{f^{(n+1)}(\xi_{n+1})}{(n+1)!}$ für ein $\xi_{n+1}$ zwischen $\xi_n$ und $x_0$\\
setze $\xi=\xi_{n+1}$, Behauptung folgt

% % % % % % % % %
% % % 12.6    % %
% % % % % % % % %
\subsection{Bemerkung}
\begin{enumerate}
	\item
	Der Satz besagt:\\
	$f(x)$ kann bis auf $R_n(x)$ als Polynom n-ten Grades dargestellt werden.\\
	Je größer n, desto besser sollte diese Annäherung sein. Insbesondere ist interessant: gilt $R_n(x)\rightarrow 0$ für $n\rightarrow \infty$?
	
	\item
	Es gibt auch andere Darstellungen des Restglieds, z.B: mit Integral.
\end{enumerate}

% % % % % % % % %
% % % 12.7    % %
% % % % % % % % %
\subsection{Beispiel}
\begin{enumerate}
	\item
	$f(x)=e, \ x_0=0$\\
	$f^{(k)}=e^x \ \forall k\in \mathbb{N}_0$\\
	$f^{(k)}(x_0)=e^0=1$\\
	$\Rightarrow T_n(x)=\sum_{k=0}^n \frac{f^{(k)}(0)}{k!}(x-0)^k = \sum_{k=0}^n \frac{1}{k!}x^k$\\
	$\Rightarrow \underbrace{e^x}{f(x)}=\underbrace{\sum_{k=0}^n \frac{x^k}{k!}}_{T_n(x)}+\underbrace{\frac{e^\xi}{(n+1)!}\cdot x^{n+1}}_{R_n(x)}$ ($\xi$ zwischen 0 und x)\\
	$e^\xi$ ist beschränkt durch $e^0$ oder $e^x$, $\frac{x^{n+1}}{(n+1)!}\rightarrow 0$ für $n\rightarrow \infty$
	
	Also:\\
	%
	\[e^x=\sum_{k=0}^\infty \frac{x^k}{k!} \ \ \forall x\in\mathbb{R}\]
	% 
\end{enumerate}