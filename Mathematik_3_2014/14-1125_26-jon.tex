%section 4
\section{Wiederholung und Erweiterung der linearen Algebra aus Mathe II}
%4.1
\subsection{Beispiel}

\begin{enumerate}
	\item
	$ K = \Z , V_1= \Z_2^2=\aset{
	\vct{x_1 \\ x_2} : x_1, x_2 \in \Z_2}$

	$V_1$ hat 4 Elemente: $
	\vct{0 \\0},\vct{0 \\ 1},\vct{1 \\ 0},\vct{1 \\ 1}$
	
	$\zerovec= \vct{0 \\ 0}, \vct{0 \\ 1} + \vct{0 \\ 1}= \vct{0 \\ 0},$ d.h. $-\vct{0 \\ 1} = \vct{0 \\ 1}, \vct{0 \\ 1} + \vct{1 \\ 1}=\vct{0 \\ 1}$
	 
	$\forall v \in V: 0 \cdot v = \zerovec = \vct{0 \\ 0}$ und $1\cdot v = v$
	\item
	$K = \Z_5, V_2=\Z_5^3=\aset{\vct{x_1 \\ x_2 \\ x_3}}$
	
	$v=\vct{0 \\ 1\\ 2 }, w = \vct{3 \\ 2 \\ 4} \in \Z_5^3$
	
	$-v= \vct{0 \\ 4 \\ 3}, -w = \vct{2 \\ 3 \\ 1}, v + w  \vct{3 \\ 2 \\ 1} $
	
	$1 \cdot w = w, 2 \cdot w = \vct{1 \\ 4 \\ 3}, 3 \cdot w = \cdots$
	
	$|V| = 5 \cdot 5 \cdot 5 = 125$
	\item 
	$U = \left\lbrace 
	\vct{x_1 \\ x_2} \in V_1: x_1 \oplus x_2 = 0 \right\rbrace$ ist UR von $V_1$
	\begin{itemize}
		\item
		$U = \left\lbrace 
		\vct{0 \\ 0}, \vct{1 \\ 1}  \right\rbrace \neq \emptyset$
		\item
		Sei $u= \vct{u_1 \\ u_2} \in U $, d.h. $u_1 \oplus u_2 = 0$ 
	\end{itemize}
	 $\Rightarrow$ f"ur $\lambda \cdot u= \vct{\lambda u_1 \\ \lambda u_2}$ gilt $\lambda u_1 \oplus \lambda u_2 = \lambda \cdot \underbrace{(u_1 \oplus u_2)}_{0}=0$ 
	 \item
	 $\Z_3^3:$
	 
	 $\vct{0 \\ 0 \\ 0}$ l.a.; $\vct{0 \\ 1 \\2}$ l.u.; $\vct{0 \\ 1 \\ 2}, \vct{0 \\ 2 \\ 1}$ sind l.a.
	 \item
	 Kanonische Basis von $V_2$ (Bsp. b)):
	 
	 $B_1 = \underbrace{\aset{e_1 = \vct{1 \\ 0 \\ 0}, e_2 = \vct{0 \\ 1 \\ 0}, e_3= \vct{0 \\ 0 \\ 1}}}_{\text{geordnete Basis}},\; \dim V_2 = 3$
	 
	 z.B.: $\vct{2 \\ 3 \\ 1}= \alpha \cdot e_1 + \beta \cdot e_2 + \gamma \cdot e_3$ mit $\alpha = 2, \;\beta = 3, \;\gamma =1 $ und $\alpha, \; \beta, \; \gamma $ sind die kartesischen Koordinaten.
	 
	 Eine andere (geordnete) Basis, z.B.:
	 
	 $B_2 = \aset {\vct{2 \\ 0 \\ 0}, \vct{0 \\ 1 \\ 1}, \vct{1 \\2 \\ 3}}$
	 
	 Zeige Vektoren sind linear unabh"angig:
	 
	 $\alpha \cdot \vct{2 \\ 0 \\ 0} + \beta \cdot \vct{0 \\  1\\ 1} + \gamma \cdot \vct{1 \\ 2 \\ 3} = \zerovec$
	 
	 $\Rightarrow  \cdots  \Rightarrow \cdots \Rightarrow \alpha = \beta = \gamma = 0$
	 
	 Koordinaten von $\vct{2 \\ 3 \\ 1}$ in $B_2$?
	 
	 Stelle LGS auf und l"ose es \dots
	 \end{enumerate}
	 % 4.2
	 \subsection{Definition}
	 
	 $A \in M_{n,n}(K)$ heißt \emph{invertierbar}, falls $\exists  A^{-1} \in M_{n,n}(K)$ mit $A^{-1} \cdot A = A \cdot A^{-1} = E_n$
	 
% % % % % % % % % %
% %  2014-11-25 % %
% % % % % % % % % %
%TODO Create another file for this code
%5
\section{Lineare Abbildungen}
%5.1
\subsection{Definition}

Seien $V,W$ $K$-Vektorr"aume.
\begin{enumerate}
	\item
	$\varphi: V \rightarrow W$ heißt \emph{lineare} Abbildung ($VR$-Homomorphismus), falls:
	\begin{itemize}
		\item
		$\forall v_1, v_2 \in V: \varphi(v_1+v_2) = \varphi(v_1) + \varphi(v_2)$ (Additivit"at)
		\item
		$\forall v \in V, \forall \lambda \in K: \varphi(\lambda \cdot v) = \lambda \cdot \varphi(v)$ (Homogenit"at)
	\end{itemize}
	\item
	Ist die lineare Abbildung $\varphi: V \rightarrow W$ bijektiv, so heißt $\varphi$ \emph{Isomorphismus}, $V$ und $W$ heißen dann \emph{isomorph}, $V \cong W$.
\end{enumerate}
%5.2
\subsection{Bemerkung}
$\varphi: V \rightarrow W$ ist eine lineare Abbildung:
\begin{enumerate}
	\item
	$\varphi( \zerovec) = \zerovec$
	\item
	$\varphi \left(\sum_{i=1}^{n} \lambda_i v_i \right) = \sum_{i=1}^{n} \lambda_i \varphi(v_i)$ 
\end{enumerate}
%5.3
\subsection{Beispiel}
\begin{enumerate}
	\item
	Nullabbildung:\\
	$\varphi: V \rightarrow W, v \mapsto \zerovec$
	\item
	$\varphi : V \rightarrow V, v \mapsto \lambda v$ f"ur jedes festes $\lambda \in K$ ist lineare Abbildung $(\lambda = 1: \varphi = \id_V )$
	\item
	$\varphi : \R^3 \rightarrow \R^3, \vct{x_1\\ x_2\\ x_3}\mapsto \vct{x_1 \\ x_2 \\ x_3}$ ist eine lineare Abbildung (Spiegelung an $x_1, x_2$-Ebene) 
	\item
	$\varphi : \R^ 2 \rightarrow R^2, \left(	\vct{x_1 \\ x_2} \mapsto \vct{(x_1)^2 \\ x_2 }\right)$ ist nicht linear
	
	$v= \vct{x_1 \\ x_2}, \lambda =3:$
	
	 $\varphi(3v)= \varphi\left(\vct{3 \\ 6}\right)=  \vct{9 \\ 6} \neq \vct{3 \\ 9} = 3 \cdot \vct{1\\2}= 3 \cdot \varphi \left(\vct{1\\2} \right) = 3 \cdot \varphi(v)$
	 \\

\end{enumerate}

% % % 5.4
\subsection{Satz}
\label{linAbb}

$A \in M_{m,n}(K)$

Dann ist $\varphi: K^n \rightarrow K^m$, $x \mapsto Ax$
%TODO: Zeichnung einfügen

eine lineare Abbildung \bigskip

% % %
\subsubsection*{Beweis}

folgt aus Rechenregeln für Matrizen:

\begin{align*}
\varphi(x+y) = A(x+y) &= Ax + Ay\\
&= \varphi(x) + \varphi(y)\\
\\
\varphi(\lambda \cdot x) = A(\lambda x) &= \lambda A x\\
&= \lambda \varphi(x)
\end{align*}
\qed

Alle bisherigen Beispiele waren von dieser Form!

5.3
\begin{enumerate}
\item $A = 0 = Nullmatrix$

\item $A =
\begin{pmatrix}
\lambda & \cdots & 0\\
\vdots & \ddots & \vdots\\
0 & \cdots & \lambda
\end{pmatrix} =
\lambda \cdot E_n
$
\medskip

\item $A =
\begin{pmatrix}
1 & \cdots & 0\\
\vdots & \ddots & \vdots\\
0 & \cdots & 1
\end{pmatrix}
$ 
\end{enumerate}

Es gilt ($\rightarrow$ später):

\underline{alle} lineare Abbildungen $K^n \rightarrow K^m$ sind von der Form in \ref{linAbb}
%TODO

% % % 5.5
\subsection{Satz}

$\varphi: V \rightarrow W$ lineare Abbildung

\begin{enumerate}[(i)]
	\item
	$U \subseteq V$ UR von $V$
	
	$\Rightarrow \varphi(U) \subseteq W$ UR von $W$ und $\varphi(V)$ (Bild von $V$) ist UR von $W$
	
	\item
	falls $\dim(U) {\text{ endlich}}: \dim(\varphi(U)) \leq \dim(U)$
\end{enumerate}

\subsubsection*{Beweis}

\begin{enumerate}[(i)]
	\item
	$U \subseteq V$ Unterraum, d.h.\ für $u,v \in U$ ist $\lambda u + \mu v \in U$
	
	$\varphi(U) = \{\varphi(u) \big | u \in U\}$ ist auch UR:
	
	für $\varphi(u), \varphi(v) \underset{\text{lin. Abb.}}{=} \varphi(\lambda u + \mu v) \in \varphi(U)$
	
	außerdem ist $\varphi(U) \neq \emptyset$, da $\varphi(\zerovec) = \zerovec$
	
	\item
	$v_1,\dots,v_k$ Basis von $U$
	
	$\Rightarrow \varphi(u_1),\cdots,\varphi(u_k)$ ist Erzeugendensystem von $\varphi(U)$
	
	$\Rightarrow$ enthält Basis (Mathe II)
	
	$\Rightarrow$ Behauptung
	\qed 
\end{enumerate}

% % % 5.6
\subsection{Definition}

$\varphi: V \rightarrow W$ lineare Abbildung, $V$ endlich dimensional

Dann heißt die $\dim(\varphi(V))$ der \underline{Rang von $\varphi$}, $\rg(\varphi)$.

% % % 5.7
\subsection{Definition/Satz}
	\label{kern}

$\varphi: V \rightarrow W$ lineare Abbildung

\begin{enumerate}[(i)]
	\item
	$\ker(\varphi) := \{v \in V\; \big |\; \varphi(v) = \zerovec\}$
	
	(alle Vektoren die von $\varphi$ auf $\zerovec$ abgebildet werden)
	
	heißt der \underline{Kern von $\varphi$} und ist ein UR von $V$.
	

	\item
	$\varphi:$ injektiv $\Leftrightarrow \ker(\varphi) = \{\zerovec\}$
\end{enumerate}

\subsubsection*{Beweis}

\begin{enumerate}[(i)]
	\item
	$\ker(\varphi)$ ist UR:
	
	\begin{itemize}
		\item
		$\ker(\varphi) \neq \emptyset$, da $\varphi(\zerovec) = \zerovec$
		
		\item
		seien $u,v \in \ker(\varphi)$, d.h. $\varphi(u) = \zerovec, \varphi(v) = \zerovec$, seien $\lambda, \mu \in K$
		
		$\Rightarrow \lambda u + \mu v \in \ker(\varphi)$, dann:
		
		$\varphi(\lambda u + \mu v) \underset{\text{lin. Abb.}}{=} \lambda \cdot \underbrace{\varphi(u)}_{\zerovec} +
		\mu \cdot \underbrace{\varphi(v)}_{\zerovec} = \zerovec$
	\end{itemize}
	
	\item
	"$\Rightarrow$"\\
	$\varphi(\zerovec) = \zerovec$, wegen Injektivität kann kein weiteres Element auf $\zerovec$ abgebildet werden.
	
	"$\Leftarrow$"\\
	Angenommen es gibt $v_1,v_2 \in V$ mit $\varphi(v_1) = \varphi(v_2)$, dann ist $\zerovec = \varphi(v_1) - \varphi(v_2) \\= \varphi(v_1 - v_2)$ (lineare Abbildung!)
	
	$\Rightarrow v_1 - v_2 = \zerovec$ (nur $\zerovec$ wird auf $\zerovec$ abgebildet)
	
	$\Rightarrow v_1 = v_2$
	
	$\Rightarrow \varphi$ injektiv
\end{enumerate}
\qed

% % % 5.8
\subsection{Beispiel}

$\varphi: \mathbb{R}^3 \rightarrow \mathbb{R}^3, \;
\begin{pmatrix}
x_1\\x_2\\x_3
\end{pmatrix} \rightarrow
\begin{pmatrix}
x_1\\2x_1\\x_1 + x_2 + 2x_3
\end{pmatrix}
\;$
 ist lineare Abbildung\bigskip

$
A =
\begin{pmatrix}
1 & 0 & 0 \\
2 & 0 & 0 \\
1 & 1 & 2 \\
\end{pmatrix}
$\bigskip

$U = \big <e_2,e_3\big >, \quad \dim(U) = 2$\bigskip

$\varphi(U),\; dim(\varphi(U)),\; ker(\varphi)$?\bigskip

$\varphi(U) = \big <\varphi(e_2), \varphi(e_3)\big > = \bigg <
\color{red}
\begin{pmatrix}
0\\0\\1
\end{pmatrix}\color{black},\color{blue}
\begin{pmatrix}
0\\0\\2
\end{pmatrix}\color{black}\bigg > = \bigg <
\begin{pmatrix}
0\\0\\1
\end{pmatrix}\bigg > = x_3$-Achse\bigskip

$\varphi(e_2) = \varphi(
\begin{pmatrix}
0\\1\\0
\end{pmatrix}) = \color{red}
\begin{pmatrix}
0\\0\\1
\end{pmatrix} \color{black} \quad
\varphi(e_3) = \varphi(
\begin{pmatrix}
0\\0\\1
\end{pmatrix}) = \color{blue}
\begin{pmatrix}
0\\0\\2
\end{pmatrix}\color{black}$\bigskip

$\dim(\varphi(U)) = 1$
