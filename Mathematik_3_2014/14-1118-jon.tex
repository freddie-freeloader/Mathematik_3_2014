% % % 3
\section{Der K"orper der $\C$ der Komplexen Zahlen} 

% % % 3.1
\subsection{Definition}
Eine komplexe Zahl $\Z$ ist von der Form $z=x+i\cdot y$ mit $x, y \in \R$ und einer ,,Zahl'' $i$ mit $i^2=-1$ (,,imagin"are Einheit''). $x$ heißt Realteil von $z$, $x=$Re $z$\\
$y$ heißt Imagin"arteil, $y=$Im $z$.

Die Menge aller komplexen Zahlen bezeichnen wie mit $\C$ und definieren auf $\C$ Addition und Multiplikation wie folgt:

F"ur $z=x+iy$ und $w=a+ib$ ist \\$z+w:= (x+a) + i (y+b)$, \\$z-w:= (x-a) + i (y-b)$ und \\$z\cdot w:= (xa-yb) + i(xb+ya).$

Erl"auterung zur Multiplikation: $((x+iy)(a+ib) = xa + xib + iya + i^2yb = (xa-yb) + i(xb+ya)$.

Mit diesen Verkn"upfungen ist $\C$ ein K"orper:
\begin{enumerate}
	\item
	AG, kG, DG: nachrechnen
	\item
	$0=0+i\cdot 0$
	\item
	additiv Inverses: $-z = -x-iy$
	\item
	$1=1+i\cdot 0$
	\item
	multiplikativ Inverses: $z^{-1}= \frac{1}{z}= \frac{1}{x+iy}=\frac{1}{x+iy}\cdot\frac{x-iy}{x-iy}= \frac{x-iy}{x^2+y^2}=\underbrace{\frac{x}{x^2+y^2}}_{\in \R}+ i \cdot \underbrace{\frac{-y}{x^2+y^2}}_{\in \R}$
\end{enumerate}
	Man nennt f"ur $z=x+iy$ die Zahl $ \overline{z} = x-iy$ die zu $z$ \emph{konjugiert komplexe Zahl} und $|z| := \sqrt{x^2+y^2}$ den \emph{Betrag} von $z$.

% % % 3.2
\subsection{Beispiel}
\begin{enumerate}
	\item
	$z= 2+3i$ mit Re$(z)=2$ und Im$(z)=3$.\\
	$\overline{z} = 2-3i, |z|=\sqrt{2^2+3^2}= \sqrt{13}$\\
	$z \cdot \overline{z}=(2+3i)\cdot(2-3i)\\
	=4-6i+6i-9i^2
	=4+9=13$
	\item
	$w= 1+i = 1+1 \cdot i:\;\;$ Re$(w)=1$, Im$(w)=1$, $\overline{w}= 1-i,\; |w|=\sqrt{1^2+1^2}=\sqrt{2}$
	\item
	Selbst nachrechnen: $u=7=7+0 \cdot i,\; v= 5i=0+5i$
	\item
	$u+w+z = 7+ (1+i) + (2+3i) = 10 + 4i$\\
	$u \cdot w = 7 \cdot (1+i) = 7+7i$\\ 
	$ \frac{w}{z}= \frac{1+i}{2+3i}= \frac{(1+i)\cdot(2-3i)}{4+9} = \frac{2-3i+2i=3i^2}{13}= \frac{5-i}{13}=\frac{5}{13}-\frac{1}{13}i$
\end{enumerate}

% % % 3.3
\subsection{Bemerkung: komplexe Zahlenebene}
%TODO Tikz-Action
Man kann $\C$ veranschaulichen in der ,,Gaußschen Zahlenebene'':\\
Betrachte $z=x+iy$ als Punkt $(x|y)$ in $\R^2$:
%TODO Some more tikzzzz 

% % % 3.4
\subsection{Satz (Eigenschaften)}


\begin{enumerate}
	\item %a
	$\left.
	\begin{array}{cl}
	\overline{w+z} &= \overline{w} + \overline{z}\\
	\overline{w\cdot z} &= \overline{w} \cdot \overline{z}\\
	\overline{\frac{w}{z}} &= \frac{\overline{w}}{\overline{z}} \quad (z \neq 0)\\
	\overline{\overline{z}} &= z
	\end{array}
	\right\rbrace \mathbb{C} \rightarrow \mathbb{C}, z \mapsto \overline{z}\; \text{ist Körperisomorphismus}$
	
	\item %b
	$Re(z) = \frac{z+\overline{z}}{2}, Im(z)= \frac{z - \overline{z}}{2i}$
	\item %c
	$|z| \geq 0,\; |z| = 0$ nur für $z = 0$
	\item
	$|z| = |\overline{z}| = \sqrt{z \cdot \overline{z}}$
	\item
	$|w \cdot z| = |w| \cdot |z|$
	\item
	$|w + z| \leq |w|+|z|$ (Dreiecksungleichung)
	
	$|w + z| \geq \bigg | |w|-|z| \bigg |$
\end{enumerate}

\subsubsection*{Beweis}

z.B.: d) sei $z = x + iy \quad x,y \in \mathbb{R}$

$\Rightarrow \overline{z} = x - iy, \quad |z| = \sqrt{x^2+y^2}$

$|\overline{z}|=\dots$

% % % 3.5
\subsection{Bemerkung}
\begin{enumerate}
	\item
	In $\C$ existiert $\sqrt{-1}: \pm i$, d.h. $x^2+1=0$ ist l"osbar in $\C$, das Polynom $x^2+1$ ist nicht irreduzibel in $\C[x]$: $ x^2+1 = (x+i)(x-i)$
	\item
	Mann kann jede quadratische Gleichung $ax^2+bx+c$ ($a,b,c\in \R$) in $\C$ l"osen:\\
	$x_{1|2} = \frac{-b \pm \sqrt{b^2-4ac}}{2a}$\\
	Jedes $b^2-4ac < 0$ ist, schreibe:\\
	$\frac{-b \pm \sqrt{4ac-b^2}\cdot i}{2a}$
	\item
	Es gilt sogar: Fundamentalsatz der Algebra:\\
	Jedes Polynom $f\in \C[x]$ vom Grad $n\geq 1$ hat genau $n$ Nullstellen in $\C$. 
\end{enumerate}